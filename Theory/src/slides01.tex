\documentclass[xcolor=dvipsnames]{beamer}

\usetheme[progressbar=frametitle,numbering=fraction,block=fill]{metropolis}
\usepackage{proof}
\usepackage{multirow,bigdelim}
\usepackage[russian]{babel}
\usepackage{minted}
\usepackage{libertinus}
\usefonttheme{serif}

\usepackage[matrix,arrow]{xy}
\usepackage{tikz}
\usetikzlibrary{shapes.geometric,patterns,positioning,matrix,calc,arrows,shapes,fit,decorations,decorations.pathmorphing}
\usepackage{pifont}

\newcommand{\adisj}{\vee}
\newcommand{\aconj}{\wedge}

\newcommand{\Kc}{\mathcal{K}}

\newcommand{\Ix}{\mathbf{I}}
\newcommand{\Yx}{\mathbf{Y}}
\newcommand{\Bx}{\mathbf{B}}

\newcommand{\Fx}{\mathbf{F}}
\newcommand{\Tx}{\mathbf{T}}

\newcommand{\ifxx}[3]{\bigl(\mathbf{if}\ {#1}\ \mathbf{then}\ {#2}\ \mathbf{else}\ {#3}\bigr)}

\newenvironment{mypic}
{\begin{center}\begin{tikzpicture}[line width=1.5pt]}
{\end{tikzpicture}\end{center}}


\newcommand{\BS}{\mathop{\backslash}}
\newcommand{\SL}{\mathop{/}}
\newcommand{\Pc}{\mathcal{P}}
\newcommand{\ACT}{\mathbf{ACT}}
\newcommand{\ACTomega}{\ACT_\omega}
\newcommand{\TM}{\mathfrak{M}}
\newcommand{\Gc}{\mathcal{G}}

\newcommand{\MALC}{\mathbf{MALC}}
\newcommand{\ILL}{\mathbf{ILL}}
\newcommand{\IAL}{\mathbf{IAL}}
\newcommand{\AMALC}{\mathbf{AMALC}}

\newcommand{\eL}{\mathbf{\boldsymbol{!}L}}
\newcommand{\rL}{\mathbf{\boldsymbol{!}^r L}}
\newcommand{\reL}{\mathbf{\boldsymbol{!}^{re} L}}

\newcommand{\exL}{\boldsymbol{!}_{\leqslant 1}\mathbf{L}}
\newcommand{\rxL}{\boldsymbol{!}_{\leqslant 1}^{\mathbf{r}} \mathbf{L}}
\newcommand{\rexL}{\boldsymbol{!}_{\leqslant 1}^{\mathbf{re}} \mathbf{L}}

\newcommand{\Dc}{\mathcal{D}}

\newcommand{\Int}{\mathbf{Int}}

\newcommand{\Factx}{\mathrm{Fact}}
\newcommand{\Prevx}{\mathbf{Prev}}

\newtheorem{theoremr}{Теорема}

\begin{document}

\title{Функциональное программирование}
\subtitle{Лекция 1}
\date{}
\author{Степан Львович Кузнецов}
\institute{НИУ ВШЭ, факультет компьютерных наук}

\maketitle

\begin{frame}{О курсе}

\begin{itemize}[<+->]
 \item Курс посвящён функциональной парадигме программирования на примере одного из наиболее известных функциональных языков --- Haskell.
 \item На лекциях: теоретические основы функционального программирования ($\lambda$-исчисление, выведение типов, монады-лимонады и проч.).
 \item На практических занятиях (Даниил Рогозин и Юрий Сыровецкий): программирование на Haskell'е.
 \item Теория и практика связаны между собой: как в шутку говорят, {\em Haskell --- это язык, в котором нельзя напечатать ``Hello, World'' без знания теории категорий.}
\end{itemize}

 
\end{frame}

\begin{frame}{Функции как объекты}

\begin{itemize}[<+->]
 \item Функциональная парадигма программирования существенно отличается от обычной (императивной).
 \item Мы постепенно будем обсуждать её особенности.
 \item {\bf Первое свойство,} объясняющее термин {\em <<функциональный>>:} функции являются полноправными <<гражданами>> (объектами) языка. Functions are first-class citizens.
 \item В частности, функция может быть передана как аргумент другой функции. В этом случае последняя называется {\em функцией высшего порядка.}
\end{itemize}

 
\end{frame}


\begin{frame}[fragile]{Функция как объект в C}
 
\begin{itemize}[<+->]
  \item Начнём с примеров функций высших порядков, которые встречаются в императивных языках.
  \item Так, в стандартной библиотеке C есть функция сортировки:
  \begin{minted}{c}
   void qsort(void *base, size_t nmemb, size_t size,
    int (*compar)(const void *, const void *));
  \end{minted}

  \item Эта функция может сортировать массив {\em произвольных данных.}
  \item Отсюда тип \mintinline{c}{void*} --- указатель на произвольный объект. (При этом не производится проверка корректности типов данных, что плохо.)
\end{itemize}

\end{frame}

\begin{frame}[fragile]{Функция как объект в C}
 
\begin{itemize}[<+->]
 \item Функция \mintinline{c}{compar} возвращает значение $<0$, если первый аргумент меньше второго, $=0$, если равны, и $>0$, если второй аргумент меньше.
 \item Например, для сравнения строк (\mintinline{c}{char*}) по алфавиту используется функция \mintinline{c}{strcmp} с соответствующим приведением типов:
 \begin{minted}{c}
int cmpstringp(const void *p1, const void *p2)
{
  return strcmp(*(const char **) p1, *(const char **) p2);
}
 \end{minted}
\item Технически передача функции как аргумента реализуется в C как передача указателя на место в памяти, где находится код этой функции --- так что сгенерировать новую функцию <<на лету>> не получится.
\end{itemize}

 
\end{frame}


\begin{frame}[fragile]{Функция как объект в Python'е}

\begin{itemize}[<+->]
 \item Аналогично устроена сортировка в Python'е:
\begin{minted}{python}
bigrams = {"AB": [10, 11, 12], "BC": [5, -5, 8],
  "CD": [105, 1, 0], "DE": [6, 6], "EF": [15, 20, 15], 
  "FG": [22, 11, 32], "GH": [20, 20, 20]}
srtbg = sorted(bigrams, key=lambda key: sum(bigrams[key]), 
  reverse=True)
\end{minted}
\item Здесь ради эффективности используется не функция сравнения, а функция вычисления ключа (которые потом сравниваются как целые числа).
\item Интересно использование ключевого слова \mintinline{python}{lambda} для создания безымянной функции <<на месте>>.
\end{itemize}
 
\end{frame}

\begin{frame}{$\lambda$-оператор}

\begin{itemize}[<+->]
 \item Знаком $\lambda$ выделяется та переменная, которую мы будем считать аргументом функции.
 \item В теоретическом материале мы будем использовать обозначение $\lambda x. u$, где $u$ --- выражение ({\em терм}), возможно содержащее $x$:
 \[
  \lambda x . \underbrace{\mbox{\fbox{ $\ldots x \ldots x \ldots x \ldots$}}}_u
 \]
 \item При вычислении значения функции $\lambda x . u$ на аргументе $x = a$ нужно подставить $a$ вместо $x$ вместо всех {\em свободных} (т.е. не связанных другими $\lambda$'ми) вхождений $x$ в $u$.
 \item Это называется {\em $\beta$-преобразованием,} о нём мы поговорим позже.
 \item В математике вместо $\lambda x .u $ пишут $x \mapsto u$.
\end{itemize}


 
\end{frame}

\begin{frame}[fragile]{Функции и изменяемость}

\begin{itemize}[<+->]
 \item Посмотрим на следующий код:
\begin{minted}{python}
x=5
f=lambda y : y+x
print f(2)
x=7
print f(2)
\end{minted}
\item Значение \mintinline{python}{f(2)} изменилось: действительно, \mintinline{python}{lambda} создаёт новую безымянную функцию, которая, помимо своего аргумента \mintinline{python}{y} имеет также неявный доступ к переменной \mintinline{python}{x}.
\end{itemize}

 
\end{frame}


\begin{frame}[fragile]{Функции и изменяемость}
 
 \begin{itemize}[<+->]
  \item Таким образом, в Python'е функции, введённые с помощью \mintinline{python}{lambda}, ведут себя всё же не совсем как обычные объекты.
  \item Действительно, если бы вместо \mintinline{python}{f} была бы числовая переменная:
\begin{minted}{python}
x=5
f=x+2
print f
x=7
print f
\end{minted}
то значение бы не поменялось.
 \end{itemize}

\end{frame}

\begin{frame}{Функции и изменяемость}

\begin{itemize}[<+->]
 \item Во многих функциональных языках (в частности, в Haskell'е) проблема с изменением значений решена радикально.
 \item В этом состоит {\bf второе свойство} --- immutability: значения переменных вообще запрещено изменять!
 \item Это свойство выглядит довольно дико, принуждая к созданию большого числа объектов вместо изменений одного. Однако этот негативный эффект компенсируется сборкой мусора и оптимизацией.
 \end{itemize}
\end{frame}

\begin{frame}{Неизменяемость}
\begin{itemize}[<+->]
 \item За счёт неизменяемости функции получаются {\em чистыми} в математическом смысле: возвращаемое значение однозначно определяется значениями аргументов, и при этом функция не имеет {\em побочных эффектов.}
 \item В императивных языках, наоборот, функции взаимодействуют с неким {\em состоянием внешнего мира.}
 \item Это делает осмысленными, в частности, функции, которые ничего не принимают и не возвращают:
 \mintinline{c}{void func();}
 \item Конечно, связь с внешним миром нужна, но в <<чистых>> функциональных языках она прописывается явно.
 \item В Haskell'е для этого (в частности --- для ввода-вывода) используется механизм {\em монад,} основанный на теоретико-категорной конструкции.
\end{itemize}
\end{frame}

\begin{frame}{Вычисление как преобразование}

\begin{itemize}[<+->]
 \item Наконец, ещё более фундаментальное {\bf третье свойство,} отличающее функциональные языки от императивных, заключается в самом понятии вычислительного процесса.
 \item В функциональной парадигме вычисление есть последовательное {\em преобразование} некоего выражения ({\em терма}), пока он не дойдёт до некоторой далее не преобразуемой формы. (Например, терм, в явном виде представляющий натуральное число.)
 \item Преобразования призваны <<упрощать>> терм, и поэтому также называются {\em редукциями.}
 \item В реальности редукции не всегда упрощают терм, и возможны бесконечные их последовательности (что соответствует неостанавливающейся программе на императивном языке).
\end{itemize}

 
\end{frame}

\begin{frame}{Вычисление как преобразование}

\begin{itemize}[<+->]
 \item В этом смысле исполнение функциональной программы напоминает вычисление арифметического выражения:
 \[
  (1+2) \cdot (3+4) \to 
  3 \cdot (3+4) \to 3 \cdot 7 \to 21.
 \]
\item При этом, в отличие от императивной программы, порядок преобразований не задан жёстко:
\[
 (1+2) \cdot (3+4) \to 
 (1+2) \cdot 7 \to 3 \cdot 7 \to 21.
\]

\item Преобразование можно применить к любому подвыражению (подтерму), которое может быть упрощено. Такой подтерм называется {\em редексом.}
\end{itemize}


 
\end{frame}


\begin{frame}{Конфлюэнтность}

\begin{itemize}[<+->]
\item Если синтаксис разработан неправильно, то разные последовательности редукций могут давать разные ответы. Например, так получится, если не использовать скобки:
\[
\xymatrix@-1em{
1+2 \cdot 3+4 \ar[r]\ar[dd] & 3 \cdot 3 + 4 \ar[r]\ar[d] & 3 \cdot 7 \ar[r] & 21\\
& 9 + 4 \ar[r] & 13\\
1+6+4 \ar[r] & 7+4 \ar[r] & 11
}
\]
\item В <<хороших>> системах этого не происходит за счёт {\em конфлюэнтности (свойства Чёрча -- Россера):}
если $u \twoheadrightarrow v_1$ и $u \twoheadrightarrow v_2$, то существует такой терм $w$, что $v_1 \twoheadrightarrow w$ и $v_2 \twoheadrightarrow w$.
\end{itemize}

 
\end{frame}

\begin{frame}{Ленивость}

\begin{itemize}[<+->]
 \item За счёт порядка преобразований какие-то подтермы могут оказаться вообще не вычисленными.
 \item Например, $\mathrm{length} [ u,v,w ]$ можно сразу редуцировать к $3$, не пытаясь вычислить значения $u$, $v$, $w$.
\item Это свойство называется {\em ленивостью} вычислений.
 \end{itemize}

 
\end{frame}


\begin{frame}{$\lambda$-исчисление}

\begin{itemize}
 \item $\lambda$-исчисление --- простейшая модель и основа функциональных языков программирования.
 \item<2-> Термы $\lambda$-исчисления ({\em $\lambda$-термы}) строятся из переменных с помощью всего лишь двух операций:
 \begin{itemize}
 \item {\em применение:} если $u$ и $v$ --- термы, то $(uv)$ --- терм;
 \item {\em $\lambda$-абстракция:} 
 если $u$ --- терм, $x$ --- переменная, то
 $\lambda x . u$ --- терм.
 \end{itemize}
 \item<3-> Запись $(uv)$ означает применение функции $u$ к  $v$. 
 \item<4-> Более привычное обозначение было бы $u(v)$, однако бесскобочное обозначение также применяется в математике --- например, $\sin \alpha$.
 \item<5-> В функциональных языках чаще используется бесскобочная запись.
\end{itemize}

 
\end{frame}

\begin{frame}{Функции многих аргументов}

\begin{itemize}[<+->]
 \item С помощью $\lambda$-абстракции можно задать функцию {\em одного} аргумента $x$. Как быть с функциями многих аргументов?
 \item Для этого используется приём, называемый {\em каррированием} (в честь Х.\,Карри): $f = \lambda x. \lambda y.\lambda z. u$.
 \item В каррированном виде функция $f$ является функцией одного аргумента ($x$), возвращающая, в свою очередь, опять же функцию одного аргумента ($y$) и т.д.
 \item Для каррированных функций многих аргументов используется сокращённое обозначение $\lambda xyz. u$.
\end{itemize}

 
\end{frame}

\begin{frame}[fragile]{Примеры и бестиповость}
 
 \begin{itemize}[<+->]
  \item Простейший пример $\lambda$-терма: $\Ix = \lambda x.x$. Этот терм реализует тождественную функцию:
\begin{minted}{python}
def identity(x):
  return x
\end{minted}
  \item Отметим, что наше $\lambda$-исчисление (как и Python) {\em бестиповое:} любой терм можно применить, как функцию, к любому другому.
  
  \item Более содержательный пример --- абстрактная программа для композиции функций
  \[
   \Bx = \lambda f g x . f(gx).
  \]

 \end{itemize}

\end{frame}

\begin{frame}{Преобразования $\lambda$-термов}

\begin{itemize}[<+->]
 \item Главное преобразование термов в $\lambda$-исчислении --- {\em $\beta$-редукция:}
 \[
  (\lambda x . u) v \to_\beta
  u[x := v].
 \]
 \vspace*{-1em}
 \begin{itemize}
 \item Запись $u[x:=v]$ означает подстановку $v$ вместо каждого свободного вхождения $x$ в $u$.
 \item Условие корректности подстановки: переменные, свободные в $v$, не должны оказаться связанными в $u$. (Например,
 $(\lambda x. \lambda y. x) y \mathop{\not\to_\beta} \lambda y .y$.)
 \end{itemize}
 \item $\beta$-редукция может применяться к произвольному редексу вида $(\lambda x .u)v$:
 \[
  \mbox{\fbox{$\quad\ldots\quad (\lambda x.u)v \quad \ldots\quad$}} \to_\beta
  \mbox{\fbox{$\quad\ldots\quad u[x:=v] \quad \ldots\quad$}}
 \]

\end{itemize}

 
\end{frame}

\begin{frame}{Преобразования $\lambda$-термов}

\begin{itemize}[<+->]
 \item Помимо $\beta$-редукции имеется вспомогательное преобразование --- {\em $\alpha$-конверсия:}
 \[
  \lambda x. u \to_\alpha \lambda y. u[x := y],
 \]
 где $y$ --- новая переменная.

 \item Термы, которые можно свести к одному и тому же $\alpha$-конверсиями, называются {\em $\alpha$-равными} и в дальнейшем считаются вариантами одного терма.
 
 \item $\alpha$-конверсия помогает решить проблему с недопустимой подстановкой при $\beta$-редукции:
 \[
  (\lambda x .\lambda y. x) y =_\alpha 
  (\lambda x . \lambda z. x) y \to_\beta
  \lambda z. y
 \]

\end{itemize}

 
\end{frame}

\begin{frame}{Нормализация}

\begin{itemize}[<+->]
 \item {\em Нормальная форма} --- это терм, в котором нет $\beta$-редексов (т.е. который далее нельзя редуцировать).
 
 \item {\bf Теорема.} $\beta$-редукция обладает свойством Чёрча -- Россера:
 \[
  \xymatrix{
   & v_1 \ar@{-->>}[dr] & \\
  u \ar@{->>}[ur]\ar@{->>}[dr] & & w \\
  & v_2 \ar@{-->>}[ur]
  }
 \]

 \begin{itemize}
 \item Доказательство этой, как и некоторых других, теорем, будет опубликовано в виде конспекта.
 \end{itemize}
 
 
 \item {\bf Следствие.} Данный терм не может редуцироваться к двум $\alpha$-разным нормальным формам.
\end{itemize}

 
\end{frame}

\begin{frame}{Нормализация}

\begin{itemize}[<+->]
 \item Поскольку нормальная форма не зависит от пути, которым мы к ней пришли, её можно считать {\em результатом вычисления значения} данного $\lambda$-терма.
 
 \item Однако всё не так просто.
 
 \item Бывают термы, которые вообще не приводятся к нормальной форме (любое вычисление бесконечно).
 
 \item Бывают и такие, для которых один путь приводит к нормальной форме ({\em слабая нормализуемость}), а другой бесконечен.
 
 \item Наконец, если все пути приводят к нормальной форме, то такой терм {\em сильно нормализуем.}
\end{itemize}

 
\end{frame}

\begin{frame}{Примеры}

\begin{itemize}
 \item Пусть $\omega = \lambda x. (xx)$, а $\Omega = \omega\omega$. Тогда $\Omega$ редуцируется только сам к себе: 
 \[\Omega = (\lambda x.(xx)) (\lambda x. (xx)) \to_\beta (xx)[x := \omega] = \omega\omega = \Omega,\]
 значит, он не нормализуем.
 \item Можно построить и терм, который будет при <<редукции>> бесконечно разрастаться.
 \item Бывает и слабо нормализуемый терм, не являющийся сильно нормализуемым: например, $(\lambda x . y) \Omega$.
\end{itemize}

 
\end{frame}


\begin{frame}{Нормализуемость}

\begin{itemize}[<+->]
 \item Из-за существования слабо, но не сильно нормализуемых термов важен порядок, или {\em стратегия,} применения редукций.
 \item О различных стратегиях редукций мы поговорим на следующей лекции.
 \item А пока что коротко обсудим вычислительные возможности бестипового $\lambda$-исчисления.
\end{itemize}

 
\end{frame}


\begin{frame}{Натуральные числа по Чёрчу}

\begin{itemize}[<+->]
 \item Натуральное число $n$ можно представить следующим образом с помощью константы $o$ (ноль) и функции $s$ (взятие следующего):
 \[
  \underbrace{s(s \ldots (s}_{\text{$n$ раз}}
  o) \ldots)
 \]

 \item В <<чистом>> $\lambda$-исчислении у нас нет констант, поэтому мы просто абстрагируем $s$ и $o$ как переменные, получив замкнутый (без свободных переменных) терм, называемый {\em нумералом Чёрча:}
 \[
  \underline{n} = \lambda s o. \underbrace{s(s \ldots (s}_{\text{$n$ раз}}
  o) \ldots)
 \]

\end{itemize}

 
\end{frame}

\begin{frame}{Представление функций}

\begin{itemize}[<+->]
 \item Заметим, что нумералы Чёрча не содержат $\beta$-редексов, т.е. являются нормальными формами.
 
 \item Таким образом, можно считать, что некий $\lambda$-терм $F$ является программой, вычисляющей $k$-местную функцию $f$ на натуральных числах, если 
 \[F \, \underline{n_1} \ldots \underline{n_k}
 \twoheadrightarrow_\beta 
 \underline{f(n_1, \ldots, n_k)}\]
 
 \item Здесь, в соответствии с функциональной парадигмой, процесс {\em вычисления} значения функции $f$ представляется в виде {\em редукции} терма $F \, \underline{n_1} \ldots \underline{n_k}$.
 
 \item В силу конфлюэнтности, результат вычисления определяется однозначно.
\end{itemize}

 
\end{frame}

\begin{frame}{Представление функций}

\begin{itemize}[<+->]
 \item Однако возможна ситуация слабой нормализуемости, при которой мы можем пойти по <<неправильному>> пути и не достичь нормальной формы (которая при этом существует).
 
 \item Бороться с этим нужно выбором правильной {\em стратегии нормализации,} о чём мы поговорим на следующей лекции.
 
 \item Короткий ответ: если нормальная форма существует, то её можно достичь, всегда редуцируя {\em самый левый} (считая по начальной $\lambda$'е) $\beta$-редекс.
\end{itemize}


\end{frame}


\begin{frame}{Представление функций}

\begin{itemize}
 \item На нумералах Чёрча легко определить операции сложения и умножения:
 \begin{align*}
 & \underline{n} + \underline{m} = 
 \lambda s o. (\underline{n} s)(\underline{m} s o);
 \\
 & \underline{n} \cdot \underline{m} = 
 \lambda s o. \underline{m}(\underline{n} s) o. 
 \end{align*}

 \item Абстрагируя, получаем термы для (двуместных) функций сложения и умножения:
 \begin{align*}
  & \boldsymbol{+} = \lambda xyso. (xs)(yso);\\
  & \boldsymbol{\cdot} = \lambda xyso. x(ys)o.
 \end{align*}

 \item {\bf Задача.} Задайте $\lambda$-термом функцию <<предшественник>>:
 \[
  \mathrm{Prev}(n) = \left\{
  \begin{aligned}
   & 0, && \mbox{если $n = 0$;}\\
   & n-1, && \mbox{если $n > 0$.}
  \end{aligned} \right.
 \]

\end{itemize}

 
 
\end{frame}

\begin{frame}{Представление функций}
 
 \begin{itemize}[<+->]
  \item На самом деле, $\lambda$-термы умеют намного больше, чем сложение и умножение: с их помощью можно записать {\bf любую алгоритмически вычислимую} функцию на натуральных числах.
  \item При этом функция может быть не всюду определённой --- тогда на соответствующих значениях аргументов терм $F\, \underline{n_1} \ldots \underline{n_k}$ будет ненормализуемым.
  \item Мы обсуждаем такой <<низкоуровневый>> язык, как $\lambda$-исчисление, чтобы не перегружать изложение синтаксическими деталями.
  \item Можно сказать, что всё остальное в функциональных языках --- надстройка для удобства, <<синтаксический сахар>>.
 \end{itemize}

\end{frame}


\begin{frame}{Булевы операции}
 
 \begin{itemize}[<+->]
  \item В $\lambda$-исчислении можно ввести константы <<истина>> и <<ложь>> как функции выбора из двух аргументов:
  \[
   \Tx = \lambda t. \lambda f. t; \qquad
   \Fx = \lambda t. \lambda f. f.
  \]

  \item Условный оператор:
  \(
   \ifxx{b}{u}{v} = b\,u\, v.
  \)

  \item Логические операции:
  \begin{align*}
& (b_1 \mathop{\mathbf{and}} b_2) = 
\ifxx{b_1}{\ifxx{b_2}{\Tx}{\Fx}}{\Fx}\\
& \ldots
  \end{align*}

  \item Проверка на ноль: 
  $\mathbf{Zero} = \lambda x. (x \, (\lambda z. \Fx) \, \Tx)$.
 \end{itemize}

 
\end{frame}


\begin{frame}{Рекурсия}

\begin{itemize}[<+->]
 \item Чтобы достичь полноты по Тьюрингу, осталось реализовать {\bf рекурсию} (которая в функциональных языках используется повсеместно, в т.ч. вместо циклов).
 
 \item Пример: факториал $f(n) = n! = 1 \cdot 2 \cdot \ldots \cdot n$.
 
 \item Рекурсивная реализация:
 \[
  \Factx = \lambda x. 
  \ifxx{\mathbf{Zero}\ x}{\underline{1}}{(\Factx\,(\Prevx\ x) \cdot x)}
 \]

 \item Проблема: $\Factx$ определяется через самоё себя.
 
 \item С помощью $\lambda$-абстракции можно сделать зависимость в правой части явной (функциональной):
 \[
  \Factx = \underbrace{\bigl(\lambda g. \lambda x. 
  \ifxx{\mathbf{Zero}\ x}{\underline{1}}{(g\,(\Prevx\ x) \cdot x)} \bigr)}_F \, 
  \Factx
 \]

\end{itemize}

 
\end{frame}



\begin{frame}{Рекурсия}

\begin{itemize}[<+->]
 \item Чтобы реализовать рекурсивно определённую функцию, используется {\em комбинатор неподвижной точки} ($\Yx$-комбинатор, или комбинатор Карри) со следующим свойством: $\Yx\, F =_\beta F(\Yx\,F)$.
 \item $\Yx = \lambda f. \bigl(
 (\lambda x. f(xx)) (\lambda x. f(xx)) \bigr)$
 \item Имеем $\Yx\, F \to_\beta Y_F = 
 (\lambda x. F(xx)) (\lambda x. F(xx))$, при этом $Y_F$ --- неподвижная точка для $F$: $Y_F \to_\beta F(Y_F)$.
 \item Терм, использующий $\Yx$-комбинатор, никогда не будет сильно нормализуемым:
 $Y_F \to_\beta F(Y_F) \to_\beta F(F(Y_F)) \to_\beta \ldots$
 \item Однако если разбирать $F$, то процесс может сойтись: например, $\Factx \, \underline{n} \twoheadrightarrow_\beta \underline{n!}$
\end{itemize}

 
 
\end{frame}


\begin{frame}{Рекурсия}
 \[\hspace*{-1em}
  \Factx\, \underline{0} \to_\beta
  Y_F \, \underline{0} \to_\beta F \, Y_F \, \underline{0} \twoheadrightarrow_\beta 
  \ifxx{\mathbf{Zero}\, \underline{0}}{\underline{1}}{(Y_F \, (\Prevx\ \underline{0})) \cdot \underline{0}} \twoheadrightarrow_\beta \underline{1}
 \]
 
 \visible<2->{
 \begin{multline*}
  \Factx\, \underline{n+1} \to_\beta
  Y_F \, \underline{n+1} \to_\beta
  F \, Y_F \, \underline{n+1} 
  \twoheadrightarrow_\beta\\
  \twoheadrightarrow_\beta
  \ifxx{\mathbf{Zero}\, \underline{n+1}}
  {\underline{1}}
  {(Y_F \, (\Prevx\ \underline{n+1})) \cdot \underline{n+1}} 
  \twoheadrightarrow_\beta\\
  \twoheadrightarrow_\beta
  (Y_F  \, \underline{n}) \cdot \underline{n+1}
 \end{multline*}

 }

 
\end{frame}


\begin{frame}{Рекурсия}
 
 \begin{itemize}[<+->]
  \item Если рекурсивное определение <<плохое>> (например, забыто $\Prevx$), то терм будет ненормализуемым: {\bf любая} последовательность редукций бесконечна.
  \item Статически проверить это невозможно, поскольку задача останова алгоритмически неразрешима.
  \item $\Yx$ --- не единственный комбинатор неподвижной точки. Таков, например, также {\em комбинатор Тьюринга}
  $\boldsymbol{\Theta} = 
  (\lambda xy. y(xxy)) (\lambda xy. y(xxy))$
  \item ... и даже $??????????????????????????$, где\\
  \mbox{$? = \lambda abcdef ghijklmnopqstuvwxyzr.r(thisisaf ixedpointcombinator)$}
 \end{itemize}

\end{frame}

\begin{frame}[fragile]{Y-комбинатор в Python'е}

\begin{itemize}[<+->]
 \item Терм, содержащий $\Yx$-комбинатор, не будет корректным, если следить за типами данных. Действительно, он содержит $xx$, значит, переменная $x$ должна одновременно быть некоторого типа $A$ и типа функции $A \to B$.
 \item Тем не менее, в бестиповом языке, таком как Python, $\Yx$-комбинатор можно реализовать.
 \item Наивная попытка:
\begin{minted}{python}
Y = lambda f : ((lambda x : f(x(x))) (lambda x : f(x(x))))
fact = Y (lambda g : lambda n : (n and n * g(n-1)) or 1)
\end{minted}
\item Не работает: ``maximum recursion depth exceeded''. Python использует не тот порядок вычислений и уходит в бесконечное вычисление.
\end{itemize}

 
\end{frame}

\begin{frame}[fragile]{Y-комбинатор в Python'е}
 
 \begin{itemize}
  \item Положение можно исправить, заменив $xx$ на $\lambda z . xxz$:
  \end{itemize}
  \vspace*{-1.5em}
  {\footnotesize
  \hspace*{-2em}
\begin{minted}{python}
Y = lambda f : ((lambda x : f(x(x))) (lambda x : f(lambda z : x(x)(z))))   
 \end{minted}
}
 
 
 \begin{itemize}
  \item<2-> Математически $h$ и $\lambda z. hz$ эквивалентны (если $h$ не зависит от $z$), однако $\beta$-редукцией друг к другу не сводятся --- это {\em $\eta$-эквивалентность.}
  \item<3-> Вычисление откладывается до тех пор, пока \mintinline{python}{lambda z : x(x)(z)} окажется к чему-то применено.
  \item<4-> Теперь всё работает: например, \mintinline{python}{fact(6)} даёт 720.
 \end{itemize}


 
\end{frame}

\begin{frame}{Y-комбинатор в Python'е}

\begin{itemize}[<+->]
 \item {\bf Упражнение.} Реализуйте на Python'е комбинатор Тьюринга $\boldsymbol{\Theta} = 
  (\lambda xy. y(xxy)) (\lambda xy. y(xxy))$ (с соответствующим $\eta$-преобразованием).
 \item {\bf Упражнение.} Верно ли, что для комбинатора $\widetilde{\Yx} = \lambda f. (\lambda x. f(xx)) (\lambda x. f(\lambda z. xxz))$ и терма $F$ из определения факториала терм $\widetilde{\Yx} F$ сильно нормализуем?
\end{itemize}

 
\end{frame}



\end{document}

