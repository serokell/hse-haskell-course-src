\documentclass[xcolor=dvipsnames]{beamer}

\usetheme[progressbar=frametitle,numbering=fraction,block=fill]{metropolis}
\usepackage{proof}
\usepackage{multirow,bigdelim}
\usepackage[russian]{babel}
\usepackage{minted}
\usepackage{amssymb,amsmath}
\usepackage{libertinus}
\usefonttheme{serif}

\usepackage[matrix,arrow]{xy}
\usepackage{tikz}
\usetikzlibrary{shapes.geometric,patterns,positioning,matrix,calc,arrows,shapes,fit,decorations,decorations.pathmorphing}
%\usepackage{pifont}
\setmathfont{Asana Math}[range={\cup}]

\newcommand{\bind}{\mathop{\mbox{\tt {>}>=}}}

\newcommand{\Rc}{\mathcal{R}}
\definecolor{lightblue}{HTML}{ADD8E6}

\newcommand{\letin}[2]{\mathbf{let}\ {#1}\ \mathbf{in}\ {#2}}

\newcommand{\NN}{\mathbb{N}}
\newcommand{\One}{\mathbf{1}}

\newcommand{\Int}{\mbox{\texttt{Int}}}
\newcommand{\Bool}{\mbox{\texttt{Bool}}}
\newcommand{\Char}{\mbox{\texttt{Char}}}


\newcommand{\Cx}{\mathbf{C}}

\newcommand{\Cc}{\mathcal{C}}

\newcommand{\Sc}{\mathcal{S}}
\newcommand{\Yr}{\mathrm{Y}}


\newcommand{\Ix}{\mathbf{I}}
\newcommand{\Yx}{\mathbf{Y}}
\newcommand{\Bx}{\mathbf{B}}
\newcommand{\Yb}{\mathbb{Y}}

\newcommand{\Fx}{\mathbf{F}}
\newcommand{\Tx}{\mathbf{T}}
\newcommand{\Kx}{\mathbf{K}}

\newcommand{\adisj}{\vee}
\newcommand{\aconj}{\wedge}

\newcommand{\Kc}{\mathcal{K}}



\newcommand{\ifxx}[3]{\bigl(\mathbf{if}\ {#1}\ \mathbf{then}\ {#2}\ \mathbf{else}\ {#3}\bigr)}

\newenvironment{mypic}
{\begin{center}\begin{tikzpicture}[line width=1.5pt]}
{\end{tikzpicture}\end{center}}


\newcommand{\BS}{\mathop{\backslash}}
\newcommand{\SL}{\mathop{/}}
\newcommand{\Pc}{\mathcal{P}}
\newcommand{\ACT}{\mathbf{ACT}}
\newcommand{\ACTomega}{\ACT_\omega}
\newcommand{\TM}{\mathfrak{M}}
\newcommand{\Gc}{\mathcal{G}}

\newcommand{\MALC}{\mathbf{MALC}}
\newcommand{\ILL}{\mathbf{ILL}}
\newcommand{\IAL}{\mathbf{IAL}}
\newcommand{\AMALC}{\mathbf{AMALC}}

\newcommand{\eL}{\mathbf{\boldsymbol{!}L}}
\newcommand{\rL}{\mathbf{\boldsymbol{!}^r L}}
\newcommand{\reL}{\mathbf{\boldsymbol{!}^{re} L}}

\newcommand{\exL}{\boldsymbol{!}_{\leqslant 1}\mathbf{L}}
\newcommand{\rxL}{\boldsymbol{!}_{\leqslant 1}^{\mathbf{r}} \mathbf{L}}
\newcommand{\rexL}{\boldsymbol{!}_{\leqslant 1}^{\mathbf{re}} \mathbf{L}}

\newcommand{\Dc}{\mathcal{D}}


\newcommand{\Factx}{\mathrm{Fact}}
\newcommand{\Prevx}{\mathbf{Prev}}

\newtheorem{theoremr}{Теорема}

\definecolor{lightcyan}{HTML}{E0FFFF}

\begin{document}

\title{Функциональное программирование}
\subtitle{Лекция 6}
\date{}
\author{Степан Львович Кузнецов}
\institute{НИУ ВШЭ, факультет компьютерных наук}


\maketitle

\begin{frame}{Функторы}
 
 \begin{itemize}[<+->]
  \item Функторы и, как частный случай, монады --- это {\em преобразователи типов.} В Haskell'е реализуются как параметрические классы типов.
  \item Функтор позволяет <<поднимать>>, с помощью {\tt fmap}, функции {\tt a -> b} до {\tt f a -> f b} (математически: $h \colon X \to Y$ преобразуется в $F h \colon FX \to FY$).
  \begin{itemize}
  \item При этом {\tt fmap} согласована с композицией и тождественным отображением.
  \item Проверка этого на совести автора реализации!
  \end{itemize}
 \end{itemize}

\end{frame}

\iffalse 

\begin{frame}{Классы типов}
 
 \begin{itemize}[<+->]
  \item Классы типов {\em (type classes)} служат для реализации ad hoc полиморфизма.
  \item Принадлежность типа некоторому классу влечёт необходимость реализации для этого типа функций, свойственных данному классу.
  \item При этом, в отличие от классов в ООП, сам класс типов абстрактен, и содержит только объявления функций (<<методов>>) класса.
  \begin{itemize}
    \item Таким образом, класс в смысле ООП будет состоять из класса типов и его конкретной реализации (как правило, алгебраического типа).
    \item Есть и {\em наследование} (абстрактных) классов типов: класс-наследник расширяет исходный класс новыми функциями.
  \end{itemize}
 \end{itemize}

\end{frame}


\begin{frame}[fragile]{Классы типов}
 
 \begin{itemize}[<+->]
  \item Пример: класс <<числовых>> типов \mintinline{haskell}{Num}.
{\scriptsize
\begin{minted}{haskell}
type Num :: * -> Constraint
class Num a where
  (+) :: a -> a -> a
  (-) :: a -> a -> a
  (*) :: a -> a -> a
  negate :: a -> a
  abs :: a -> a
  signum :: a -> a
  fromInteger :: Integer -> a
  {-# MINIMAL (+), (*), abs, signum, fromInteger, (negate | (-)) #-}
    -- Defined in ‘GHC.Num’
\end{minted}
}
\begin{itemize}
  \item Следующие типы реализуют этот класс: \mintinline{haskell}{Int},
  \mintinline{haskell}{Integer} (``big int''), \mintinline{haskell}{Float}, \mintinline{haskell}{Double}
\end{itemize}
 \item {<<Тип>>} для \mintinline{haskell}{Num}, равный \mintinline{haskell}{* -> Constraint}, это {\em сорт (kind).}
 \item Если <<применить>> \mintinline{haskell}{Num} к переменной по типам \mintinline{text}{a}, то получится ограничение {\em (constraint):} \mintinline{haskell}{Num a => ...}
 \end{itemize}

\end{frame}


\begin{frame}[fragile]{Классы типов}
 
 \begin{itemize}[<+->]
  \item Ограничения учитываются при выведении типов.
  \item Это происходит одновременно с унификацией: например, если в контексте появляется \texttt{+}, то на соответствующие типы \texttt{a} накладывается ограничение \mintinline{haskell}{Num a}.
  \item В простом случае, \mintinline{haskell}{(\x y -> x + 2 * y) :: Num a => a -> a -> a} \\
  --- это условие на переменную по типам.
  \item Однако тип может стать и сложным (за счёт унификации):
  \begin{minted}{haskell}
(\f g x -> (f + g) x)
  :: Num (t1 -> t2) => (t1 -> t2) -> (t1 -> t2) -> t1 -> t2
  \end{minted}
    \begin{itemize}
    \item Для этого нужна прагма 
    \mintinline{haskell}{{-# LANGUAGE FlexibleContexts #-}}
    \end{itemize}
 \end{itemize}

\end{frame}

\begin{frame}[fragile]{Классы типов}
 \begin{itemize}[<+->]
  \item Функция, конечно, не число, но операции можно определить, например, покоординатно:
  {\scriptsize
\begin{minted}{haskell}
 {-# LANGUAGE FlexibleContexts, NoMonomorphismRestriction #-}
 
 instance (Num a, Num b) => Num (a -> b) where
    f + g = \x -> (f x + g x)
    f * g = \x -> (f x * g x)
    abs f = \x -> (abs (f x))
    signum f = \x -> (signum (f x))
    fromInteger n = \x -> fromInteger n
    negate f = \x -> (negate (f x))

  fff = (\f g x -> (f + g*g) x)
\end{minted}
  }
  \begin{itemize}
  \item Здесь появляется ещё одна прагма: \texttt{NoMonomorphismRestriction}. 
  \item Без неё тип для \texttt{fff} будет не полиморфным, а конкретизируется, и при её различном 
  использовании выведение типов не сработает. 
  \end{itemize}
 \end{itemize}

\end{frame}


\begin{frame}[fragile]{Параметрические классы типов}
 
 \begin{itemize}[<+->]
  \item Бывают и более сложные классы типов. 
  \item Рассмотрим классы типов сорта \mintinline{haskell}{(* -> *) -> Constraint}
  \item Сорт \mintinline{haskell}{* -> *} содержит {\em однопараметрические конструкторы типов.} 
  \item Например, таковы уже известный нам \mintinline{haskell}{Maybe} и конструктор типа списков \mintinline{text}{[ ]}: каждый из них по типу-аргументу строит новый тип.
  \begin{itemize}
  \item Заметим, что конструктор типов (например, \mintinline{haskell}{Maybe}) не следует путать с конструктором объектов алгебраического типа, таким как \mintinline{haskell}{Just}. Последний имеет тип (а не сорт) \texttt{a}\mintinline{haskell}{ -> Maybe a}
  \item Сложные сорта содержат не только алгебраические типы: например, \mintinline{haskell}{(->) :: * -> * -> *}
  \end{itemize}
  \item В \mintinline{haskell}{(* -> *) -> Constraint}  тоже живут классы типов.
 \end{itemize}

\end{frame}

\begin{frame}[fragile]{Параметрические классы типов}

\begin{itemize}[<+->]
 \item Например, класс типов \mintinline{haskell}{Foldable} включает однопараметрические типы, для которых определено <<сворачивание>>:
 
{\scriptsize
\begin{minted}{haskell}
type Foldable :: (* -> *) -> Constraint
class Foldable t where
  ...
  foldl :: (b -> a -> b) -> b -> t a -> b
  ...
\end{minted}

}

\item Естественная реализация (instance) класса \mintinline{haskell}{Foldable} --- это тип списков \texttt{[a]}.
\item Здесь \texttt{foldl op w}\mintinline{haskell}{ [x,y,z]} означает
\mintinline{haskell}{((w `op` x) `op` y) `op` z}:

{\scriptsize
\begin{minted}{haskell}
foldl f z []     = z
foldl f z (x:xs) = let z' = z `f` x in foldl f z' xs
\end{minted}
}
\item Однако \mintinline{haskell}{Maybe} также \mintinline{haskell}{Foldable} (угадайте, как там работает \texttt{foldl}).

\end{itemize}

\end{frame}

\begin{frame}{<<Категория типов>>}
 
 \begin{itemize}[<+->]
  \item Систему типов можно представить как очень большой граф,
  где вершины --- типы, а стрелки (рёбра) --- функции между ними.
  \begin{itemize}
  \item Этот граф ориентированный, в нём есть петли и кратные рёбра.
  \end{itemize}
  \item В математике такие графы, с естественными условиями на стрелки (композиция, существование петли-<<единицы>>), называются {\bf категориями.}
    \item Вершины называются {\em объектами} категории, а стрелки --- {\em морфизмами.}
 \end{itemize}

\end{frame}

\begin{frame}{<<Категория типов>>}

\begin{itemize}[<+->]
 \item  В Haskell'е система типов содержит функциональные (<<стрельчатые>>) типы, т.е. совокупность морфизмов между двумя объектами сама представляется как объект.
 \item Категории с таким свойством называются {\bf декартово замкнутыми.}
 \item Ближайший пример из математики: категория $\mathsf{SET}$ множеств и функций между ними.
 \item Огрубляя, можно мыслить типы как множества, но есть и тонкие отличия.
 \begin{itemize}
 \item Например, объединение $A \cup B$ множеств содержит как подмножества $A$ и $B$, а для его ближайшего аналога в системе типов, конструкции \texttt{Either}, имеются только функции-вложения \texttt{Left} и \texttt{Right}.
 \end{itemize}
\end{itemize}

\end{frame}

\begin{frame}{<<Категория типов>>}
 
 \begin{itemize}[<+->]
  \item Из теории категорий <<импортированы>> некоторые идеи в функциональном программировании, о которых пойдёт речь дальше.
  \item Аккуратное изложение самой теории категорий целью этого курса не является.
  \item Нам достаточно будет категорного взгляда <<с высоты птичьего полёта>>, при котором большие и сложные типы представляются точками.
  \item Итак, теоретико-категорные конструкции
{\em мотивируют} конструкции в системе типов, в т.ч. {\em монады.}
 \end{itemize}

\end{frame}

\begin{frame}{Функторы}
 
 \begin{itemize}[<+->]
  \item Сами категории можно также воспринимать как
алгебраические структуры, и между ними можно строить
<<морфизмы>>.
  \item Такие <<морфизмы категорий>> называются {\em функторами.}
  \item Функтор $F \colon \Cc \Rightarrow \Dc$ отображает объекты в объекты, морфизмы в морфизмы, при этом сохраняя начало/конец:
 \[
  \xymatrix{
  A \ar_f[d]\ar@{=>}^{F}[r] & F A 
     \ar^{F f}[d] \\
  B \ar@{=>}^{F}[r] & F B
  }
 \]
\vspace*{-10pt}
\item При этом функтор <<уважает>> композицию и единицу: $F(f \circ g) = (F f) \circ (F g)$ и $F \One_A = \One_{F A}$.
 \end{itemize}

\end{frame}

\begin{frame}{Функторы}
\begin{itemize}[<+->]
 \item Кабы не логические парадоксы, можно было бы говорить о
<<категории всех категорий>>, где морфизмы --- это
функторы.
 \item У нас одна категория --- абстрактная категория, приближённо описывающая систему типов языка.
 \item Поэтому нас будут интересовать {\em эндофункторы}
 $F \colon \Cc \Rightarrow \Cc$ (т.е. действующие
внутри одной категории).
\item С точки зрения системы типов (эндо)функтор должен реализоваться как {\em преобразование типов} и связанная с ним {\em функция на функциях.}
\end{itemize}

 

\end{frame}



\begin{frame}{Функторы}
 
 \begin{itemize}[<+->]
  \item Пример: взятие списка элементов данного типа, \texttt{a} $\Rightarrow$ \texttt{[a]}.
  \item Соответствующее преобразование на функциях \mintinline{haskell}{map :: (a -> b) -> [a] -> [b]}
  \item Близкая математическая конструкций --- функтор $\Pc$ в категории $\mathsf{SET}$, сопоставляющий множеству множество всех его подмножеств.
  \begin{itemize}
 \item Для $f \colon A \to B$ и $X \subseteq A$ имеем $\Pc f \colon X \mapsto \{ f(x) \mid x \in X \}$.
 \item Можно рассмотреть $\Pc_{\mathrm{fin}}$ --- взятие множества конечных подмножеств.
 \item Кстати, список в Haskell'е не обязательно конечный.
 \item Отличие множества от списка --- во множестве не имеет значения порядок и кратность элементов.
 \item К функтору $\Pc_{\mathrm{fin}}$ мы ещё вернёмся.
 \end{itemize}
 \end{itemize}

\end{frame}

\begin{frame}[fragile]{Функторы}

\begin{itemize}[<+->]
 \item В общем случае понятие функтора реализуется как параметрический класс типов \mintinline{haskell}{Functor}
 
 {\footnotesize
\begin{minted}{haskell}
type Functor :: (* -> *) -> Constraint
class Functor f where
  fmap :: (a -> b) -> f a -> f b 
\end{minted}

 }
 \item При этом реализация этого класса --- это не обязательно алгебраический тип, т.е. \texttt{f a} не обязательно является <<новым>> типом, построенным из \texttt{a}.
 \item Пример:
 
{\footnotesize
 \begin{minted}{haskell}
instance Functor ((->) r)
 \end{minted}

 }
 
 \item Этот функтор преобразует тип \texttt{a} в \texttt{r -> a} (\texttt{r} --- фиксированный тип-параметр).
 \begin{itemize}
 \item Здесь \mintinline{text}{fmap h = \g z -> h (g z)}
 \item \texttt{a} $\Rightarrow$ \texttt{a -> r} --- это {\em контравариантный функтор} (переворачивает стрелки).
 \end{itemize}
\end{itemize}

 
\end{frame}

\begin{frame}{Условия функториальности}
 \begin{itemize}[<+->]
 \item От типа, реализующего класс \mintinline{haskell}{Functor}, требуется соблюдение условий функториальности: $F(f \circ g) = (F f) \circ (F g)$ и $F \One_A = \One_{FA}$.
 \item В терминах \mintinline{text}{fmap}:\\
 \mintinline{text}{fmap (f . g) = (fmap f) . (fmap g)} и \mintinline{text}{fmap (\x -> x) = \x -> x}.
 \item При работе с типом класса \mintinline{haskell}{Functor} можно предполагать, что эти равенства выполнены, однако, к сожалению, они не {\em верифицируются} в самом Haskell'е.
 \item Например, для списков можно определить альтернативный \mintinline{haskell}{fmap' = \f -> reverse .                                                                                                                                                                                                                                                                                                                                                                                                                                                                                                                   (map f)}, для которого эти равенства неверны.
\end{itemize}

\end{frame}

\begin{frame}{Функтор}
 
 \begin{itemize}[<+->]
  \item Вернёмся к примеру \texttt{a} $\Rightarrow$ \texttt{r -> a}. <<Внутри>> \texttt{r -> a} находится бесконечно много элементов типа \texttt{a}, по одному для каждого \texttt{z :: r}.
  \item То же происходит для функтора взятия списка или множества.
  \item Уже знакомый нам конструктор типов \mintinline{haskell}{Maybe} также является функтором, и в \mintinline{haskell}{Maybe a} лежит либо один элемент, либо ни одного элемента типа \texttt{a}.
 \end{itemize}

\end{frame}
\fi

\begin{frame}{<<Чёрный ящик>>}

\begin{itemize}[<+->]
\item Таким образом, если $X$ --- тип, а $F$ --- функтор, то $FX$ --- это <<чёрный ящик>>, в котором каким-то образом спрятаны элементы типа $X$, к которым можно применить функцию $f \colon X \to Y$.
\begin{center}
\begin{tikzpicture}
 \draw[rounded corners,thick,fill=gray] (0,0) rectangle (6,2) node[below left] {$FX$};
 \draw[rounded corners,thick,fill=lightcyan]
 (.5,.5) rectangle (1.5,1.5) node[below left] {$X$};
 \draw[rounded corners,thick,fill=lightcyan] (2,.5) rectangle (3,1.5) node[below left] {$X$};
 \draw[rounded corners,thick,fill=lightcyan] (3.5,.5) rectangle (4.5,1.5) node[below left] {$X$};
\end{tikzpicture}
\end{center}
\item Однако чего-то не хватает: непонятно, как <<положить>> что-то в $FX$, и вообще, как создать объект типа $FX$.
\end{itemize}
 
 
\end{frame}



\begin{frame}{Монада}
 
 \begin{itemize}[<+->]
  \item {\em Монада} $M$ --- это эндофунктор с дополнительными операциями.
  \item Первая из них позволяет <<положить элемент в чёрный ящик>>, $\eta_X \colon X \to M X$.
  \begin{itemize}
  \item \mintinline{haskell}{return :: Monad m => a -> m a}
  \end{itemize}
  \item Вторая более сложная и позволяет <<поднимать>> аргумент функции, ведущей в монаду. Для $f \colon X \to M Y$ эта операция даёт $f^* \colon M X \to M Y$.
  \begin{itemize}
  \item \mintinline{haskell}{(>>=) :: Monad m => m a -> (a -> m b) -> m b}
  \end{itemize}
  %\item При этом должны выполняться определённые условия, о которых чуть позже.
  \item Набор $(M, \eta, {}^*)$ в теории категорий называется {\em тройкой Клейсли,} а собственно монадой --- другая, но эквивалентная конструкция.
 \end{itemize}
\end{frame}

\begin{frame}{{\tt {>}>=}}
\begin{itemize}[<+->]
 \item Операция $^*$, или {\tt {>}>=}, соответствует идее, что можно внутри монады войти ещё раз в монаду, и остаться в той же монаде.
 \item Пример: \mintinline{haskell}{[1,2,3] >>= (\x -> [x,x])} даёт \mintinline{haskell}{[1,1,2,2,3,3]}
 \item Если бы вместо $f^*$ применили $M$ как функтор, то получилось бы $Mf \colon MX \to M(MY)$.
 
 \begin{center}
\begin{tikzpicture}
 \draw[rounded corners,thick,fill=gray] (0,0) rectangle (8.5,3.2) node[below left] {$M(MY)$};
 \draw[rounded corners,thick,fill=lightcyan]
 (.5,.5) rectangle (4,2.5) node[below left] {$MY$};
 \draw[rounded corners,thick,fill=lightcyan] (4.5,.5) rectangle (8,2.5) node[below left] {$MY$};
 \draw[rounded corners,thick,fill=white]
 (1,1) rectangle (2,2) node[below left] {$Y$};
 \draw[rounded corners,thick,fill=white]
 (2.5,1) rectangle (3.5,2) node[below left] {$Y$};
 \draw[rounded corners,thick,fill=white]
 (5,1) rectangle (6,2) node[below left] {$Y$};
 \draw[rounded corners,thick,fill=white]
 (6.5,1) rectangle (7.5,2) node[below left] {$Y$};
 %\draw[rounded corners,thick,fill=lightcyan] (3.5,.5) rectangle (4.5,1.5) node[below left] {$X$};
\end{tikzpicture}
\end{center}

\end{itemize}
\end{frame}

\begin{frame}{$\mu$}

\begin{itemize}[<+->]
 \item Таким образом,  $f^*$ можно свести к более простой операции <<разглаживания>> двойной монады в одинарную, $\mu_X \colon M(MY) \to MY$.
 \item Тогда $f^* = \mu \circ (M f)$.
 \item Собственно, в теории категорий именно эндофунктор $M$, оснащённый семействами морфизмов $\eta$ и $\mu$, удовлетворяющий огромному количеству условий корректности, и называют монадой.
\end{itemize}

\end{frame}

\begin{frame}{Условия монады}
 
  Для тройки Клейсли $(M, \eta, {}^*)$ условия формулируются намного короче:
  \begin{center}
   \begin{tabular}{ll}
    $\eta_X^* = \One_{M X}$ & 
    \mintinline{text}{(z >>= return) = z} \\
    \multicolumn{2}{l}{\hspace*{1em} $MX \to MX$} \\[10pt]
    $f^* \circ \eta_X = f$ &
    \mintinline{text}{(return x >>= f) = (f x)} \\
    \multicolumn{2}{l}{\hspace*{1em} $X \stackrel{\eta}{\to} MX \stackrel{f^*}{\to} MY$} \\[10pt]
    $g^* \circ f^* = (g^* \circ f)^*$ &
    \mintinline{text}{((x >>= f) >>= g) =}
    \\&\hspace*{10pt}
    \mintinline{text}{(x >>= (\y -> (f y >>= g)))} \\
    \multicolumn{2}{l}{\hspace*{1em} 
    $MX \stackrel{f^*}{\to} MY \stackrel{g^*}{\to} MZ$} \\
    \multicolumn{2}{l}{\hspace*{1em} 
    $X \stackrel{f}{\to} MY \stackrel{g^*}{\to} MZ$} \\
   \end{tabular}
  \end{center}
\end{frame}


\iffalse
\begin{frame}{Выражение остальных операций}

\begin{itemize}[<+->]
 \item Через $\eta$ и $^*$ можно выразить всё остальное.
 \item Так, $\mu \colon M(MY) \to MY$, или {\tt join}, выражается так: $\mu = \One_{MY}^*$.
 \begin{itemize}
 \item \mintinline{haskell}{join = \x -> (x >>= (\y -> y))}
 \end{itemize}
 \item Преобразование на морфизмах (функтор), $Mf \colon MX \to MY$ (для $f \colon X \to Y$) также выражается: $Mf = (\eta_Y \circ f)^*$.
 \begin{itemize}
 \item \mintinline{haskell}{fmap = (\f -> (\z -> (z >>= \x -> return (f x))))}
 \item При этом $Mg \circ Mf = (\eta_Z \circ g)^* \circ (\eta_Y \circ f)^* = ((\eta_Z \circ g)^* \circ \eta_Y \circ f)^* = (\eta_Z \circ g \circ f)^* = M(g \circ f)$.
 \item Таким образом, в определении монады не нужно требовать, что это функтор.
 \end{itemize}
 \item Более того, каждая монада --- это {\em аппликативный функтор} с операцией применения функции внутри монады:
 \mintinline{haskell}
 {(<*>) :: m (a -> b) -> m a -> m b}
 \begin{itemize}
 \item
  \mintinline{text}{h } \mintinline{haskell}{<*> t = h >>= \f -> fmap f t}
 \end{itemize}
\end{itemize}

 
\end{frame}
\fi

\iffalse
\begin{frame}{Монада IO}
 
 
 \begin{itemize}[<+->]
  \item При вычислении последовательности функций <<внутри>> монады сама монада каждый раз заменяется на новую.
  \item Таким образом поддерживается порядок вычислений, что приближает работу в монаде к императивному языку.
  \begin{itemize}
  \item Явно это выражается в т.н. do-нотации, о которой чуть позже.
  \end{itemize}
  \item В частности, с помощью специальной монады можно реализовать взаимодействие с <<внешним миром>> (побочные эффекты вычислений), в частности, ввод-вывод.
  \item Эта специальная монада называется \mintinline{haskell}{IO}.
 \end{itemize}
 
\end{frame}
\fi

\begin{frame}{Монада IO}
 \begin{itemize}[<+->]
  \item Одна из наиболее известных монад --- это монада \mintinline{haskell}{IO}, с помощью которой реализуется <<выход во внешний мир>>. 
  \item Объект типа \mintinline{haskell}{IO a} можно понимать как объект типа \mintinline{haskell}{a}, помещённый в большой и страшный внешний мир.
  \item При этом с IO-монадическими объектами можно работать <<чисто функциональным>> образом (они существуют как задумки, а реализуются только при финальном вычислении).
  \item Однако сегодня мы поговорим о монадах, которые упрощают чисто функциональное программирование.
 \end{itemize}

 
\end{frame}

\iffalse
\begin{frame}{Монада IO}
 
 \begin{itemize}
  
 \end{itemize}

 \visible<2->{
 \begin{center}
  \includegraphics[scale=.3]{ezhik.jpg}
  \\ \vspace*{-5pt}
  {\tiny\color{gray}\sf <<Ёжик в тумане>>, Союзмультфильм}
 \end{center}

 }
 
\end{frame}
\fi

\iffalse
\begin{frame}{Монада IO}
 
 \begin{itemize}[<+->]
  \item Грубое приближение монады \mintinline{haskell}{IO} --- это взятие пары с контекстом: ``({\tt a}, {\tt RealWorld})''. При операциях с монадой состояние {\tt RealWorld} может меняться. 
  \item Есть операция \mintinline{text}{return}, погружающая объект в окружение \mintinline{haskell}{IO} (<<выпускающая во внешний мир>>), а вот обратного преобразования нет.
  \item \mintinline{haskell}{putChar :: Char -> IO ()} берёт символ и возвращает новый <<мир>>, в котором растворился (был напечатан в консоли) этот символ.
  \item \mintinline{haskell}{getChar :: IO Char} --- мы можем получить символ из внешнего мира, но только внутри монады.
  \item Достать его из монады мы не можем, но можем работать с ним внутри \mintinline{haskell}{IO} с помощью \mintinline{haskell}{>>=}.
 \end{itemize}

 
\end{frame}

\begin{frame}{Монада IO}

\begin{itemize}[<+->]
\item Например: \mintinline{text}{getChar} \mintinline{haskell}{>>= (\x -> (putChar x >> putChar x))}
\item Здесь \mintinline{text}{>>} --- это версия \mintinline{text}{>>=}, игнорирующая аргумент ({\tt putChar} всё равно ничего не выдаёт, а настоящий {\tt x} ранее абстрагирован).
 \item В процессе выполнения программы, содержащей \mintinline{haskell}{IO}, объекты типов \mintinline{haskell}{IO a} остаются временно невычисленными, как задумки.
 \item Например, если мы где-то напишем \mintinline{text}{putChar} \mintinline{haskell}{'a'}, то символ не будет тут же напечатан.
 \item Вместо этого нужно дождаться, пока соберётся <<главный>> объект типа \mintinline{haskell}{IO ()}, и уже при его вычислении все операции с внешним миром будут выполнены, причём в правильном порядке.
\end{itemize}

 
 
\end{frame}
\fi

\begin{frame}[fragile]{do-нотация}
 
 \begin{itemize}[<+->]
  \item {\em do-нотация} --- это альтернативный синтаксис работы с \mintinline{text}{>>=} и \mintinline{text}{>>}, делающий код похожим на императивный.
  \item Пример:
\begin{minted}{haskell}
main :: IO ()
main = do
  putStrLn "What's your name?"
  name <- getLine
  putStrLn $ "Hello, " ++ name ++ "!"
\end{minted}
\item do-нотация раскрывается так. <<Команды>>, у которых нет возвращаемого значения, соединяются с помощью \mintinline{text}{>>}. Если возвращаемое значение есть: \mintinline{text}{x} \mintinline{haskell}{<- ...}, то пишется \mintinline{haskell}{... >>= \x -> ...}
 \end{itemize}

\end{frame}


\begin{frame}[fragile]{do-нотация}
 
 \begin{itemize}[<+->]
 \item Таким образом переменная по имени {\tt x} становится доступной в дальнейшем контексте.
 \item Более того, в <<присваивании>> \mintinline{haskell}{<-} можно (как в императивных языках) использовать одно и то же имя несколько раз.
 \begin{itemize}
 \item При этом более раннее забывается, поскольку переменная связана более глубокой лямбдой: $\lambda x. (... \lambda x. (... x ...) ...)$.
 \end{itemize}
 \item Пример:
 \begin{minted}{haskell}
main =
  putStrLn "What's your name?" >>
  getLine >>=
  \name -> putStrLn $ "Hello, " ++ name ++ "!"
 \end{minted}
 
 \end{itemize}

\end{frame}

\begin{frame}[fragile]{do-нотация}
 \begin{itemize}[<+->]
 \item do-нотация работает не только с \mintinline{haskell}{IO}, но и с любой другой монадой.
 \item Например, вот такой код рекурсивно генерирует все булевы наборы данной длины:
 
 {
\scriptsize
\begin{minted}{haskell}
allVals 0 = [[]]
allVals n = do
  v <- allVals (n-1)
  [True:v, False:v]
\end{minted}

}

\end{itemize}
\end{frame}


\iffalse
\begin{frame}{Монады}
 \begin{itemize}[<+->]
  \item Итак, {\em монада} --- это абстрактная конструкция преобразования типов: $X$ преобразуется в $MX$, при этом внутри <<монадического>> объекта типа $MX$ в некотором <<живут>> элементы исходного типа $X$:
  \begin{center}
\begin{tikzpicture}
 \draw[rounded corners,thick,fill=gray] (0,0) rectangle (6,2) node[below left] {$MX$};
 \draw[rounded corners,thick,fill=lightcyan]
 (.5,.5) rectangle (1.5,1.5) node[below left] {$X$};
 \draw[rounded corners,thick,fill=lightcyan] (2,.5) rectangle (3,1.5) node[below left] {$X$};
 \draw[rounded corners,thick,fill=lightcyan] (3.5,.5) rectangle (4.5,1.5) node[below left] {$X$};
\end{tikzpicture}
\end{center}
\item Внутри монады можно применять функцию к исходному типу --- $f \colon X \to Y$ <<поднимается>> до $Mf \colon MX \to MY$, т.е. монада является функтором.
 \end{itemize}

 
\end{frame}

\begin{frame}{Монады}
 
 \begin{itemize}[<+->]
  \item Более того, $f$ сама может создавать монадический объект, и этот объект помещается в исходную монаду.
  \item А именно, $f \colon X \to MY$ даёт $f^* \colon MX \to MY$. Другое обозначение:
  $f^*(t) = (t \bind f)$.
  \item Можно определить $f^*$ через операцию <<разглаживания двойной монады>> $\mu \colon MMY \to MY$, а именно, $f^* = \mu \circ f$.
  \item Наконец, в монаду можно помещать объект <<чистого>> типа, $\eta \colon X \to MX$.
  \item Для корректно определённых монад эти операции должны удовлетворять нескольким естественным соотношениям.
 \end{itemize}

 
\end{frame}


\begin{frame}{Монады}
 
 \begin{itemize}[<+->]
  \item Основный примером является монада \mintinline{haskell}{IO}, осуществляющая взаимодействие с <<внешним миром>>.
  \item Рассмотрим ещё два примера использования монад --- для моделирования недетерминированных и вероятностных вычислений.
  \item Общая идея: помещение внутрь монады помещает вычисления в некоторое своеобразное окружение.
 \end{itemize}

\end{frame}
\fi

\begin{frame}{Монада для недетерминизма}
 
 \begin{itemize}[<+->]
  \item В первом примере мы будем моделировать поведение {\em недетерминированного конечного автомата (НКА).}
  \item Напомним, что конечный автомат имеет конечное множество {\em состояний} $Q$ и перемещается между ними в зависимости от очередного символа входного слова $w$.
  \item Автомат недетерминированный, если буква входного слова не обязательно однозначно определяет переход:
  
  \begin{center}
  \begin{tikzpicture}
   \node[circle,draw] (p) at (0,0) {$p$};
   \node[circle,draw] (q1) at (3,.5) {$q_1$};
   \node[circle,draw] (q2) at (3,-.5) {$q_2$};
   
   \draw[->] (p)--(q1) node[midway,above] {$a$};
   \draw[->] (p)--(q2) node[midway,below] {$a$};
  \end{tikzpicture}
  \end{center}
 \end{itemize}

\end{frame}

\begin{frame}{Монада для недетерминизма}
 
 \begin{itemize}[<+->]
  \item Таким образом, после $n$ шагов, т.е. входного слова $w = a_1 \ldots a_n$, мы имеем некоторое подмножество {\em возможных} состояний автомата, $\Delta^n(w) \subseteq Q$.
  \item При этом сама функция перехода действует из состояния во множество состояний, $\delta \colon \Sigma \to (Q \to \Pc Q)$.
  \item Это в точности ситуация {\em монадического связывания} для монады $\Pc$:
  
  \centerline{\(
   \Delta(w) = (\eta\, q_0) \bind \delta(a_1) \bind \ldots \bind \delta(a_n) 
  \)}
  \begin{itemize}
  \item Здесь $\eta\, q_0 = \mbox{\tt return\ } q_0 = \{ q_0 \}$.
  \end{itemize}

  
  %FIXME пример конечного автомата
 \end{itemize}

\end{frame}

\begin{frame}{Пример конечного автомата}
 \begin{tikzpicture}
   \node[circle,draw] (q0) at (0,0) {$q_0$};
   \node[circle,draw] (q1) at (3,1) {$q_1$};
   \node[circle,draw] (q2) at (3,-.5) {$q_2$};
   \node[circle,draw] (q3) at (6,-.5) {$q_3$};
   
   \draw (q0) edge[->,bend left] node[midway,above] {$a$} (q1) ;
   \draw[->] (q1)--(q0) node[midway,below] {$a$};
   \draw[->] (q0)--(q2) node[midway,below] {$a$};
   \draw (q2) edge[->,bend left] node[midway,above] {$a$} (q3);
   \draw (q3) edge[->,bend left] node[midway,below] {$a$} (q2);
   
   \draw (q1) edge[->,loop] node[midway,above] {$b$} (q1);
   
   \draw (q2) edge[->,out=-45,in=-135,looseness=7] node[midway,below] {$b$} (q2);
 \end{tikzpicture}
 
 \begin{itemize}
  \item<2-> На входном слове из $(ab^*a)^*$ этот автомат переходит либо в $q_0$, либо в $q_3$.
 \end{itemize}


\end{frame}


\begin{frame}[fragile]{Монада для недетерминизма}
 
 Идея монады $\Pc$ для недетерминизма в точности реализуется: % на практике:

 
\begin{minted}{haskell}
import Control.Monad
import Data.Set.Monad

data NFA q sigma = NFA (sigma -> q -> Set q) q

runq :: NFA q sigma -> q -> [sigma] -> Set q
runq (NFA delta qi) q0 [] = return q0
runq (NFA delta qi) q0 (a:ss) = do
    q1 <- delta a q0
    q2 <- runq (NFA delta qi) q1 ss
    return q2

run (NFA delta qi) a = runq (NFA delta qi) qi a
\end{minted}


 
\end{frame}

\begin{frame}[fragile]{Монада для недетерминизма}
 
 Далее, можно закодировать автомат из нашего примера:
 
 {\footnotesize
 \begin{minted}{haskell}
data Q = Q0 | Q1 | Q2 | Q3 deriving (Eq, Ord, Show)
  
delta :: Char -> Q -> Set Q
delta 'a' Q0 = fromList [Q1,Q2]
delta 'b' Q0 = fromList []
delta 'a' Q1 = fromList [Q0]
delta 'b' Q1 = fromList [Q1]
delta 'a' Q2 = fromList [Q3]
delta 'b' Q2 = fromList [Q2]
delta 'a' Q3 = fromList [Q2]
delta 'b' Q3 = fromList []

automaton = NFA delta Q0
 \end{minted}
}

\end{frame}

\begin{frame}[fragile]{Монада для недетерминизма}
 \begin{itemize}[<+->]
  \item Наконец, запускаем:
\begin{minted}{haskell}
main = putStrLn $ show $ run automaton "abbaabbba"
\end{minted}
и получаем ответ: {\tt fromList [Q0,Q3]}.
\item Заметим, что если использовать монаду списка вместо \mintinline{haskell}{Set}, то состояния будут дублироваться: {\tt [Q0,Q3,Q3]}.
\item Это может привести к экспоненциальному расходу ресурсов.
\item Переход от $Q$ к $\Pc Q$ и замена функции перехода $\delta(a) \colon Q \to \Pc Q$ на $\delta(a)^* \colon \Pc Q \to \Pc Q$ --- это и есть {\em алгоритм детерминизации} конечного автомата.
\item В нашей реализации мы не храним детерминированную версию конечного автомата, тем самым избегая экспоненциального расхода памяти (ленивость!).
 \end{itemize}

 
\end{frame}



\begin{frame}{Случайные числа}

\begin{itemize}[<+->]
 \item Функции в Haskell'е должны быть чистыми. Таким образом, мы не можем определить функцию (без аргументов) ``random'', которая будет при каждом вызове давать новое (псевдо)случайное число.
 \item Однако это возможно внутри монады. 
 \item В частности, в монаде \mintinline{haskell}{IO} имеется {\tt randomIO}, который выдаёт псевдослучайное число (в зависимости от конкретизации типа).
 \item Однако Haskell поддерживает и более абстрактный способ работы с вероятностными объектами.
\end{itemize}


 %FIXME ... поговорить, что в Хаскеле нельзя просто написать random, из-за проблем с чистотой. Самый простой способ --- генератор случайного числа в монаде IO
\end{frame}


\begin{frame}{Вероятностные распределения}
 
 \begin{itemize}[<+->]
  \item Вероятностное распределение на конечном множестве $X = \{ x_1, \ldots, x_n \}$ (дискретное) задаётся набором чисел $(p_1, \ldots, p_n)$, где $p_i \geqslant 0$ и $p_1 + \ldots + p_n = 1$.
  \item $p_i = p(x_i)$.
  \item Для события $A \subseteq X$ имеем $P(A) = \sum\limits_{x\in A} p(x)$.
  \item В непрерывном случае вероятностное распределение задано функцией плотности $p$; $p(x) \geqslant 0$ и $\int\limits_X p(x) \, dx = 1$.
  \item При этом $P(A) = \int\limits_A p(x)\, dx$.
 \end{itemize}

 
\end{frame}

\begin{frame}{Монада Жири}
 
 \begin{itemize}[<+->]
  \item Обозначим через $\Rc X$ множество всех вероятностных распределений на $X$.
  \item $\Rc$ является функтором. Функция $f \colon X \to Y$ переносит вероятностное распределение c $X$ на $Y$ следующим образом: $p'(y) = P(f^{-1}(y)) = \sum\limits_{f(x) = y} p(x)$.
  \begin{itemize}
  \item При этом выполняются условия функтора.
  \end{itemize}
  \item Более того, $\Rc$ является монадой.
  \item Эта монада называется {\em монадой Жир\'{и}} (Giry monad).
  \begin{itemize}
  \item M. Giry. A categorical approach to probability theory
  \end{itemize}
 \end{itemize}

 
\end{frame}

\begin{frame}{Монада Жири}
 
 \begin{itemize}[<+->]
  \item Морфизм $\eta \colon X \to \Rc X$ реализуется взятием распределения, сосредоточенного в одной точке:
\[
 p(y) = \left\{ \begin{aligned}
       & 1, && \mbox{если $y = x$;}\\
       & 0, && \mbox{иначе.}
                \end{aligned}
\right.
\]
  \item Операцию связывания ($\bind$) будем определять через $\mu$ (join): если $f \colon X \to \Rc Y$, то $(p \bind f) = \mu \circ (\Rc f)$. 
  \item Морфизм $\mu \colon \Rc\Rc X \to \Rc X$ делает обычное распределение из <<распределения распределений>>.
  \item Чтобы получить случайный $x$ по распределению $P \in \Rc \Rc X$,  мы сначала случайно выбираем распределение $p \in \Rc X$, а потом по этому распределению выбираем $x \in X$.
  \item $(\mu P)(x) = \sum\limits_{p \in \Rc X} (P(p) \cdot p(x))$
  \end{itemize}


\end{frame}

\begin{frame}{Монада Жири}
 
 \begin{itemize}[<+->]
  \item Связывание, $p \bind f$, где $p : \Rc X$ и $f : X \to \Rc Y$, получается таким образом:
  \[
   (p \bind f) (y) = \sum_{x \in X} (p(x) \cdot f(x)(y))
  \]
  \item Вероятностный смысл: выбираем случайный $x \in X$ и по нему строим новое распределение $f(x)$, с помощью которого выбираем $y \in Y$.
 \end{itemize}

\end{frame}


\begin{frame}[fragile]{Вычисления со случайностью}

\begin{itemize}[<+->]
 \item Внутри монады $\Rc$ можно <<присваивать>> переменной случайное значение (в рамках do-нотации):
 \begin{minted}{haskell}
 
import Control.Monad.Random
fair = fromList [("heads",0.5),("tails",0.5)]
quart = do
    a <- fair
    b <- fair
    if (a == "heads") && (b == "heads") 
       then return "heads"
       else return "tails"
 \end{minted}
\item Однако здесь просто {\em вычисляются} параметры вероятностного распределения, а не {\em генерируется случайное число.}
\end{itemize}

 
\end{frame}

\begin{frame}[fragile]{Абстрактная работа со случайными объектами}
 
 \begin{itemize}[<+->]
  \item Посмотрим внимательнее на типы.
\begin{minted}{haskell}
fromList :: MonadRandom m => [(a, Rational)] -> m a
fair :: MonadRandom m => m [Char] 
\end{minted}
  \item Мы видим {\em полиморфный} объект, тип которого параметризован {\em произвольной} <<монадой случайности>> {\tt m} (которая как-то хранит вероятностное распределение).
  \item На этом этапе детали реализации {\tt m} не важны, они понадобятся в тот момент, когда мы захотим на самом деле <<подбросить монеты>>.

 \end{itemize}

 
\end{frame}


\begin{frame}{Реализация MonadRandom}
 \begin{itemize}[<+->]
  \item Стандартная реализация класса \mintinline{haskell}{MonadRandom} даётся двупараметрическим типом \mintinline{haskell}{Rand g a} при фиксированном первом параметре.
  \item Параметр-тип {\tt g}, который должен принадлежать классу \mintinline{haskell}{RandomGen}, задаёт тип {\em генератора} (источника) случайности.
  \item Одним из таких типов является \mintinline{haskell}{StdGen}.
  \item Из объекта типа \mintinline{haskell}{StdGen} (источник случайности) можно извлечь некоторое случайное значение и, кроме того, новый, модифицированный источник.
  \item Таким образом реализуется последовательность псевдослучайных чисел.
  \item Важно понимать, что сам источник --- это константа, и он всегда будет выдавать одно и то же значение.
 \end{itemize}

 
\end{frame}

\begin{frame}{Реализация MonadRandom}
 
 \begin{itemize}[<+->]
  \item В нашем примере {\tt quart} имеет полиморфный тип \mintinline{haskell}{MonadRandom m => m [Char]}, который может конкретизироваться в \mintinline{haskell}{Rand StdGen [Char]}.
  \item Имеется функция \mintinline{haskell}{evalRand :: Rand g a -> g -> a} \\ (в частности, \mintinline{haskell}{evalRand :: Rand StdGen a -> StdGen -> a}), которая выдаёт случайное значение, с данным распределением, используя данный источник случайности.
  \item Чтобы получить новый источник, нужно воспользоваться другой функцией, 
  \mintinline{haskell}{runRand :: Rand g a -> g -> (a,g)}
  \item Остаётся последний вопрос --- откуда изначально взять источник случайности?
  %FIXME: поговорить о чистом коде и методах Монте-Карло...
  %FIXME: работающий пример с числом $\pi$
  %FIXME: runRand --- выдаёт новый генератор!
  %FIXME: примеры с бесконечным вычислением (thunk)
 \end{itemize}

\end{frame}

\begin{frame}{Метод Монте-Карло}

\begin{itemize}[<+->]
\item Иногда бывает достаточно использовать фиксированный источник псевдослучайных чисел (<<зерно>>), что позволяет писать чистый и фактически детерминированный код.
\item Одна из таких ситуаций --- вычисление меры и интеграла {\em методом Монте-Карло.}
\item Пример: случайно <<бросаем>> много точек в квадрат $[0,1] \times [0,1]$ и вычисляем долю точек, попавших в круг:
\begin{center}
 \begin{tikzpicture}[scale=2]
  \draw[thick] (0,0)--(0,1)--(1,1)--(1,0)--(0,0);
  \draw[thick,fill=lightblue] (0.5,0.5) circle (0.5);
 \end{tikzpicture}
\end{center}
\item Результат по вероятности стремится к площади круга: $\pi/4$.
\end{itemize}

 
\end{frame}

\begin{frame}[fragile]{Метод Монте-Карло}
 
 \begin{itemize}[<+->]
  \item Для реализации метода Монте-Карло вычисления числа $\pi$ нам понадобится следующая операция на монадах:
  \begin{minted}{haskell}
replicateM :: Applicative m => Int -> m a -> m [a]   
  \end{minted}
  \item В частности, для монады {\tt replicateM} можно определить так:
  \begin{minted}{haskell}
myreplicateM 0 mon = return []
myreplicateM n mon = mon >>= (\x -> myreplicateM (n-1) xs 
    >>= \xs -> return (x:xs))
  \end{minted}
  или в do-нотации:
  \begin{minted}{haskell}
myreplicateM n mon = do
    x <- mon
    xs <- myreplicateM (n-1) mon
    return (x:xs)
  \end{minted}


 \end{itemize}

\end{frame}


\begin{frame}[fragile]{Метод Монте-Карло}
 
 \begin{itemize}[<+->]
  \item Получается, что {\tt replicateM} создаёт {\em выборку} --- список значений случайной величины, причём каждый раз используется обновлённый генератор.
  \item Таким образом, значения получаются псевдонезависимы, что достаточно для статистических целей.
  \item Выбор изначального генератора в этом случае не имеет значения, и можно создать константный генератор с помощью {\tt mkStdGen} --- например, {\tt mkStdGen}\ \mintinline{haskell}{42}.
  \item Функция {\tt mkStdGen} чистая:
  \begin{minted}{haskell}
mkStdGen :: Int -> StdGen
  \end{minted}
  \item Равномерное распределение на $[0,1)$ даётся функцией {\tt getRandom} с конкретизированным типом:
  \begin{minted}{haskell}
unif = getRandom  :: MonadRandom m => m Double
  \end{minted}

 \end{itemize}

 
\end{frame}


\begin{frame}[fragile]{Метод Монте-Карло}

\begin{minted}{haskell}
import Control.Monad.Random

unif = getRandom :: MonadRandom m => m Double

inCircle x y = ( (x-0.5)^2 + (y-0.5)^2 <= 0.25 )

mcCheck = do
    x <- unif
    y <- unif
    return (inCircle x y)

xs = evalRand (replicateM 100000 mcCheck) (mkStdGen 42)
t = 4 * (length [x | x <- xs, x == True])
main = putStrLn $ show t
\end{minted}
\end{frame}

\begin{frame}{Метод Монте-Карло}

\begin{itemize}[<+->]
 \item Эта программа функционально чистая, и потому всегда выдаёт одно и то же значение:
 $314128 \approx \pi \cdot 100000$.
 \item Кстати, профилирование показывает, что с ленивостью здесь всё в порядке: вся выборка (100000 булевых значений) в памяти не хранится.
 \item Однако как же сгенерировать <<настоящее>> случайное число, т.е. взять источник случайности из системы?
 \item Поскольку здесь происходит взаимодействие с <<внешним миром>>, понадобится монада {\tt IO}.
\end{itemize}

 
\end{frame}

\begin{frame}[fragile]{Внешний генератор случайных чисел}
 \begin{itemize}[<+->]
  \item Монада \mintinline{haskell}{IO} хранит {\em глобальный} генератор случайности, который инициализируется в начале работы программы.
  \item Доступ к нему
\begin{minted}{haskell}
getStdGen :: IO StdGen
\end{minted}
  \item При этом этот генератор --- это просто глобальная переменная, и два вызова {\tt getStdGen} дадут одно и то же.
  \item Генератор можно <<обновить>> с помощью
\begin{minted}{haskell}
newStdGen :: IO StdGen
\end{minted}
  но при этом он будет просто заменён на следующее значение псевдослучайной последовательности.
  \item Чтобы получить <<настоящее>> новое случайное число, см. {\tt Data.Random.Source.DevRandom}
 \end{itemize}

\end{frame}


\begin{frame}{Бесконечные вероятностные вычисления}
 
 \begin{itemize}[<+->]
  \item Ленивость позволяет программировать внутри монады класса \mintinline{haskell}{MonadRandom} вычисления со случайными числами, которые при некоторых их значениях длятся бесконечно долго.
  \item Пример: дана <<нечестная>> монета, выпадающая орлом с вероятностью $p$, где $0 < p < 1$, $p \ne 1/2$. Нужно с её помощью сымитировать <<честное>> бросание, с вероятностью $1/2$.
  \item Соображение: если бросить монету два раза, то последовательности орёл-решка и решка-орёл равновероятны.
  \item В случаях орёл-орёл или решка-решка пробуем ещё раз.
  \item Процесс может оказаться бесконечным, но вероятность этого равна нулю.
 \end{itemize}

 
\end{frame}


\begin{frame}[fragile]{Бесконечные вероятностные вычисления}
 
 \begin{minted}{haskell}
unfair = fromList [("heads",0.25),("tails",0.75)]

fair = do
    a <- unfair
    b <- unfair
    if (a /= b) then (return a) else fair
 \end{minted}

 \begin{itemize}
  \item<2-> Если попытаться предвычислить распределение {\tt fair}, то мы уйдём в бесконечный цикл (хотя математически оно равно $(1/2,1/2)$).
  \item<3-> К счастью, Haskell ленив, и {\tt fair} остаётся как задумка (thunk). Настоящее вычисление произойдёт потом, когда мы уже получим конкретные значения.
 \end{itemize}

 
\end{frame}

\end{document}
