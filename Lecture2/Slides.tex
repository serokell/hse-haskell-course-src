\documentclass[10pt]{beamer}
\usepackage[cache=false]{minted}
\usepackage[utf8x]{inputenc}
\usepackage{hyperref}
\usepackage{fontawesome}
\usepackage{graphicx}
\usepackage[english,ngerman]{babel}
\usepackage{bussproofs}

% ------------------------------------------------------------------------------
% Use the beautiful metropolis beamer template
% ------------------------------------------------------------------------------
\usepackage[T1]{fontenc}
\usepackage{fontawesome}
\usepackage{FiraSans}
\newtheorem{defin}{Definition}
\newtheorem{theor}{Theorem}
\newtheorem{prop}{Proposition}
\mode<presentation>
{
  \usetheme[progressbar=foot,numbering=fraction,background=light]{metropolis}
  \usecolortheme{default} % or try albatross, beaver, crane, ...
  \usefonttheme{default}  % or try serif, structurebold, ...
  \setbeamertemplate{navigation symbols}{}
  \setbeamertemplate{caption}[numbered]
  %\setbeamertemplate{frame footer}{My custom footer}
}

% ------------------------------------------------------------------------------
% beamer doesn't have texttt defined, but I usually want it anyway
% ------------------------------------------------------------------------------
\let\textttorig\texttt
\renewcommand<>{\texttt}[1]{%
  \only#2{\textttorig{#1}}%
}

% ------------------------------------------------------------------------------
% minted
% ------------------------------------------------------------------------------


% ------------------------------------------------------------------------------
% tcolorbox / tcblisting
% ------------------------------------------------------------------------------
\usepackage{xcolor}
\definecolor{codecolor}{HTML}{FFC300}

\usepackage{tcolorbox}
\tcbuselibrary{most,listingsutf8,minted}

\tcbset{tcbox width=auto,left=1mm,top=1mm,bottom=1mm,
right=1mm,boxsep=1mm,middle=1pt}

\newtcblisting{myr}[1]{colback=codecolor!5,colframe=codecolor!80!black,listing only,
minted options={numbers=left, style=tcblatex,fontsize=\tiny,breaklines,autogobble,linenos,numbersep=3mm},
left=5mm,enhanced,
title=#1, fonttitle=\bfseries,
listing engine=minted,minted language=r}


% ------------------------------------------------------------------------------
% Listings
% ------------------------------------------------------------------------------
\definecolor{mygreen}{HTML}{37980D}
\definecolor{myblue}{HTML}{0D089F}
\definecolor{myred}{HTML}{98290D}

\usepackage{listings}

% the following is optional to configure custom highlighting
\lstdefinelanguage{XML}
{
  morestring=[b]",
  morecomment=[s]{<!--}{-->},
  morestring=[s]{>}{<},
  morekeywords={ref,xmlns,version,type,canonicalRef,metr,real,target}% list your attributes here
}

\lstdefinestyle{myxml}{
language=XML,
showspaces=false,
showtabs=false,
basicstyle=\ttfamily,
columns=fullflexible,
breaklines=true,
showstringspaces=false,
breakatwhitespace=true,
escapeinside={(*@}{@*)},
basicstyle=\color{mygreen}\ttfamily,%\footnotesize,
stringstyle=\color{myred},
commentstyle=\color{myblue}\upshape,
keywordstyle=\color{myblue}\bfseries,
}

\title{Functional programming, Seminar No. 2}
\author{Danya Rogozin \\ Institute for Information Transmission Problems, RAS \\ Serokell O\"{U}}
\date{Higher School of Economics \\ The Faculty of Computer Science}
\begin{document}

\maketitle

\begin{frame}
  \frametitle{Intro}

  On the previous seminar, we:

\onslide<1->{
  \begin{itemize}
    \item discussed the general aspects of Haskell
    \item took a look at the Haskell ecosystem
  \end{itemize}
  }

\vspace{\baselineskip}

\onslide<2->{
  Today, we:

  \begin{itemize}
    \item study the basic Haskell syntax
    \item study the list data type as one of the Haskell data structures
    \item realise why Haskell is a lazy language
  \end{itemize}
  }
\end{frame}

\begin{frame}[fragile]
  \frametitle{Bindings}

  The equality sign in Haskell denotes binding:

\metroset{block=fill}
\begin{exampleblock}{Example}
\begin{minted}{haskell}
      fortyTwo = 42
      coolString = "coolString"
\end{minted}
\end{exampleblock}

Local binding with the \verb"let"-keyword:
\metroset{block=fill}
\begin{exampleblock}{Example}
  \begin{minted}{haskell}
    fortyTwo = let number = 43 in number - 1
  \end{minted}
  \end{exampleblock}
\end{frame}

\begin{frame}[fragile]
  \frametitle{Function definitions}

  The following functions are also defined as bindings:

\metroset{block=fill}
\begin{exampleblock}{Example}
  \begin{minted}{haskell}
      add x y = x + y
      userName name = "Username: " ++ name
      id x = x
  \end{minted}
\end{exampleblock}

  \vspace{\baselineskip}

  The same functions defined with lambda:
  \metroset{block=fill}
  \begin{exampleblock}{Example}
  \begin{minted}{haskell}
    add = \x y -> x + y
    userName = \name -> "Username: " ++ name
    id = \x -> x
  \end{minted}
  \end{exampleblock}
\end{frame}

\begin{frame}[fragile]
  \frametitle{Function application}

  As in the lambda calculus, function application is left associative by default

  \metroset{block=fill}
  \begin{exampleblock}{Example}
  \begin{minted}{haskell}
    {-
    foo x y z = f x y z = ((f x) y) z
    -}
  \end{minted}
  \end{exampleblock}

\vspace{\baselineskip}

  One may use the dollar infix operator. That would allow us to avoid too many brackets.
  For example, the functions \verb"function" and \verb"function1" are equivalent:
  \metroset{block=fill}
  \begin{exampleblock}{Example}
  \begin{minted}{haskell}
    function f x y z = f ((x y) z)
    function1 f x y z = f $ x y $ z
  \end{minted}
  \end{exampleblock}
\end{frame}

\begin{frame}[fragile]
  \frametitle{Prefix and infix notation}
  Any operator or function might be called in prefix and infix:

  \metroset{block=fill}
  \begin{exampleblock}{Example}
  \begin{minted}{haskell}
> map (\x -> x * pi * 100) [1..3]
[314.1592653589793,628.3185307179587,942.4777960769379]
> (\x -> x * pi * 100) `map` [1..3]
[314.1592653589793,628.3185307179587,942.4777960769379]
  \end{minted}
  \end{exampleblock}
  One may declare an operator defining its priority and associativity explicitly. Here's an example:
\metroset{block=fill}
\begin{exampleblock}{Example}
\begin{minted}{haskell}
(^) :: (Num a, Integral b) => a -> b -> a
infixr 8 ^
\end{minted}
\end{exampleblock}
\end{frame}

\begin{frame}[fragile]
  \frametitle{Currying and partial application}

Recall the function \verb"add" once more. Here is an example of partial application:

\metroset{block=fill}
\begin{exampleblock}{Example}
\begin{minted}{haskell}
add x y = x + y
addFive = add 5
twentyEight = addFive 23
  -- 28
\end{minted}
\end{exampleblock}

Partial application is well-defined since all many-argument functions in Haskell are curried by default.

\end{frame}

\begin{frame}[fragile]
  \frametitle{Immutability and laziness}

In Haskell, values are immutable. A small example:

\metroset{block=fill}
\begin{exampleblock}{Example}
\begin{minted}{haskell}
  > list = [1,2,3,4]
  > reverse list
  [4,3,2,1]
  > list
  [1,2,3,4]
  > 10 : list
  [10,1,2,3,4]
  > list
  [1,2,3,4]
\end{minted}
\end{exampleblock}
\end{frame}

\begin{frame}[fragile]
  \frametitle{Recursion}

  The straighforward factorial and the tail-recursive one:
  \metroset{block=fill}
  \begin{exampleblock}{Example}
  \begin{minted}{haskell}
    factorial n
      = if n == 0 then 1 else n * factorial (n - 1)

    tailFactorial n = helper 1 n
      where
      helper acc x =
        if x > 1
        then helper (acc * x) (x - 1)
        else acc
  \end{minted}
  \end{exampleblock}
\end{frame}

\begin{frame}[fragile]
  \frametitle{Guards}

  Let us take a look at the factorial implementation via guards:
  \metroset{block=fill}
  \begin{exampleblock}{Example}
  \begin{minted}{haskell}
    tailFactorial n = helper 1 n
      where
      helper acc x | x > 1 = helper (acc * x) (x - 1)
                   | otherwise acc
  \end{minted}
  \end{exampleblock}
\end{frame}

\begin{frame}[fragile]
  \frametitle{Basic datatypes}
  The basic datatypes are:
  \begin{itemize}
    \item \verb"Bool"
    \item \verb"Int"
    \item \verb"Integer"
    \item \verb"Char"
    \item \verb"()"
    \item If \verb"a" and \verb"b" are types, then \verb"a -> b" is a type
    \item If \verb"a" and \verb"b" are types, then \verb"(a,b)" is a type
    \item If \verb"a" is a type, then \verb"[a]" is a type
  \end{itemize}

  A type declaration has the following form:

  \begin{minted}{haskell}
    term :: type
  \end{minted}
\end{frame}

\begin{frame}[fragile]
  \frametitle{Datatypes and constructors}
 We take the list of basic data types and associate constructors with these types. A constructor is a term that allows one to obtain a value of a given type.
\begin{center}
\begin{tabular}{ |c|c| }
\hline
\verb"Bool" & \verb"True" and \verb"False" \\
\verb"Int" & Integers from $-2^{29}$ to $2^{29} - 1$ \\
\verb"Integer" & The set of integers \\
\verb"Char" & Characters \verb"'0'", ..., \verb"'9'", \verb"'a'", ..., \verb"'z'", etc \\
\verb"()" & \verb"()" only \\
\verb"a -> b" & $\lambda x \rightarrow m$ \\
\verb"(a,b)" & if \verb"x :: a" and \verb"y :: b", then \verb"(x, y) :: (a,b)" \\
\verb"[a]" & the empty list \verb"[]" \\
\verb"[a]" & if \verb"x :: a" and \verb"xs :: [a]", then \verb"x : xs :: [a]" \\
\hline
\end{tabular}
\end{center}
\end{frame}

\begin{frame}[fragile]
  \frametitle{Types in GHCi}

  Use the GHCi command \verb":t" to know a type of an expression:

  \metroset{block=fill}
  \begin{exampleblock}{Example}
  \begin{minted}{haskell}
  > :t 5
  5 :: Num p => p
  > :t not
  not :: Bool -> Bool
  > :t [0.5, 0.6, 0.7]
  [0.5, 0.6, 0.7] :: Fractional a => [a]
  > :t (\x -> "dratuti, " ++ x)
  (\x -> "dratuti, " ++ x) :: [Char] -> [Char]
  > :t 'x'
  'x' :: Char
  \end{minted}
  \end{exampleblock}

\end{frame}

\begin{frame}[fragile]
  \frametitle{Function declaration with datatypes}

  Let us recall the examples of function declarations:
  \metroset{block=fill}
  \begin{exampleblock}{Example}
    \begin{minted}{haskell}
      add x y = x + y
      userName name = "Username: " ++ name
    \end{minted}
  \end{exampleblock}

    \vspace{\baselineskip}

    One may annotate these functions with type signatures as follows:
    \metroset{block=fill}
    \begin{exampleblock}{Example}
    \begin{minted}{haskell}
      add :: Int -> Int -> Int
      add x y = x + y

      userName :: String -> String
      userName name = "Username: " ++ name
    \end{minted}
    \end{exampleblock}
\end{frame}

\begin{frame}[fragile]
  \frametitle{Lists}

  Let's talk about lists. In Haskell, a list is a homogeneous collection of elements.
  \metroset{block=fill}
  \begin{exampleblock}{Example}
  \begin{minted}{haskell}
    empty :: [Int]
    empty = []

    ten :: [Int]
    ten = [10]

    tenEleven :: [Int]
    tenEleven = 11 : ten

    tenElevenTwelve :: [Int]
    tenElevenTwelve = 12 : tenEleven
    -- 12 : (11 : [])
  \end{minted}
  \end{exampleblock}

\end{frame}

\begin{frame}[fragile]
  \frametitle{Lists. Ranges}
  \metroset{block=fill}
  \begin{exampleblock}{Example}
  \begin{minted}{haskell}
  oneToFive :: [Int]
  oneToFive = [1..5]

  oneToSevenOdd :: [Int]
  oneToSevenOdd = [1,3..7]

  nat :: [Int]
  nat = [0,1..]

  evens :: [Int]
  evens = [0,2,4..]
  \end{minted}
  \end{exampleblock}
\end{frame}

\begin{frame}[fragile]
  \frametitle{Lists. Heads and Tails}
  \metroset{block=fill}
  \begin{exampleblock}{Example}
  \begin{minted}{haskell}
  > tail [1..3]
  [2,3]
  > head [1..3]
  1
  > head []
  *** Exception: Prelude.head: empty list
  > tail []
  *** Exception: Prelude.tail: empty list
  \end{minted}
    \end{exampleblock}
\onslide<2->{
\begin{center}
\includegraphics[scale=0.2]{Pics/Doge.png}
\end{center}}

\end{frame}

\begin{frame}[fragile]
  \frametitle{Other helpful list functions}

  \metroset{block=fill}
  \begin{exampleblock}{Example}
  \begin{minted}{haskell}
  Prelude> drop 3 [1..7]
  [4,5,6,7]
  Prelude> take 4 ['a'..'h']
  "abcd"
  Prelude> replicate 3 "d"
  ["d","d","d"]
  Prelude> replicate 3 'd'
  "ddd"
  Prelude> zip [1,2,3] "this is a word"
  [(1,'t'),(2,'h'),(3,'i')]
  Prelude> unzip [(1,'t'),(2,'h'),(3,'i')]
  ([1,2,3],"thi")
  Prelude> ['a'..'h'] !! 3
  'd'
  \end{minted}
  \end{exampleblock}
\end{frame}

\begin{frame}[fragile]
  \frametitle{List compeherension}
  \metroset{block=fill}
  \begin{exampleblock}{Example}
  \begin{minted}{haskell}
  > take 4 [(i, j) | i <- [1..10], j <- [1..10], i == j*j]
  [(1,1),(4,2),(9,3),(16,4)]
  > [ i | i <- "a cool sentence", i < 'h']
  "a c eece"
  > [ i | i <- "a cool sentence", fromEnum i < 100 ]
  "a c c"
  \end{minted}
  \end{exampleblock}
\end{frame}

\begin{frame}[fragile]
  \frametitle{Higher order functions}

  Function is a first-class object and one may pass any function as an argument:
  \metroset{block=fill}
  \begin{exampleblock}{Example}
  \begin{minted}{haskell}
  inc :: Int -> Int
  inc x = x + 1

  changeTwiceBy :: (Int -> Int) -> Int -> Int
  changeTwiceBy operation value
    = operation (operation value)

  seven :: Int
  seven = changeTwiceBy inc 5
  \end{minted}
  \end{exampleblock}
\end{frame}

\begin{frame}[fragile]
  \frametitle{\verb"Case"-expressions}

\verb"Case"-expressions allows one to perform case analysis within a function body.

\metroset{block=fill}
\begin{exampleblock}{Example}
  \begin{minted}{haskell}
  getFont :: Int -> String
  getFont n =
    case n of
      0 -> "PLAIN"
      1 -> "BOLD"
      2 -> "ITALIC"
      _ -> "UNKNOWN"
  \end{minted}
  \end{exampleblock}
\end{frame}

\begin{frame}
  \frametitle{Theoretical Flashback. Reduction strategies}

\onslide<1->{
  We recall a couple of definitions related to the lambda calculus:
}
\onslide<2->{
  \begin{enumerate}
    \item A term $M$ is called \emph{weakly normalisable} (WN), if there exists some halting reduction path that starts from $M$}
    \onslide<3->{\item A term $M$ is called \emph{strongly normalisable} (SN), if any reduction path that starts from $M$ terminates}
  \end{enumerate}

\onslide<4->{
  It is clear, that SN implies WN, not vice versa. In other words, there exists a term that has an infinite reduction path, but it has a finite reduction path at the same time.
  }
\end{frame}

\begin{frame}
  \frametitle{Theoretical Flashback. Reduction strategies}

\onslide<1->{
  Let us consider the example of the following ridiculous term: $(\lambda x y. x) (\lambda z. z) ((\lambda x. x x) (\lambda x. x x))$. One may reduce this term in two ways:
}

  \vspace{\baselineskip}
\onslide<2->{
From the one hand:

  $\begin{array}{lll}
  &(\lambda x y. x) (\lambda z. z) ((\lambda x. x x) (\lambda x. x x)) \rightarrow_{\beta} & \\
  &(\lambda y. [x := (\lambda z. z)]) ((\lambda x. x x) (\lambda x. x x)) \rightarrow_{\beta} & \\
  &(\lambda y. \lambda z. z) ((\lambda x. x x) (\lambda x. x x)) \rightarrow_{\beta}& \\
  &(\lambda z. z) [y := (\lambda x. x x) (\lambda x. x x)] \rightarrow_{\beta}& \\
  &\lambda z. z&
  \end{array}$
}
  \vspace{\baselineskip}

\onslide<3->{
  From the other hand:

  $\begin{array}{lll}
  & (\lambda x y. x) (\lambda z. z) ((\lambda x. x x) (\lambda x. x x)) \rightarrow_{\beta} & \\
  & (\lambda x y. x) (\lambda z. z) (x x)(x := [\lambda x. x x]) \rightarrow_{\beta}& \\
  & (\lambda x y. x) (\lambda z. z) ((\lambda x. x x) (\lambda x. x x)) \rightarrow_{\beta} \dots & \\
  & \text{ spasiti pamagiti pajalusta ya tak bolshe ni magu (((((((9(99( lol kek}&
  \end{array}$
}
\end{frame}

\begin{frame}
  \frametitle{Theoretical Flashback. Reduction strategies}
  \onslide<1->{
  Let us consider the example above in some other aspects.
  } \onslide<2->{
  \begin{itemize}
    \item In the first case, we have got a sensible result by several reduction steps. On the other hand, we have a loop in the second case.
    } \onslide<3->{
    \item Also, in the first case, we start our reduction from the leftmost innermost redex. But when we were trying to reduce the term $(\lambda x. x x) (\lambda x. x x)$, we have seen that something went wrong.
    } \onslide<4->{
    \item Can we distinguish all possible ways of term reduction?
    }
  \end{itemize}

\onslide<5->{
  In fact, we need to distinguish possible ways of application reduction, so far as we have no other options in the remaining cases:

  \begin{enumerate}
    \item If $x$ is a variable, then $x$ is already in normal form
    \item If a term has the form $\lambda x. M$, then we reduce $M$
  \end{enumerate}
  }
\end{frame}

\begin{frame}
  \frametitle{Theoretical Flashback. Reduction strategies}

\onslide<1->{
  Thus, one needs to overview of the possible ways of application reduction. We have two chairs:
} \onslide<2->{
  \begin{enumerate}
    \item $(\lambda x_1 \dots x_n. M) N_1 \dots N_n$: we firtsly reduce $(N_i)_{i \in \{ 1, \dots, n \}}$
    \item $(\lambda x_1 \dots x_n. M) N_1 \dots N_n$: reduce $(\lambda x. M) N_1$ and go further from left to right
  \end{enumerate}
} \onslide<3->{

  The first way is called \emph{applicative order reduction}, the second one is normal order reduction. In the following sense, the second is better:
} \onslide<4->{
  \begin{theorem}
    Let $M$ be a term such that $M$ has a normal form $M'$, then $M$ might be reduced to $M'$ with normal order reduction.
  \end{theorem}
  }
\end{frame}

\begin{frame}
  \frametitle{Theoretical Flashback. Call-by-value and call-by-name}

\begin{itemize}
  \onslide<1->{\item The applicative (normal) order is often called call-by-value (call-by-name)}
  \onslide<2->{\item The most mainstream programming languages you know (Java, Python, Kotlin, etc) have call-by-value semantics}
  \onslide<3->{\item The Haskell reduction has a call-by-need strategy which is quite close to call-by-name. Informally, such a stragety is called \emph{lazy}. Laziness denotes that Haskell doesn't compute a value if it's not needed at the moment}
  \onslide<4->{\item Call-by-name reduction reduces reducible terms to the bitter end, but it's not always optimal, unfortunately}
\end{itemize}
\end{frame}

\begin{frame}[fragile]
  \frametitle{Haskell reduction}

  Suppose we have the following trivial function:
  \metroset{block=fill}
  \begin{exampleblock}{Example}
  \begin{minted}{haskell}
  square :: Int -> Int
  square x = x * x
  \end{minted}
  \end{exampleblock}

\onslide<2->{
  If we call this function on $(1 + 2)$, then we would have the following story:

  $\operatorname{square} \: (1 + 2) = (1 + 2) * (1 + 2) = 3 * (1 + 2) = 3 * 3 = 9$
}
\onslide<3->{
  We evalutate $(1 + 2)$ twice, even if we know that $1 + 2 = 3$ a priori.} \onslide<4->{The question of performance is still relevant.
    \begin{center}
    \includegraphics[scale=0.25]{Pics/WhatToDo.jpeg}
    \end{center}
  }
\end{frame}

\begin{frame}[fragile]
  \frametitle{The notion of a weak head normal form}

  In Haskell, reduction evalutates a term to its weak head normal form, where the outermost must be either constructor or lambda. Here are examples: WHNFs from the left and non-WHNFs from the right

\begin{minipage}{0.5\textwidth}
  \begin{flushleft}
    \begin{minted}{haskell}
    78

    2 : [1,2]

    'p' : ("ri" ++ "vet")

    [1, 1 + 2, 1 + 3]

    ("hel" ++ "lo", "world")

    \x -> (x + 2) + 2
    \end{minted}
  \end{flushleft}
\end{minipage}\hfill
\begin{minipage}{0.5\textwidth}
  \begin{flushright}
    \begin{minted}{haskell}
    1 + 665

    (\x -> x ++ "ab") "cd"

    length [1..145]

    (\f g x -> f (g x)) id
    \end{minted}
  \end{flushright}
\end{minipage}
\end{frame}

\section{Pure functions and side-effects}

\begin{frame}
  \frametitle{Pure functions and side-effects}

  \begin{itemize}
    \onslide<1->{\item A function is called \emph{pure} if it has no side effects}
    \onslide<2->{\item It means that such a function behaves 'in the same way' at every point. This principle is also called \emph{referential transparency}
    }
    \onslide<3->{\item A side-effect function is a function that may yield different values passing the same arguments. Mathematically, such a function is not function at all.
    } \onslide<4->{
    \item Haskell functions are (mostly) pure ones, but Haskell is not confluent as a version of the lambda calculus
    }
  \end{itemize}
\end{frame}

\begin{frame}[fragile]
  \frametitle{The failure of the Church-Rosser property}

  Let us consider the following quite simple example. In Haskell one has a function called \verb"seq". According to Hackage, ``The value of \verb"seq a b" is bottom if a is bottom, and otherwise equal to b.'' This function is a sort of instrument to introduce the restricted strictness to Haskell. The listing below demostrates the failure of the CRP:

  \begin{minted}{haskell}
  seq :: a -> b -> b
  seq _|_ _ = _|_
  seq _ b   = b

  bottom = undefined

  seq bottom 14         == bottom
  seq (bottom  . id) 14 == 14
  \end{minted}

\end{frame}

\begin{frame}
  \frametitle{Finally}

\onslide<1->{
  On this seminar, we
  \begin{itemize}
    \item got acquinted with the basic Haskell syntax and basic data types
    \item discussed pure functions and the example of the confluence failure
  \end{itemize}
}

\onslide<2->{
On the next seminar, we will
\begin{itemize}
  \item start to learn polymorphism and its advantages
  \item introduce typeclasses
  \item study the very first examples of typeclasses
\end{itemize}
}
\end{frame}

\begin{frame}
  \frametitle{Thank you!}
\end{frame}

\end{document}
