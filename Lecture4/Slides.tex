\documentclass[10pt,pdf,utf8,russian,aspectratio=169]{beamer}
\usepackage[T2A]{fontenc}
\usetheme{Singapore}
\usepackage{setspace}
\usepackage{amsmath}
\usepackage{pgfplots}
\usepackage[utf8]{inputenc}
\usepackage{tikz-cd}
\usepackage[all, 2cell]{xy}
\usepackage{amssymb}
\usepackage{verba tim}
\usepackage[all]{xy}
\usepackage{tikz}
\usepackage{bussproofs}
\usepackage{dsfont}
\usepackage{mathabx}
\usepackage{animate}
\usetikzlibrary{graphs}
\usetikzlibrary{arrows}
\usepackage{hyperref}
\usepackage[english,russian]{babel}
\usepackage{listings}
\usepackage{color}
\usepackage{tikz}
\usepackage{listings}
\makeatletter
\newcommand{\xRightarrow}[2][]{\ext@arrow 0359\Rightarrowfill@{#1}{#2}}
\makeatother
\pgfplotsset{compat=1.16}
\newtheorem{defin}{Definition}
\newtheorem{theor}{Theorem}
\newtheorem{prop}{Proposition}
\title{Functional programming, Seminar No. 4}
\author{Danya Rogozin \\ Lomonosov Moscow State University, \\ Serokell O\"{U}}
\date{Higher School of Economics \\ Faculty of Computer Science}
\begin{document}
\lstset{
  frame=none,
  xleftmargin=2pt,
  stepnumber=1,
  numbers=left,
  numbersep=5pt,
  numberstyle=\ttfamily\tiny\color[gray]{0.3},
  belowcaptionskip=\bigskipamount,
  captionpos=b,
  escapeinside={*'}{'*},
  language=haskell,
  tabsize=2,
  emphstyle={\bf},
  commentstyle=\it,
  stringstyle=\mdseries\rmfamily,
  showspaces=false,
  keywordstyle=\bfseries\rmfamily,
  columns=flexible,
  basicstyle=\small\sffamily,
  showstringspaces=false,
  morecomment=[l]\%,
}
\maketitle

\begin{frame}
  \frametitle{Intro}

  On the previous seminar, we
  \begin{itemize}
    \item introduced parametric polymorphism
    \item discussed type classes and their examples (\verb"Show", \verb"Eq", \verb"Ord", etc)
  \end{itemize}

\vspace{\baselineskip}

\onslide<2->{
  Today, we
  \begin{itemize}
    \item study pattern matching and such type constructions as algebraic data types, new types, type synonyms, and records
    \item learn folds
    \item talk about lazy evalution enforcing more systematically
  \end{itemize}
}

\end{frame}

\begin{frame}[fragile]
\frametitle{Pattern matching}
  Let us take a look at the following functions:
  \begin{lstlisting}[language=Haskell]
  swap :: (a, b) -> (b, a)
  swap (a, b) = (b, a)

  length :: [a] -> Int
  length [] = 0
  lenght (x : xs) = 1 + length xs
  \end{lstlisting}

\onslide<2->{
  \begin{itemize}
    \item Expressions like \verb"(a,b)", \verb"[]", and \verb"(x : xs)" are often called patterns
    \item In such calls as \verb"swap (45, True)" or \verb"lenght [1,2,3]", we deal with \emph{pattern matching}}
\onslide<3->{
    \item One needs to check that constructors \verb"(,)" and \verb"( : )" are relevant.
    \item In the call \verb"swap (45, True)", variables \verb"a" and \verb"b" are binded with the values \verb"45" and \verb"True".
    \item In the call \verb"lenght [1,2,3]", variables \verb"x" and \verb"xs" are binded with the values \verb"1" and \verb"[2,3]"
  \end{itemize}
  }
\end{frame}

\begin{frame}[fragile]
\frametitle{Algebraic data types. Enumerations}

\begin{itemize}
\item The simplest example of an algebraic data type is a data type defined with an enumeration of constructors that stores no values.

\begin{lstlisting}[language=Haskell]
data Colour = Red | Blue | Green | Purple | Yellow
  deriving (Show, Eq)
\end{lstlisting}

\item The example of pattern matching for this data type

\begin{lstlisting}[language=Haskell]
isRGB :: Colour -> Bool
isRGB Red  = True
isRGB Blue = True
isRGB Green = True
isRGB _     = False       -- Wild-card
\end{lstlisting}
\end{itemize}
\end{frame}

\begin{frame}[fragile]
\frametitle{Algebraic data types. Products}

\begin{itemize}
\item The example of a product data type:

\begin{lstlisting}[language=Haskell]
data Point = Point Double Double
  deriving Show
\end{lstlisting}

\begin{lstlisting}[language=Haskell]
> :type Point
Point :: Double -> Double -> Point
\end{lstlisting}
\item The example of a function

\begin{lstlisting}[language=Haskell]
taxCab :: Point -> Point -> Double
taxCab (Point x1 y1) (Point x2 y2) =
  abs (x1 - x2) + abs (y1 - y2)
\end{lstlisting}

\item The example in a GHCi session
\begin{lstlisting}[language=Haskell]
> taxCab (Point 3.0 5.0) (Point 7.0 9.0)
8.0
\end{lstlisting}

\end{itemize}
\end{frame}

\begin{frame}[fragile]
  \frametitle{Polymorphic data types}

\begin{itemize}
  \item The point data type might parametrised with a type parameter:
  \begin{lstlisting}[language=Haskell]
  data Point a = Point a a
    deriving Show
  \end{lstlisting}

  \item The type of the \verb"Point" constructor
  \begin{lstlisting}[language=Haskell]
  > :type Point
  Point :: a -> a -> Point a
  \end{lstlisting}
  \item \verb"Point" is a type operator. One also has a type (kind) system for type operators:
  \begin{lstlisting}[language=Haskell]
  > :k Point
  Point :: * -> *
  \end{lstlisting}
\end{itemize}
\end{frame}

\begin{frame}[fragile]
  \frametitle{Polymorphic data types and type classes}

\begin{itemize}
  \item Suppose we have a function:

  \begin{lstlisting}[language=Haskell]
 midPoint :: Fractional a => Point a -> Point a -> Point a
 midPoint (Pt x1 y1) (Pt x2 y2) = Pt ((x1 + x2) / 2) ((y1 + y2) / 2)
  \end{lstlisting}
  \item Playing with GHCi:
  \begin{lstlisting}[language=Haskell]
  > :t midPoint (Pt 3 5) (Pt 6 4)
  midPoint (Pt 3 5) (Pt 6 4) :: Fractional a => Point a
  > midPoint (Pt 3 5) (Pt 6 4)
  Pt 4.5 4.5
  > :t it
  it :: Fractional a => Point a
  \end{lstlisting}
  \onslide<2->{
  \item The type of point is a polymorphic itself. But one needs to use ad hoc polymorphism (the \verb"Fractional" context) to apply division.
  \item On the other hand, polymorphism here is ambiguous. The fractional type is \verb"Double" by default. Haskell has a defaulting mechanism for numerical data types
  }
\end{itemize}
\end{frame}

\begin{frame}[fragile]
  \frametitle{Inductive data types}

\begin{itemize}
  \item The list is the first example of an inductive data type

  \begin{lstlisting}[language=Haskell]
  data List a = Nil | Cons a (List a)
    deriving Show
  \end{lstlisting}
  \item The data constructors are \verb"Nil :: List a" and \verb"Cons :: a -> List a -> List a"
  \item Here come pattern matching and recursion
  \begin{lstlisting}[language=Haskell]
  concat :: List a -> List a -> List a
  concat Nil ys = ys
  concat (Cons x xs) ys = Cons x (xs `concat` ys)
  \end{lstlisting}
  \item The GHCi session:
  \begin{lstlisting}[language=Haskell]
  > x = Cons 'a' (Cons 'b' Nil)
  > y = Cons 'c' (Cons 'd' Nil)
  > concat x y
  Cons 'a' (Cons 'b' (Cons 'c' (Cons 'd' Nil)))
  \end{lstlisting}
\end{itemize}
\end{frame}

\begin{frame}[fragile]
  \frametitle{Standard lists}

  \begin{itemize}
    \item The list data type is a default one, but its approximate definition is the following one:
    \begin{lstlisting}[language=Haskell]
    infixr 5 :
    data [] a = [] | a : ([] a)
      deriving Show
    \end{lstlisting}
    \item Some syntax sugar
    \begin{lstlisting}[language=Haskell]
    [1,2,3,4] == 1 : 2 : 3 : 4 : []
    \end{lstlisting}
    \item The example of a definition with built-in lists:
    \begin{lstlisting}[language=Haskell]
    infixr 5  ++
    (++) :: [a] -> [a] -> [a]
    (++) []     ys = ys
    (++) (x:xs) ys = x : xs ++ ys
    \end{lstlisting}
  \end{itemize}
\end{frame}

\begin{frame}[fragile]
  \frametitle{\verb"case ... of ..." expressions}
  \begin{itemize}
    \item \verb"case ... of ..." expressions allows one to perform pattern matching everywhere
    \begin{lstlisting}[language=Haskell]
    filter :: (a -> Bool) -> [a] -> [a]
    filter p [] = []
    filter p (x : xs) =
      case p x of
        True  -> x : filter p xs
        False -> filter p xs
    \end{lstlisting}
    \item The pattern matching from the previous slide is a syntax sugar for the corresponding \verb"case ... of ..." expression
  \end{itemize}
\end{frame}

\begin{frame}[fragile]
  \frametitle{Semantical aspects of pattern matching}
  \begin{itemize}
    \item Pattern matching is performed from up to down and from left to right after that.
    \item Pattern matching is either
    \begin{itemize}
      \item succeed
      \item or failed
      \item or diverged
    \end{itemize}
    \item Here is an example:
    \begin{lstlisting}[language=Haskell]
    foo (1,4) = 7
    foo (0,_) = 8
    \end{lstlisting}
    \item \verb"(0, undefined)" fails in the first case and it's succeed in the second one
    \item \verb"(undefined, 0)" is diverged automatically
    \item \verb"(2,1)" is a diverged pattern
    \item What about \verb"(1,7-3)"?
  \end{itemize}
\end{frame}

\begin{frame}[fragile]
  \frametitle{As-patterns}
  \begin{itemize}
  \item Suppose we have the following function
  \begin{lstlisting}[language=Haskell]
  dupHead :: [a] -> [a]
  dupHead (x : xs) = x : x : xs
  \end{lstlisting}
  \item One may rewrite this function as follows:
  \begin{lstlisting}[language=Haskell]
  dupHead :: [a] -> [a]
  dupHead s@(x : xs) = x : s
  \end{lstlisting}
  \item Here, the name \verb"s" is assigned to the whole pattern \verb"x : xs"
  \item In fact, such a construction is a syntax sugar for the following one. It is not so hard to ensure that both functions have the same behaviour
  \begin{lstlisting}[language=Haskell]
  dupHead :: [a] -> [a]
  dupHead (x : xs) =
    let s = (x : xs) in x : s
  \end{lstlisting}
\end{itemize}
\end{frame}

\begin{frame}[fragile]
  \frametitle{Irrefutable patterns}
  \begin{itemize}
    \item Irrefutable patterns are wild-cards, variables, and lazy patterns
    \item The example of a lazy pattern:
    \begin{lstlisting}[language=Haskell]
    > (***) f g (a,b) = (f a, g b)
    > const 2 *** const 1 $ undefined
    *** Exception: Prelude.undefined
    > (***) f g ~(a,b) = (f a, g b)
    > const 2 *** const 1 $ undefined
    (2,1)
    \end{lstlisting}
  \end{itemize}
\end{frame}

\begin{frame}[fragile]
\frametitle{The \verb"newtype" and \verb"type" declarations}

\begin{itemize}
  \item The keyword \verb"type" allows one to introduce type synonyms. The example given
  \begin{lstlisting}[language=Haskell]
  type String = [Char]
  \end{lstlisting}
  \item In Haskell, the string data type type is merely a type synonym for the list of characters
  \item The keyword \verb"newtype" defines a new type with the single constructor that packs an existing types
  \begin{lstlisting}[language=Haskell]
  newtype Age = Age Int
  \end{lstlisting}
  \item The same type \verb"Age" defined with the equipped function \verb"runAge"
  \begin{lstlisting}[language=Haskell]
  newtype Age = Age { runAge :: Int }
  \end{lstlisting}
  \item The type of \verb"runAge"
  \begin{lstlisting}[language=Haskell]
  > :t runAge
  runAge :: Age -> Int
  \end{lstlisting}
\end{itemize}
\end{frame}

\begin{frame}[fragile]
  \frametitle{Field labels}

  \begin{itemize}
    \item Sometimes product data types are too cumbersome:
    \begin{lstlisting}[language=Haskell]
    data Person = Person String String Int Float String
    \end{lstlisting}
    \item As an alternative, one may define a data type with field labels
    \begin{lstlisting}[language=Haskell]
    data Person =
      Person { firstName :: String
             , lastName :: String
             , age :: Int
             , height :: Float
             , phoneNumber :: String
             }
     \end{lstlisting}
     \item Such a data type is a record with accessors, e.g.
     \verb"firstName :: Person -> String"
     \item In fact, this data type is a product data type with accessor function
  \end{itemize}
\end{frame}

\begin{frame}[fragile]
  \frametitle{Field labels and type classes}

  \begin{itemize}
    \item Let us recall the \verb"Eq" type class once more
    \begin{lstlisting}[language=Haskell]
    class Eq a where
      (==) :: a -> a -> Bool
      (/=) :: a -> a -> Bool

    instance Eq Int where
      x == y = x `eqInt` y

    isZero :: Int -> Bool
    isZero x = if x == y then True else False

    eqFunction :: Eq a => a -> a -> Int
    eqFunction x y =
      case x == y of
        True -> 42
        False -> 0
    \end{lstlisting}
    \item In fact, type classes are syntax sugar for records defined with field labels
    \item The constraint \verb"Eq a" is an additional argument
  \end{itemize}
\end{frame}

\begin{frame}[fragile]
  \frametitle{Field labels and type classes}

  \begin{itemize}
    \item The previous listing has the following meaning (very roughly):
    \begin{lstlisting}[language=Haskell]
    data Eq a =
      Eq { eq :: a -> a -> Bool
         , neq :: a -> a -> Bool
         }

    intInstance :: Eq Int
    intInstance = Eq eqInt (\x y -> not $ x `eqInt` y)

    isZero :: Int -> Bool
    isZero x = if (eq eqInstance) x 0 then True else False

    eqFunction :: Eq a -> a -> a -> Int
    eqFunction eqInst x y =
      case ((eq eqInst) x y) of
        True -> 42
        False -> 0
    \end{lstlisting}
  \end{itemize}
\end{frame}

\begin{frame}[fragile]
  \frametitle{Some of standard algebraic data types}
  \begin{itemize}
    \item The \verb"Maybe a" data type allows one to define an optional value:
    \begin{lstlisting}[language=Haskell]
    data Maybe a = Nothing | Just a

    maybe :: b -> (a -> b) -> Maybe a -> b
    maybe b _ Nothing = b
    maybe b f (Just x) = f x
    \end{lstlisting}
    \item The simple example given
    \begin{lstlisting}[language=Haskell]
    safeHead :: [a] -> Maybe a
    safeHead [] = Nothing
    safeHead (x : _) = Just x
    \end{lstlisting}
    \begin{lstlisting}[language=Haskell]
    \item The GHCi session:
    > maybe (maxBound :: Int) (+ 176) (safeHead [])
    9223372036854775807
    > maybe (maxBound :: Int) (+ 176) (safeHead [1..1500])
    177
    \end{lstlisting}
  \end{itemize}
\end{frame}

\begin{frame}[fragile]
  \frametitle{Some of standard algebraic data types}
  \begin{itemize}
    \item The \verb"Either" data type describes one or the other value

    \begin{lstlisting}[language=Haskell]
    data Either e a = Left e | Right a

    either :: (a -> c) -> (b -> c) -> Either a b -> c
    either f _ (Left x) = f x
    either _ g (Right x) = g x
    \end{lstlisting}
    \item The example given:
    \begin{lstlisting}[language=Haskell]
    safeTail :: [a] -> Either String [a]
    safeTail [] = Left "I have no tail, mate"
    safeTail (_ : xs) = Right xs
    \end{lstlisting}
    \item The GHCi example
    \begin{lstlisting}[language=Haskell]
    > either id (map succ) (safeTail [])
    "I have no tail, mate"
    > either id (map succ) (safeTail "\USdmbqxos\USld+\USokd`rd")
    "encrypt me, please"
    \end{lstlisting}
  \end{itemize}
\end{frame}

\begin{frame}[fragile]
  \frametitle{Folds and lists. Motivation}

\begin{itemize}
  \item Suppose we have these functions
  \begin{lstlisting}[language=Haskell]
  sum :: [Integer] -> Integer
  sum [] = 0
  sum (x : xs) = x + sum xs

  product :: [Integer] -> Integer
  product [] = 1
  product (x : xs) = x * product xs

  concat :: [[a]] -> [a]
  concat [] = []
  concat (x : xs) = x ++ concat xs
  \end{lstlisting}
  \item It is clear that one has a common recursion pattern
\end{itemize}
\end{frame}

\begin{frame}[fragile]
\frametitle{The definition of a right fold}
\begin{itemize}
  \item The definition of a right is the following one
  \begin{lstlisting}[language=Haskell]
  foldr :: (a -> b -> b) -> b -> [a] -> b
  foldr _ ini [] = []
  foldr f ini (x : xs) = f x (foldr f ini xs)
  \end{lstlisting}
  \item Informally, this function behaves as follows:
  \begin{lstlisting}[language=Haskell]
  foldr f z [x1, x2, ..., xn] == x1 `f` (x2 `f` ... (xn `f` z)...)
  \end{lstlisting}
\end{itemize}
\end{frame}

\begin{frame}
  \frametitle{The definition of a right fold}
One may visualise the story above for some list \verb"[a,b,c]". The list from the left and its right fold from the right:

\begin{small}
\xymatrix{
 & \verb":" \ar[dl] \ar[dr] &&&&& \verb"f" \ar[dl] \ar[dr] \\
 \verb"a" && \verb":" \ar[dl] \ar[dr] &&& \verb"a" && \verb"f" \ar[dl] \ar[dr] \\
 & \verb"b" && \verb":" \ar[dl] \ar[dr] &&& \verb"b" && \verb"f"  \ar[dl] \ar[dr] \\
 && \verb"c" && \verb"[]" &&& \verb"c" && \verb"ini"
}
\end{small}
\end{frame}

\begin{frame}[fragile]
  \frametitle{Functions \verb"sum", \verb"product", and \verb"concat" via \verb"foldr"}
  \begin{itemize}
    \item Let us rewrite those functions with \verb"foldr"
    \begin{lstlisting}[language=Haskell]
    sum :: [Integer] -> Integer
    sum = foldr (+) 0

    product :: [Integer] -> Integer
    product = foldr (*) 1

    concat :: [[a]] -> [a]
    concat = foldr (++) []
    \end{lstlisting}
    \item What about \verb"foldr (:) []"?
  \end{itemize}
\end{frame}

\begin{frame}[fragile]
  \frametitle{The universal property of a right fold}
\begin{block}{The universal property}
  Let \verb"f" be a function defined by the following equations:
  \begin{lstlisting}[language=Haskell]
  g [] = v
  g (x : xs) = f x (g xs)
  \end{lstlisting}

  then one has $\forall \: \verb"xs :: [a]" \:\: (\verb"g xs" \equiv \verb"foldr f v xs")$
\end{block}
\begin{itemize}
  \item The universal property is proved by induction on \verb"xs"
  \item The converse implication is quite trivial
  \item The meaning of this fact: \verb"foldr f v" and \verb"g" are interchangeable in this case
\end{itemize}
\end{frame}

\begin{frame}[fragile]
  \frametitle{The definition of a left fold}
  \begin{itemize}
    \item In addition to a right fold, one also has a left one
    \begin{lstlisting}[language=Haskell]
    foldl :: (b -> a -> b) -> b -> [a] -> b
    foldl _ ini [] = ini
    foldl f ini (x : xs) = foldl f (f ini x) xs
    \end{lstlisting}
    \item Informally:
    \begin{lstlisting}[language=Haskell]
    foldl f ini [x1, x2, ..., xn] == (...((z `f` x1) `f` x2) `f`...) `f` xn
    \end{lstlisting}
    \onslide<2->{
    \item The implementation of the left fold function might be optimised. Here we have an increasing thunk
    \item We take a look at the strict version of \verb"foldl"
    \item \verb"foldl" is the most optimal function, but we are not capable of processing infinite lists using the left fold function.
    }
  \end{itemize}
\end{frame}

\begin{frame}
  \frametitle{The definition of a left fold}
  One may visualise left fold in the same manner:
  \xymatrix{
  & \verb":" \ar[dl] \ar[dr] &&&&&&&& \verb"f" \ar[dl] \ar[dr] \\
  \verb"a" && \verb":" \ar[dl] \ar[dr] &&&&&& \verb"f" \ar[dl] \ar[dr]  && \verb"c" \\
  & \verb"b" && \verb":" \ar[dl] \ar[dr] &&&& \verb"f" \ar[dl] \ar[dr] && \verb"b" \\
  && \verb"c" && \verb"[]" && \verb"ini" && \verb"a"
  }
\end{frame}

\begin{frame}[fragile]
  \frametitle{Are \verb"foldr" and \verb"foldl" equivalent?}

\begin{itemize}
  \item Note that \verb"foldr" and \verb"foldl" are not equivalent to each other
  \begin{lstlisting}[language=Haskell]
  > foldl (/) 64 [4,2,4]
  2.0
  > foldr (/) 64 [4,2,4]
  0.125
  > foldl (\x y -> 2*x + y) 4 [1,2,3]
  43
  > foldr (\x y -> 2*x + y) 4 [1,2,3]
  16
  \end{lstlisting}
  \item \verb"foldr" and \verb"foldl" are equivalent if the folding operation is commutative
\end{itemize}
\end{frame}

\begin{frame}[fragile]
  \frametitle{The right scan}
  \begin{itemize}
    \item The right scan is the foldr that yields a list that contains all intermediate values
    \begin{lstlisting}[language=Haskell]
    scanr :: (a -> b -> b) -> b -> [a] -> [b]
    scanr _ ini [] = [ini]
    scanr f ini (x:xs) = f x q : qs
      where qs@(q:_) = scanr f ini xs
    \end{lstlisting}
    \item \verb"foldr" and \verb"scanr" are connected with each other as follows
    \begin{center}
      $\verb"head (scanr f z xs)" \equiv \verb"foldr f z xs"$
    \end{center}
    \item The examples are
    \begin{lstlisting}[language=Haskell]
    >  scanr (++) "!" ["aa","bb","cc"]
    ["aabbcc!","bbcc!","cc!","!"]
    > scanr (:) [] [1,2,3]
    [[1,2,3],[2,3],[3],[]]
    > scanr (+) 0 [1..10]
    [55,54,52,49,45,40,34,27,19,10,0]
    > scanr (*) 1 [1..10]
    [3628800,3628800,1814400,604800,151200,30240,5040,720,90,10,1]
    \end{lstlisting}
  \end{itemize}
\end{frame}

\begin{frame}[fragile]
  \frametitle{The left scan}
  \begin{itemize}
  \item One also has a scan function for the \verb"foldl" function:
  \begin{lstlisting}[language=Haskell]
  scanl :: (b -> a -> b) -> b -> [a] -> [b]
  scanl f q ls = q : (case ls of
                        []   -> []
                        x:xs -> scanl f (f q x) xs)
  \end{lstlisting}
  \item \verb"foldl" and \verb"scanl" are connected with each other as follows:
  \begin{center}
   $\verb"last (scanl f z xs)" \equiv \verb"foldl f z xs"$
  \end{center}
  \item The examples:
  \begin{lstlisting}[language=Haskell]
  > scanl (++) "!" ["a","b","c"]
  ["!","!a","!ab","!abc"]
  > scanl (*) 1 [1..] !! 5
  120
  \end{lstlisting}
  \end{itemize}
\end{frame}

\begin{frame}[fragile]
  \frametitle{The presence of a bottom}

  \begin{itemize}
    \item Any well-written expression in Haskell has a type
    \item Prima facie, the Bool data type has two values: \verb"False" and \verb"True" according to its definition:
    \begin{lstlisting}[language=Haskell]
    data Bool = False | True
    \end{lstlisting}
    \item One may define an expession \verb"dno :: Bool" which is recursively defined as \verb"dno = not dno"
    \item \verb"dno" is neither \verb"False" nor \verb"True", but it's a Boolean value!
    \item This value is a bottom ($\bot$). In Haskell, $\bot$ is a value that has a type \verb"forall a. a". Such errors as \verb"undefined" have this type.
  \end{itemize}
\end{frame}

\begin{frame}
  \frametitle{Strict function}
  \begin{itemize}
    \item Haskell has the call-by name semantics. That's the reason according to which \verb"const 42 undefined" yields \verb"42"
    \item Lazy functions are non-strict ones
    \onslide<2->{
    \item In constrast to lazy functions, strict functions satisfy this equivation
    \begin{center}
      $\verb"f" \: \bot = \bot$
    \end{center}
    \item That is, the strict \verb"const" should yield \verb"undefined" in the call \verb"const 42 undefined"
    }
  \end{itemize}
\end{frame}

\begin{frame}[fragile]
  \frametitle{Strictness in Haskell. The \verb"seq" function}
\begin{itemize}
  \item We took a look at the \verb"seq" function. Let us recall it.
  \item \verb"seq" is a combinator that enforce a computation.
  \item This combinator has a type $a \to b \to b$.
  \item It seems that the body of \verb"seq" looks like $\setminus\verb"x y -> y"$, but \verb"seq" satisfies the following equations:
  \begin{center}
    $\verb"seq" \: \bot \: x = \bot$

    $\verb"seq" \: value \: x = x$
  \end{center}
  \item Such an enforcing breaks the lazy semantics of Haskell! But this enforcing is not so far-reaching. Data constructors and lambdas put a barrier for the $\bot$ expansion:
  \begin{lstlisting}[language=Haskell]
  > seq (4,undefined) 5
  5
  > seq (\x -> undefined) 5
  5
  > seq (id . undefined) 5
  5
  \end{lstlisting}
\end{itemize}
\end{frame}

\begin{frame}[fragile]
  \frametitle{Strictness in Haskell. The strict application}
  \begin{itemize}
    \item One may implement the strict appication using \verb"seq"
    \begin{lstlisting}[language=Haskell]
    infixr 0 $!
    ($!) :: (a -> b) -> a -> b
    f $! x = x `seq` f x
    \end{lstlisting}
    \item That is, this application behaves as usual if the second argument is not bottom.
  \end{itemize}
\end{frame}

\begin{frame}[fragile]
  \frametitle{Strictness in Haskell. The strict application}
  \begin{itemize}
    \item Let us recall the tail-recursive factorial:
    \begin{lstlisting}[language=Haskell]
      tailFactorial n = helper 1 n
        where
        helper acc x =
          if x > 1
          then helper (acc * x) (x - 1)
          else acc
    \end{lstlisting}
    \item The optimal version of the tail-recursive factorial is the following one:
    \begin{lstlisting}[language=Haskell]
    tailFactorial n = helper 1 n
      where
      helper acc x =
        if x > 1
        then (helper $! (acc * x)) (x - 1)
        else acc
      \end{lstlisting}
  \end{itemize}
\end{frame}

\begin{frame}[fragile]
  \frametitle{The strict \verb"foldl"}

  \begin{itemize}
    \item A strict version of \verb"foldl"
    \begin{lstlisting}[language=Haskell]
    foldl' :: (a -> b -> a) -> a -> [b] -> a
    foldl' f ini [] = ini
    foldl' f ini (x:xs) = foldl' f arg xs
      where arg = (f ini) $! x
    \end{lstlisting}
  \end{itemize}
\end{frame}

\begin{frame}[fragile]
  \frametitle{Strictness in Haskell. Bang patterns}
  \begin{itemize}
    \item A data type might contain strict values with the strictness flag \verb"!", e.g.
    \begin{lstlisting}[language=Haskell]
    data Complex a = !a :+ !a
      deriving Show
    infix 6 :+

    im :: Complex a -> a
    im (x :+ y) = y
    \end{lstlisting}
    > im (undefined :+ 5)
    *** Exception: Prelude.undefined
    \begin{lstlisting}[language=Haskell]
    \end{lstlisting}

  \item The \verb"BangPatterns" allows one to make pattern a strict one
  \begin{lstlisting}[language=Haskell]
  > foo !x = True
  > foo undefined
  *** Exception: Prelude.undefined
  \end{lstlisting}
  \end{itemize}
\end{frame}

\end{document}
