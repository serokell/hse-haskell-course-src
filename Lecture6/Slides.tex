\documentclass[10pt]{beamer}
\usepackage[cache=false]{minted}
\usepackage[utf8x]{inputenc}
\usepackage{hyperref}
\usepackage{fontawesome}
\usepackage{graphicx}
\usepackage[english,ngerman]{babel}
\usepackage{bussproofs}

% ------------------------------------------------------------------------------
% Use the beautiful metropolis beamer template
% ------------------------------------------------------------------------------
\usepackage[T1]{fontenc}
\usepackage{fontawesome}
\usepackage{FiraSans}
\newtheorem{defin}{Definition}
\newtheorem{theor}{Theorem}
\newtheorem{prop}{Proposition}
\mode<presentation>
{
  \usetheme[progressbar=foot,numbering=fraction,background=light]{metropolis}
  \usecolortheme{default} % or try albatross, beaver, crane, ...
  \usefonttheme{default}  % or try serif, structurebold, ...
  \setbeamertemplate{navigation symbols}{}
  \setbeamertemplate{caption}[numbered]
  %\setbeamertemplate{frame footer}{My custom footer}
}

% ------------------------------------------------------------------------------
% beamer doesn't have texttt defined, but I usually want it anyway
% ------------------------------------------------------------------------------
\let\textttorig\texttt
\renewcommand<>{\texttt}[1]{%
  \only#2{\textttorig{#1}}%
}

% ------------------------------------------------------------------------------
% minted
% ------------------------------------------------------------------------------


% ------------------------------------------------------------------------------
% tcolorbox / tcblisting
% ------------------------------------------------------------------------------
\usepackage{xcolor}
\definecolor{codecolor}{HTML}{FFC300}

\usepackage{tcolorbox}
\tcbuselibrary{most,listingsutf8,minted}

\tcbset{tcbox width=auto,left=1mm,top=1mm,bottom=1mm,
right=1mm,boxsep=1mm,middle=1pt}

\newtcblisting{myr}[1]{colback=codecolor!5,colframe=codecolor!80!black,listing only,
minted options={numbers=left, style=tcblatex,fontsize=\tiny,breaklines,autogobble,linenos,numbersep=3mm},
left=5mm,enhanced,
title=#1, fonttitle=\bfseries,
listing engine=minted,minted language=r}


% ------------------------------------------------------------------------------
% Listings
% ------------------------------------------------------------------------------
\definecolor{mygreen}{HTML}{37980D}
\definecolor{myblue}{HTML}{0D089F}
\definecolor{myred}{HTML}{98290D}

\usepackage{listings}

% the following is optional to configure custom highlighting
\lstdefinelanguage{XML}
{
  morestring=[b]",
  morecomment=[s]{<!--}{-->},
  morestring=[s]{>}{<},
  morekeywords={ref,xmlns,version,type,canonicalRef,metr,real,target}% list your attributes here
}

\lstdefinestyle{myxml}{
language=XML,
showspaces=false,
showtabs=false,
basicstyle=\ttfamily,
columns=fullflexible,
breaklines=true,
showstringspaces=false,
breakatwhitespace=true,
escapeinside={(*@}{@*)},
basicstyle=\color{mygreen}\ttfamily,%\footnotesize,
stringstyle=\color{myred},
commentstyle=\color{myblue}\upshape,
keywordstyle=\color{myblue}\bfseries,
}


% ------------------------------------------------------------------------------
% The Document
% ------------------------------------------------------------------------------
\title{Functional programming, Seminar No. 6}
\author{Danya Rogozin \\ Institute for Information Transmission Problems, RAS \\ Serokell O\"{U}}

\vspace{\baselineskip}

\date{Higher School of Economics \\ The Department of Computer Science}

\begin{document}

\maketitle

\begin{frame}
  \frametitle{Today}

We will study
\begin{center}
\includegraphics[scale=0.3]{monads.png}
\end{center}
\end{frame}

\section{Monads}

\begin{frame}
  \frametitle{Motivation}

  We are going to extend pure functions \verb"a -> b",
   we would like to extend them to computations with effects:
  \begin{itemize}
    \item A computation with a possible failure: \verb"a -> Maybe b"
    \item A many-valued computation: \verb"a -> [b]"
    \item A computation either succeeds or yields an error: \verb"a -> Either e b"
    \item A computation with logs: \verb"a -> (s, b)"
    \item A computation with reading from an external environment \verb"a -> (e -> b)"
    \item A computation with a mutable state: \verb"a -> (State s) b"
    \item An input/output computation: \verb"a -> IO b"
    \item A generalised version of such a function is a Kleisli function \verb"a -> m b"
  \end{itemize}
\end{frame}

\begin{frame}[fragile]
  \frametitle{Motivation}

If one needs to provide a uniform interface to deal with Kreisli functions, then this interface should satisfy the following two requirements.

  \begin{enumerate}
    \item One needs to have an opportunity inject a pure value into the
    computational context
    \item Kleisli maps should be composable:
    \begin{minted}{haskell}
    (>=>) :: (a -> m b) -> (b -> m c) -> a -> m c
    \end{minted}
    \item In general, we \alert{cannot} extract \verb"a" from \verb"m a"
  \end{enumerate}
\end{frame}

\begin{frame}[fragile]
  \frametitle{The definition of the \verb"Monad" class}

Let us take a look the full definition of the \verb"Monad" class

  \begin{minted}{haskell}
  class Applicative m => Monad m where
    -- | Sequentially compose two actions,
    -- | passing any value produced
    -- | by the first as an argument to the second.
    (>>=) :: m a -> (a -> m b) -> m b

    -- | Sequentially compose two actions,
    -- |discarding any value produced by the first
    (>>) :: m a -> m b -> m b
    m >> k = m >>= \_ -> k

    -- | Inject a value into the monadic type.
    return :: a -> m a
    return = pure
  \end{minted}
\end{frame}

\begin{frame}[fragile]
  \frametitle{The definition of the \verb"Monad" class}
  The \verb"Monad" class has the equivalent definition, the following one:
  \begin{minted}{haskell}
  class Applicative m => Monad m where
    join :: m (m a) -> m a
  \end{minted}

\onslide<2->{
  Moreover, such a definition is closer to the original categorical definition
  of a monad.
  }
\end{frame}

\begin{frame}[fragile]
  \frametitle{The \verb"return" function}

One may convert any pure function into a Kleisli one:
\begin{minted}{haskell}
toKleisli :: Monad m => (a -> b) -> a -> m b
toKleisli f = return . f

cosM :: (Monad m, Floating b) => b -> m b
cosM = toKleisli cos
\end{minted}

It is clear that $\cos \pi = - 1$, but \verb"cosM pi" has the type \verb"(Monad m, Floating b) => m b" and we have several variants:
\begin{minted}{haskell}
> piM :: Maybe Double
Just (-1.0)
> piM :: [Double]
[-1.0]
> piM :: IO (Double)
-1.0
> piM :: Either String Double
Right (-1.0)
\end{minted}
\end{frame}

\begin{frame}[fragile]
  \frametitle{The monadic bind operator}

Take a look at the monadic type signature closely:
  \begin{minted}{haskell}
  (>>=) :: Monad m => m a -> (a -> m b) -> m b
  \end{minted}


In some sense, it is quite close to the reverse application operator.
\vspace{\baselineskip}
  \begin{minted}{haskell}
  (&) :: a -> (a -> b) -> b
  x & f = f x
  \end{minted}

\end{frame}

\begin{frame}[fragile]
  \frametitle{The monadic bind operator}

  Let's have a look at this analogy closely:
  \begin{minted}{haskell}
  fmap :: Functor f => (a -> b) -> f a -> f b
  (<*>) :: Applicative f => f (a -> b) -> f a -> f b
  flip (>>=) :: Monad m => (a -> m b) -> m a -> m b
  \end{minted}

  \vspace{\baselineskip}

  Flipped \verb"fmap", flipped \verb"(<*>)" and \verb"(>>=)":
  \begin{minted}{haskell}
  flip fmap :: Functor f => f a -> (a -> b) -> f b
  flip (<*>) :: Applicative f => f a -> f (a -> b) -> f b
  (>>=) :: Monad m => m a -> (a -> m b) -> m b
  \end{minted}
\end{frame}

\begin{frame}[fragile]
  \frametitle{The very first monad. The \verb"Identity" type}

  Let us define the following new type
  \begin{minted}{haskell}
  {-# LANGUAGE DeriveFunctor #-}

  newtype Identity a = Identity { runIdentity :: a }
    deriving (Show, Functor)

  instance Applicative Identity where
    pure = Identity
    Identity f <*> Identity x = Identity (f x)

  instance Monad Identity where
    Identity x >>= k = k x
  \end{minted}

  This is a trivial monad.
\end{frame}

\begin{frame}[fragile]
  \frametitle{Playing with the \verb"Identity" monad}

Let us consider a quite trivial example of a Kleisli function

\begin{minted}{haskell}
cosId, acosId, sinM
  :: Double -> Identity Double
cosId = Identity . cos
acosId = Identity . acos
sinM = Identity . sin
\end{minted}

\vspace{\baselineskip}

An example:

\begin{minted}{haskell}
> runIdentity $ cosId pi >>= acosId
-1.0
> runIdentity $ cosId pi >>= acosId
3.141592653589793
> runIdentity $ cosId (pi/2) >>= acosId >>= sinM
1.0
\end{minted}

In fact, \verb">>=" works similarly to \verb"(&)" in this example.
\end{frame}

\begin{frame}[fragile]
  \frametitle{Some of useful monadic functions}

Let us take a look at some widely used monadic operations:

\begin{minted}{haskell}
(>=>) :: Monad m => (a -> m b) -> (b -> m c) -> (a -> m c)
f >=> g = \x -> f x >>= g

join :: Monad m => m (m a) -> m a
join x = x >>= id

forever :: Applicative f => f a -> f b
forever a = let a' = a *> a' in a'
\end{minted}

\end{frame}

\begin{frame}[fragile]
  \frametitle{Monadic laws}

  Any monad satisfies the following law:

\begin{enumerate}
\item The left identity law:
  \begin{minted}{haskell}
  return a >>= k  = k a
\end{minted}
\item The right identity law:
\begin{minted}{haskell}
  m >>= return  = m
\end{minted}
\item The monadic bind operation is associative:
\begin{minted}{haskell}
  m >>= (\x -> k x >>= h)  =  (m >>= k) >>= h
\end{minted}
\item There is the strong connection between the notions of monad and monoid, but we drop this connection.
\item Let us illustrate these laws with the \verb"Identity" monad
\end{enumerate}
\end{frame}

\begin{frame}[fragile]
  \frametitle{The identity laws}
  According to the identity laws:
  \begin{minted}{haskell}
  return a >>= k = k a
  m >>= return = m
  \end{minted}
  the \verb"return" function is a sort of a neutral element:

  \begin{minted}{haskell}
  > runIdentity $ cosId (pi / 4)
  0.7071067811865476
  > runIdentity $ return (pi / 4) >>= cosId
  0.7071067811865476
  > runIdentity $ cosId (pi / 4) >>= return
  0.7071067811865476
  \end{minted}
\end{frame}

\begin{frame}[fragile]
  \frametitle{The associativity law}

  The associative law:
  \begin{minted}{haskell}
    m >>= (\x -> k x >>= h)  =  (m >>= k) >>= h
  \end{minted}

  claims that the monadic bind is associative as follows:
  \begin{minted}{haskell}
  > runIdentity $ cosId (pi/2) >>= acosId >>= sinM
  1.0
  > runIdentity $ cosId (pi/2) >>= (\x -> acosId x >>= sinM)
  1.0
  \end{minted}
\end{frame}

  \begin{frame}[fragile]
    \frametitle{The associativity law}
    Let us take a look at these equivalent pipelines:
    \begin{minted}{haskell}
    go = cosId (pi/2) >>=
         acosId       >>=
         sinM
    \end{minted}

    \begin{minted}{haskell}
    go2 = cosId (pi/2) >>= (\x ->
          acosId x     >>= (\y ->
          sinM y       >>= \z ->
          return z))
    \end{minted}
  \end{frame}

  \begin{frame}[fragile]
    \frametitle{Monads and pseudo-imperative programming}
    \begin{minted}{haskell}
    go2 = cosId (pi/2) >>= (\x ->
          acosId x     >>= (\y ->
          sinM y       >>= \z ->
          return z))
    \end{minted}

    \begin{minted}{haskell}
    go2 = cosId (pi/2) >>= (\x ->
          acosId x     >>= (\y ->
          sinM y       >>= \z ->
          return (x, y, z)))
    \end{minted}

    \alert{Wow, we have recently invented imperative programming!}
\end{frame}

\begin{frame}[fragile]
  \frametitle{Monads and pseudoimperative programming}
  We may ignore one of the results:
  \begin{minted}{haskell}
  go2 = let alpha = pi/2 in
        cosId alpha  >>= (\x ->
        acosId x     >>= (\y ->
        sinM y       >>
        return (alpha, x, y)))
  \end{minted}
\end{frame}

\begin{frame}[fragile]
  \frametitle{do-Notation}

  In Haskell, one has a quite useful syntax sugar to write code
  within a monad in the ``imperative'' fashion.

\begin{columns}
  \column{0.5\textwidth}
  \metroset{block=fill}
  \begin{exampleblock}{do-expression}
  \begin{minted}{haskell}
      do { e1; e2 }
  \end{minted}
  \end{exampleblock}

  \begin{exampleblock}{do-expression}
  \begin{minted}{haskell}
      do { p <- e1; e2 }
  \end{minted}
  \end{exampleblock}

  \begin{exampleblock}{do-expression}
  \begin{minted}{haskell}
      do { let v = e1; e2 }
  \end{minted}
  \end{exampleblock}
  \column{0.5\textwidth}
  \metroset{block=fill}
  \begin{exampleblock}{Unsugared version}
  \begin{minted}{haskell}
    e1 >> e2
  \end{minted}
  \end{exampleblock}

  \begin{exampleblock}{Unsugared version}
  \begin{minted}{haskell}
  e1 >>= \p -> e2
  \end{minted}
  \end{exampleblock}

  \begin{exampleblock}{Unsugared version}
  \begin{minted}{haskell}
      let v = e1 in do e2
  \end{minted}
  \end{exampleblock}
  \end{columns}
\end{frame}

\begin{frame}[fragile]
  \frametitle{do-Notation. Example}
  The example above:
  \begin{minted}{haskell}
  go2 = let alpha = pi/2 in
        cosId alpha  >>= (\x ->
        acosId x     >>= (\y ->
        sinM y       >>
        return (alpha, x, y)))
  \end{minted}

One may rewrite this example using do-notation:
\begin{minted}{haskell}
go2 = do
  let alpha = pi/2
  x <- cosId alpha
  y <- acosId x
  z <- sinM y
  return (alpha, x, y)
\end{minted}
\end{frame}

\begin{frame}[fragile]
  \frametitle{do-Notation. Example}
  Let us consider an example of a monadic function:
  \begin{minted}{haskell}
  prodM :: Monad m => (a -> m b) -> (c -> m d)
    -> m (a, c) -> m (b, d)
  prodM f g mp =
    mp >>= \(a,b) -> f a >>= \c -> g b >>= \d ->
    return (c, d)
  \end{minted}

  \vspace{\baselineskip}
  The function above might be implemented as follows with the do-notation sugar
  \begin{minted}{haskell}
  prodM :: Monad m => (a -> m b) -> (c -> m d)
    -> m (a, c) -> m (b, d)
  prodM f g mp = do
    (a, b) <- mp
    c <- f a
    d <- g b
    return (c, d)
  \end{minted}
\end{frame}

\section{The Maybe monad}

\begin{frame}[fragile]
  \frametitle{The Maybe monad}

  The \verb"Maybe" data type is one of the simplest non-trivial monads.
  \begin{minted}{haskell}
  instance Monad Maybe where
    return = Just

    Nothing  >>= _ = Nothing
    (Just x) >>= f = f x

    (Just _) >> a = a
    Nothing >> _ = Nothing
  \end{minted}
\end{frame}

\begin{frame}[fragile]
  \frametitle{The Maybe monad. Example}
  \begin{minted}{haskell}
  type Author = String
  type Book = String
  type Library = [(Author, Book)]

  books :: [Book]
  books = ["Faust", "Alice in Wonderland", "The Idiot"]

  authors :: [Author]
  authors = ["Goethe", "Carroll", "Dostoevsky"]

  library :: Library
  library = zip authors books
  \end{minted}
\end{frame}

\begin{frame}[fragile]
  \frametitle{The Maybe monad. Example}
  \begin{minted}{haskell}
  library' :: Library
  library' = ("Dostoevsky", "Demons") :
    ("Dostoevsky", "White Nights") : library

  getBook :: Author -> Library -> Maybe Book
  getBook author library = lookup author library

  getSecondbook, getLastBook :: Author -> Maybe Book
  getFirstbook author = do
    let lib' = filter (\p -> fst p == author) library'
    book <- getBook author lib'
    return book

  getLastBook author = do
    let lib' = filter (\p -> fst p == author) library'
    book <- getBook author (reverse lib')
    return book
  \end{minted}
\end{frame}

\section{The list monad}

\begin{frame}[fragile]
  \frametitle{The list instance}

The \verb"Monad" instance is the following one:

  \begin{minted}{haskell}
  instance Monad [] where
    return x = [x]
    xs >>= k = concat (map k xs)
  \end{minted}
\end{frame}

\begin{frame}[fragile]
  \frametitle{List compeherension once more}

The following functions are equivalent:
  \begin{minted}{haskell}
cartesianProduct :: [a] -> [b] -> [(a, b)]
cartesianProduct xs ys =
  xs >>= \x -> ys >>= \y -> return (x, y)

cartesianProduct' :: [a] -> [b] -> [(a, b)]
cartesianProduct' xs ys = do
  x <- xs
  y <- ys
  return (x, y)

cartesianProduct'' :: [a] -> [b] -> [(a, b)]
cartesianProduct'' xs ys = [(x, y) | x <- xs, y <- ys]
  \end{minted}
\end{frame}

\end{document}
