\documentclass[10pt]{beamer}
\usepackage[cache=false]{minted}
\usepackage[utf8x]{inputenc}
\usepackage{hyperref}
\usepackage{fontawesome}
\usepackage{graphicx}
\usepackage[english,ngerman]{babel}
\usepackage{bussproofs}

% ------------------------------------------------------------------------------
% Use the beautiful metropolis beamer template
% ------------------------------------------------------------------------------
\usepackage[T1]{fontenc}
\usepackage{fontawesome}
\usepackage{FiraSans}
\newtheorem{defin}{Definition}
\newtheorem{theor}{Theorem}
\newtheorem{prop}{Proposition}
\mode<presentation>
{
  \usetheme[progressbar=foot,numbering=fraction,background=light]{metropolis}
  \usecolortheme{default} % or try albatross, beaver, crane, ...
  \usefonttheme{default}  % or try serif, structurebold, ...
  \setbeamertemplate{navigation symbols}{}
  \setbeamertemplate{caption}[numbered]
  %\setbeamertemplate{frame footer}{My custom footer}
}

% ------------------------------------------------------------------------------
% beamer doesn't have texttt defined, but I usually want it anyway
% ------------------------------------------------------------------------------
\let\textttorig\texttt
\renewcommand<>{\texttt}[1]{%
  \only#2{\textttorig{#1}}%
}

% ------------------------------------------------------------------------------
% minted
% ------------------------------------------------------------------------------


% ------------------------------------------------------------------------------
% tcolorbox / tcblisting
% ------------------------------------------------------------------------------
\usepackage{xcolor}
\definecolor{codecolor}{HTML}{FFC300}

\usepackage{tcolorbox}
\tcbuselibrary{most,listingsutf8,minted}

\tcbset{tcbox width=auto,left=1mm,top=1mm,bottom=1mm,
right=1mm,boxsep=1mm,middle=1pt}

\newtcblisting{myr}[1]{colback=codecolor!5,colframe=codecolor!80!black,listing only,
minted options={numbers=left, style=tcblatex,fontsize=\tiny,breaklines,autogobble,linenos,numbersep=3mm},
left=5mm,enhanced,
title=#1, fonttitle=\bfseries,
listing engine=minted,minted language=r}


% ------------------------------------------------------------------------------
% Listings
% ------------------------------------------------------------------------------
\definecolor{mygreen}{HTML}{37980D}
\definecolor{myblue}{HTML}{0D089F}
\definecolor{myred}{HTML}{98290D}

\usepackage{listings}

% the following is optional to configure custom highlighting
\lstdefinelanguage{XML}
{
  morestring=[b]",
  morecomment=[s]{<!--}{-->},
  morestring=[s]{>}{<},
  morekeywords={ref,xmlns,version,type,canonicalRef,metr,real,target}% list your attributes here
}

\lstdefinestyle{myxml}{
language=XML,
showspaces=false,
showtabs=false,
basicstyle=\ttfamily,
columns=fullflexible,
breaklines=true,
showstringspaces=false,
breakatwhitespace=true,
escapeinside={(*@}{@*)},
basicstyle=\color{mygreen}\ttfamily,%\footnotesize,
stringstyle=\color{myred},
commentstyle=\color{myblue}\upshape,
keywordstyle=\color{myblue}\bfseries,
}


% ------------------------------------------------------------------------------
% The Document
% ------------------------------------------------------------------------------
\title{Functional programming, Seminar No. 8}
\author{Danya Rogozin \\ Institute for Information Transmission Problems, RAS \\ Serokell O\"{U}}

\vspace{\baselineskip}

\date{Higher School of Economics \\ The Department of Computer Science}

\begin{document}

\maketitle

\begin{frame}[fragile]
  \frametitle{Type of parser}
  What is a type of function which parses an integer?
  \begin{minted}{haskell}
    parseInteger :: String -> Bool
  \end{minted}
  Not quite, we would like to get an \verb"Integer"

  The second variant. But such a parser can fail.
  \begin{minted}{haskell}
    parseInteger :: String -> Integer
  \end{minted}

  \begin{minted}{haskell}
  parseInteger :: String -> Maybe Integer
  \end{minted}
  Now. How do you parse two integers separated by space? Or by any number of spaces? Or comma-separated list of integers? Or the same list inside square brackets [] with any number of spaces between elements?

  The solution is the following:
  \begin{minted}{haskell}
    parseInteger :: String -> Maybe (Integer, String)
  \end{minted}
\end{frame}

\begin{frame}[fragile]
  \frametitle{Type of parser}
  But instead of this:
  \begin{minted}{haskell}
  parseInteger :: String -> Maybe (Integer, String)
  \end{minted}
  We introduce \verb"Parser" with the \verb"newtype" wrapper in order to have useful instances:
  \begin{minted}{haskell}
  newtype Parser a =
    Parser { runP :: String -> Maybe (a, String) }
  \end{minted}
\end{frame}

\begin{frame}[fragile]
  \frametitle{Some usage examples}
The parser combinator type:
\begin{minted}{haskell}
newtype Parser a = Parser { runP :: String -> Maybe (a, String) }
\end{minted}

Parsing functions
\begin{minted}{haskell}
parseInteger :: Parser Integer
  -- String -> Maybe (Integer, String)
\end{minted}

How to use it and which behaviour we want?
\begin{minted}{haskell}
runP :: Parser a -> String -> Maybe (a, String)
ghci> runP parseInteger "5"
Just (5, "") :: Maybe (Integer, String)

ghci> runP parseInteger "42x7"
Just (42, "x7") :: Maybe (Integer, String)

ghci> runP parseInteger "abc"
Nothing :: Maybe (Integer, String)
\end{minted}
This behaviour allows us to combine parsers relatively easy!
\end{frame}

\begin{frame}[fragile]
  \frametitle{Now, idea and how to do it}

\begin{itemize}
  \item The key idea of parser combinators: implement manually very simple parsers, implement combinators for combining parser and then implement more complex parsers by combining simpler ones.
  \item Haskell provides combinators through standard type classes. And it's convenient to use them. What we need is just:
  \begin{minted}{haskell}
  instance Functor     Parser
    -- replace parser value
  instance Applicative Parser
    -- run parsers sequentially one after another
  instance Monad       Parser
    -- same as above but with monadic capabilities
  instance Alternative Parser
    -- allows one to choose parser
  \end{minted}
\end{itemize}
\end{frame}

\begin{frame}[fragile]
  \frametitle{Primitive parsers. \verb"ok"}

  \begin{minted}{haskell}
  newtype Parser a =
    Parser { runP :: String -> Maybe (a, String) }

  -- always succeeds without consuming any input
  ok :: Parser ()
  ok = error "this is your home"
  \end{minted}
\end{frame}

\begin{frame}[fragile]
  \frametitle{Primitive parsers. \verb"isnot"}

\begin{minted}{haskell}
newtype Parser a =
  Parser { runP :: String -> Maybe (a, String) }

-- fails w/o consuming any input if given parser succeeds,
-- and succeeds if given parser fails
isnot :: Parser a -> Parser ()
isnot parser =
  Parser $ \s -> case runP parser s of
                   Just _  -> Nothing
                   Nothing -> Just ((), s)
  \end{minted}
\end{frame}

\begin{frame}[fragile]
  \frametitle{Primitive parsers. \verb"eof"}

\begin{minted}{haskell}
newtype Parser a =
  Parser { runP :: String -> Maybe (a, String) }

-- succeeds only at the end of input stream
eof :: Parser ()
eof = Parser $ \s -> case s of
    [] -> Just ((), "")
    _  -> Nothing
\end{minted}
\end{frame}

\begin{frame}[fragile]
  \frametitle{Primitive parsers. \verb"ok"}

\begin{minted}{haskell}
newtype Parser a =
  Parser { runP :: String -> Maybe (a, String) }

-- consumes only single character and
-- returns it if predicate is true
satisfy :: (Char -> Bool) -> Parser Char
satisfy p = Parser $ \s -> case s of
  []     -> Nothing
  (x:xs) -> if p x then Just (x, xs) else Nothing
\end{minted}
\end{frame}

\begin{frame}[fragile]
  \frametitle{Combining}
\begin{minted}{haskell}
-- always fails without consuming any input
notok :: Parser ()
notok = isnot ok

-- consumes given character and returns it
char :: Char -> Parser Char
char c = satisfy (== c)

-- consumes any character or any digit only
anyChar, digit :: Parser Char
anyChar = satisfy (const True)
digit   = satisfy isDigit
\end{minted}
\end{frame}

\begin{frame}[fragile]
  \frametitle{Combining}
  \begin{minted}{haskell}
  ghci> runP eof ""
  Just ((),"")
  ghci> runP eof "aba"
  Nothing
  ghci> runP (char 'a') "aba"
  Just ('a',"ba")
  ghci> runP (char 'x') "aba"
  Nothing
  \end{minted}
\end{frame}

\begin{frame}[fragile]
  \frametitle{Instances}
  The \verb"Functor" instance
  \begin{minted}{haskell}
  newtype Parser a =
    Parser { runP :: String -> Maybe (a, String) }

  instance Functor Parser where
    fmap :: (a -> b) -> Parser a -> Parser b
    fmap f (Parser parser) = Parser (fmap (first f) . parser)

  first :: (a -> b) -> (a, c) -> (b, c)
  first f (a, c) = (f a, c)
  \end{minted}
\end{frame}

\begin{frame}[fragile]
  \frametitle{Instances}
  The \verb"Applicative" instance
  \begin{minted}{haskell}
  newtype Parser a =
    Parser { runP :: String -> Maybe (a, String) }

  instance Applicative Parser where
    pure :: a -> Parser a
    pure a = Parser $ \s -> Just (a, s)

    (<*>) :: Parser (a -> b) -> Parser a -> Parser b
    Parser pf <*> Parser pa = Parser $ \s -> case pf s of
        Nothing     -> Nothing
        Just (f, t) -> case pa t of
            Nothing     -> Nothing
            Just (a, r) -> Just (f a, r)
  \end{minted}
\end{frame}

\begin{frame}[fragile]
  \frametitle{Instances}
  The \verb"Monad" instance
  \begin{minted}{haskell}
  newtype Parser a =
    Parser { runP :: String -> Maybe (a, String) }

  instance Monad Parser where
    (>>=) :: Parser a -> (a -> Parser b) -> Parser b
    Parser pf >>= k = Parser $ \s -> do
        (x, rest1) <- runP p1 s
        runP (k x) rest1

  instance Alternative Parser where
    empty :: Parser a
    -- always fails
    (<|>) :: Parser a -> Parser a -> Parser a
    -- run first, if fails — run second
  \end{minted}
\end{frame}

\begin{frame}[fragile]
  \frametitle{Simple parser combinators core}
  \begin{minted}{haskell}
  -- type
  newtype Parser a =
    Parser { runP :: String -> Maybe (a, String) }

  -- parsers
  eof, ok :: Parser ()
  satisfy :: (Char -> Bool) -> Parser Char
  empty   :: Parser a

  -- combinators
  pure  :: a -> Parser a
  (<|>) :: Parser a -> Parser a -> Parser a         -- orElse
  (>>=) :: Parser a -> (a -> Parser b) -> Parser b  -- andThen
  \end{minted}
\end{frame}

\begin{frame}[fragile]
  \frametitle{Simple parser combinators core}
  \begin{minted}{haskell}
  -- combinators
  -- * Functor
  fmap  :: (a -> b) -> Parser a -> Parser b
  (<$)  :: a -> Parser b -> Parser a

  -- * Applicative
  (<*>) :: Parser (a -> b) -> Parser a -> Parser b
  (<*)  :: Parser a -> Parser b -> Parser a
    -- run both in sequence, result of first

  -- * Alternative
  many  :: Parser a -> Parser [a]
  some  :: Parser a -> Parser [a]

  -- * Monadic
  (>>=) :: Parser a -> (a -> Parser b) -> Parser b  -- andThen
  \end{minted}
\end{frame}

\begin{frame}[fragile]
  \frametitle{Examples}
  \begin{minted}{haskell}
  ghci> runP (ord <$> char 'A') "A"
  Just (65,"")

  ghci> runP ((\x y -> [x, y]) <$> char 'a' <*> char 'b') "abc"
  Just ("ab","c")
  ghci> runP ((\x y -> [x, y]) <$> char 'a' <*> char 'b') "xxx"
  Nothing

  ghci> runP (char 'a' <* eof) "a"
  Just ('a',"")
  ghci> runP (char 'a' <* eof) "ab"
  Nothing
  \end{minted}
\end{frame}

\begin{frame}[fragile]
  \frametitle{Examples}
  \begin{minted}{haskell}
  ghci> runP (many $ char 'a') "aaabcd"
  Just ("aaa","bcd")
  ghci> runP (many $ char 'a') "xxx"
  Just ("","xxx")
  ghci> runP (some $ char 'a') "xxx"
  Nothing

  ghci> runP (many $ char 'a') "aaabcd"
  Just ("aaa","bcd")
  ghci> runP (many $ char 'a') "xxx"
  Just ("","xxx")
  ghci> runP (some $ char 'a') "xxx"
  Nothing
  \end{minted}
\end{frame}

\begin{frame}[fragile]
  \frametitle{Examples}
  \begin{minted}{haskell}
  ghci> runP (many $ char 'a') "aaabcd"
  Just ("aaa","bcd")
  ghci> runP (many $ char 'a') "xxx"
  Just ("","xxx")
  ghci> runP (some $ char 'a') "xxx"
  Nothing

  ghci> runP (many $ char 'a') "aaabcd"
  Just ("aaa","bcd")
  ghci> runP (many $ char 'a') "xxx"
  Just ("","xxx")
  ghci> runP (some $ char 'a') "xxx"
  Nothing
  \end{minted}
\end{frame}

\begin{frame}[fragile]
  \frametitle{Examples}
  \begin{minted}{haskell}
string :: String -> Parser String
  -- like 'char' but for string
oneOf  :: [String] -> Parser String
  -- parse first matched string from list

data Answer = Yes | No

yesP :: Parser Answer
yesP = Yes <$ oneOf ["y", "Y", "yes", "Yes", "ys"]

noP :: Parser Answer
noP = No <$ oneOf ["n", "N", "no", "No"]

answerP :: Parser Answer
answerP = yesP <|> noP
  \end{minted}
\end{frame}


\end{document}
