\documentclass[10pt]{beamer}
\usepackage[cache=false]{minted}
\usepackage[utf8x]{inputenc}
\usepackage{hyperref}
\usepackage{fontawesome}
\usepackage{graphicx}
\usepackage[english,ngerman]{babel}
\usepackage{bussproofs}

% ------------------------------------------------------------------------------
% Use the beautiful metropolis beamer template
% ------------------------------------------------------------------------------
\usepackage[T1]{fontenc}
\usepackage{fontawesome}
\usepackage{FiraSans}
\newtheorem{defin}{Definition}
\newtheorem{theor}{Theorem}
\newtheorem{prop}{Proposition}
\mode<presentation>
{
  \usetheme[progressbar=foot,numbering=fraction,background=light]{metropolis}
  \usecolortheme{default} % or try albatross, beaver, crane, ...
  \usefonttheme{default}  % or try serif, structurebold, ...
  \setbeamertemplate{navigation symbols}{}
  \setbeamertemplate{caption}[numbered]
  %\setbeamertemplate{frame footer}{My custom footer}
}

% ------------------------------------------------------------------------------
% beamer doesn't have texttt defined, but I usually want it anyway
% ------------------------------------------------------------------------------
\let\textttorig\texttt
\renewcommand<>{\texttt}[1]{%
  \only#2{\textttorig{#1}}%
}

% ------------------------------------------------------------------------------
% minted
% ------------------------------------------------------------------------------


% ------------------------------------------------------------------------------
% tcolorbox / tcblisting
% ------------------------------------------------------------------------------
\usepackage{xcolor}
\definecolor{codecolor}{HTML}{FFC300}

\usepackage{tcolorbox}
\tcbuselibrary{most,listingsutf8,minted}

\tcbset{tcbox width=auto,left=1mm,top=1mm,bottom=1mm,
right=1mm,boxsep=1mm,middle=1pt}

\newtcblisting{myr}[1]{colback=codecolor!5,colframe=codecolor!80!black,listing only,
minted options={numbers=left, style=tcblatex,fontsize=\tiny,breaklines,autogobble,linenos,numbersep=3mm},
left=5mm,enhanced,
title=#1, fonttitle=\bfseries,
listing engine=minted,minted language=r}


% ------------------------------------------------------------------------------
% Listings
% ------------------------------------------------------------------------------
\definecolor{mygreen}{HTML}{37980D}
\definecolor{myblue}{HTML}{0D089F}
\definecolor{myred}{HTML}{98290D}

\usepackage{listings}

% the following is optional to configure custom highlighting
\lstdefinelanguage{XML}
{
  morestring=[b]",
  morecomment=[s]{<!--}{-->},
  morestring=[s]{>}{<},
  morekeywords={ref,xmlns,version,type,canonicalRef,metr,real,target}% list your attributes here
}

\lstdefinestyle{myxml}{
language=XML,
showspaces=false,
showtabs=false,
basicstyle=\ttfamily,
columns=fullflexible,
breaklines=true,
showstringspaces=false,
breakatwhitespace=true,
escapeinside={(*@}{@*)},
basicstyle=\color{mygreen}\ttfamily,%\footnotesize,
stringstyle=\color{myred},
commentstyle=\color{myblue}\upshape,
keywordstyle=\color{myblue}\bfseries,
}

\title{Functional programming, Seminar No. 3}
\author{Danya Rogozin \\ Institute for Information Transmission Problems, RAS \\ Serokell O\"{U}}
\date{Higher School of Economics \\ The Faculty of Computer Science}
\begin{document}

\maketitle

\begin{frame}
  \frametitle{Intro}

  Today we study
  \begin{center}
  \includegraphics[scale=0.33]{Pics/doge.png}
  \end{center}
\end{frame}

\begin{frame}[fragile]
  \frametitle{Motivation}
  Let us recall the example of a higher order function from the previous seminar:

  \begin{minted}{haskell}
  changeTwiceBy :: (Int -> Int) -> Int -> Int
  changeTwiceBy operation value = operation (operation value)
  \end{minted}

  One may implement the function for Boolean values and strings that are the 'same' as the function above:

  \begin{minted}{haskell}
  changeTwiceBy :: (Bool -> Bool) -> Bool -> Bool
  changeTwiceByBool operation value =
    operation (operation value)

  changeTwiceBy :: (String -> String) -> String -> String
  changeTwiceBy operation value =
    operation (operation value)
  \end{minted}

  \onslide<2->{
  Too much boilerplate.
  }
\end{frame}

\begin{frame}[fragile]
  \frametitle{Parametric polymorphism}

The key idea of parametric polymorphism that the same function might be called on distinct data types. Here are the first polymorphic examples:
  \begin{minted}{haskell}
  id :: a -> a
  id x = x

  const :: a -> b -> a
  const a b = a

  fst :: (a, b) -> a
  fst (a, b) = a

  snd :: (a, b) -> b
  snd = "guess what"

  swap :: (a, b) -> (b, a)
  swap (a, b) = (b, a)
  \end{minted}
\end{frame}

\begin{frame}
  \frametitle{Meme time}

  \begin{center}
  \includegraphics[scale=0.3]{Pics/Ryan.png}
  \end{center}
\end{frame}

\begin{frame}[fragile]
  \frametitle{Example}
\metroset{block=fill}
\begin{exampleblock}{GHCi session}
\begin{minted}{haskell}
  > id 6
  6
  > id 6.0
  6.0
  > const True 6
  True
  > const 6 True
  6
  > fst ('5', 5)
  '5'
  > fst (5, '5')
  5
\end{minted}
\end{exampleblock}
\end{frame}

\begin{frame}[fragile]
  \frametitle{A brief clarification}

\begin{itemize}
  \item In such signatures as $a \to b \to a$, $a, b$ are type variables that range over arbitrary data types. In fact, $a$, $b$ are bounded by universal quantifier hidden under the carpet.
  \item The functions from the previous slide have the following signatures:
  \begin{minted}{haskell}
  id :: forall a. a -> a
  id x = x

  const :: forall a b. a -> b -> a
  const a b = a

  fst :: forall a b. (a, b) -> a
  fst (a, b) = a

  swap :: forall a b. (a, b) -> (b, a)
  swap (a, b) = (b, a)
  \end{minted}
\end{itemize}
\end{frame}

\begin{frame}[fragile]
  \frametitle{Higher order functions and parametric polymorpism}
  \metroset{block=fill}
  \begin{exampleblock}{More examples}
  \begin{minted}{haskell}
  infixr 9 .
  (.) :: (b -> c) -> (a -> b) -> a -> c
  f . g = \x -> f (g x)

  flip :: (a -> b -> c) -> b -> a -> c
  flip f b a = f a b

  fix :: (a -> a) -> a
  fix f = f (fix f)

  curry :: ((a, b) -> c) -> a -> b -> c
  curry f x y = f (x, y)

  uncurry :: (a -> b -> c) -> ((a, b) -> c)
  uncurry f p = f (fst p) (snd p)
  \end{minted}
\end{exampleblock}
\end{frame}

\begin{frame}[fragile]
    \frametitle{Examples with composition}
\metroset{block=fill}
\begin{exampleblock}{More examples}
\begin{minted}{haskell}
  incNegate :: Int -> Int
  incNegate x = negate (x + 1)

  incNegate x = negate $ x + 1

  incNegate x = (negate . (+1)) x

  incNegate x = negate . (+1) $ x

  incNegate   = negate . (+1)
\end{minted}
\end{exampleblock}
\end{frame}

\begin{frame}[fragile]
  \frametitle{\verb"curry" and \verb"uncurry"}

\metroset{block=fill}
  \begin{exampleblock}{More examples}
  \begin{minted}{haskell}
    > uncurry (*) (3,4)
    12
    > curry fst 3 4
    3
    > curry id 3 4
    (3,4)
    > uncurry const (3,4)
    3
    > uncurry (flip const) (3,4)
    4
  \end{minted}
  \end{exampleblock}
\end{frame}

\begin{frame}[fragile]
  \frametitle{Examples with \verb"flip"}
  \metroset{block=fill}
  \begin{exampleblock}{More examples}
  \begin{minted}{haskell}
  show2 :: Int -> Int -> String
  show2 x y = show x ++ " and " ++ show y

  showSnd, showFst, showFst' :: Int -> String
  showSnd  = show2 1
  showFst  = flip show2 2
  showFst' = (`show2` 2)
  \end{minted}
\end{exampleblock}

  \metroset{block=fill}
  \begin{exampleblock}{GHCi session}
    \begin{minted}{haskell}
    > showSnd 10
    "1 and 10"
    > showFst 10
    "10 and 2"
    > showFst' 42
    "42 and 2"
    \end{minted}
  \end{exampleblock}
\end{frame}

\begin{frame}[fragile]
  \frametitle{Bye-bye boilerplate!}

  All those functions such as the following one
  \metroset{block=fill}
  \begin{exampleblock}{\verb"changeTwiceBy" with \verb"Int"}
  \begin{minted}{haskell}
  changeTwiceBy :: (Int -> Int) -> Int -> Int
  changeTwiceBy operation value =
    operation (operation value)
  \end{minted}
\end{exampleblock}
  might be replaced by one of the following ones:
  \metroset{block=fill}
  \begin{exampleblock}{\verb"applyTwice"}
  \begin{minted}{haskell}
  applyTwice :: (a -> a) -> a -> a
  applyTwice f a = f (f a)
  applyTwice f a = f . f $ a
  applyTwice f = f . f
  \end{minted}
  \end{exampleblock}
\end{frame}

\begin{frame}[fragile]
  \frametitle{HOF, polymorpism, and lists}
  \metroset{block=fill}
  \begin{exampleblock}{Examples}
  \begin{minted}{haskell}
  map     :: (a -> b)      -> [a] -> [b]

  filter  :: (a -> Bool)   -> [a] -> [a]

  zipWith :: (a -> b -> c) -> [a] -> [b] -> [c]

  length :: [a] -> Int
  \end{minted}
  \end{exampleblock}

\vspace{\baselineskip}

  We discuss their implementations closely on the next seminar. Here we just discuss them briefly.
\end{frame}

\begin{frame}[fragile]
    \frametitle{The composition examples + list functions}
    \metroset{block=fill}
    \begin{exampleblock}{More examples}
    \begin{minted}{haskell}
    foo, bar :: [Int] -> Int
    foo patak =
      length $ filter odd $
      map (div 2) $ filter even $ map (div 7) patak
    bar =
      length . filter odd .
      map (div 2) . filter even . map (div 7)

    zip :: [a] -> [b] -> [(a, b)]
    zip = zipWith (,)
  \end{minted}
  \end{exampleblock}
\end{frame}

\begin{frame}[fragile]
    \frametitle{The composition examples + list functions}
    \metroset{block=fill}
    \begin{exampleblock}{More examples}
    \begin{minted}{haskell}
    stringsTransform :: [String] -> [String]
    stringsTransform l =
      map (\s -> map toUpper s)
      (filter (\s -> length s == 5) l)

    stringsTransform l =
      map (\s -> map toUpper s) $
      filter (\s -> length s == 5) l

    stringsTransform l =
      map (map toUpper) $ filter ((== 5) . length) l

    stringsTransform =
      map (map toUpper) . filter ((== 5) . length)
    \end{minted}
    \end{exampleblock}
\end{frame}

\begin{frame}[fragile]
  \frametitle{Bounded polymorphism and type classes}
  The idea of bounded (ad hoc) polymorphism is that one has a general interface with instances for each concrete data type.
  \metroset{block=fill}
  \begin{exampleblock}{More examples}
  \begin{minted}{haskell}
  > :t 9
  9 :: Num p => p
  > 9 :: Int
  9
  > 9 :: Double
  9.0
  > 9 :: Rational
  9 % 1
  > 9 :: Char
  <interactive>:6:1: error:
  * No instance for (Num Char) arising from the literal '9'
  \end{minted}
\end{exampleblock}
\end{frame}

\begin{frame}[fragile]
  \frametitle{Type classes. Motivation}

\begin{itemize}
  \item Let us take a look a the following function

  \begin{minted}{haskell}
  elem :: a -> [a] -> Bool
  elem _ [] = False
  elem x (y:ys) = x == y || elem x ys
  \end{minted}

  \item \verb"a" is an arbitrary type for which equality is defined as usual.
\end{itemize}
\end{frame}

\begin{frame}[fragile]
  \frametitle{Type classes. Motivation}
  Type variables in polymorphic function are bounded with universal quantifier. In ad hoc polymorphism, type variables are also bounded with $\forall$ but with the additional condition. This kind of quantification is called \emph{bounded}.

  \begin{minted}{haskell}
  elem :: forall a. Eq a => a -> [a] -> Bool
  elem _ [] = False
  elem x (y:ys) = x == y || elem x ys
  \end{minted}
\end{frame}

\begin{frame}[fragile]
  \frametitle{The notion of a type class}

\begin{itemize}
  \item \emph{A type class} is a collection of functions with type signatures with a common type parameter. An example given:
  \begin{minted}{haskell}
  class Eq a where
    (==) :: a -> a -> Bool
    (/=) :: a -> a -> Bool
  \end{minted}
  \item A type class name introduce a constraint called \emph{context}:
  \begin{minted}{haskell}
  elem :: Eq a => a -> [a] -> Bool
  \end{minted}
  \item The definition above without a context is not well-defineds:
  \begin{minted}{haskell}
<interactive>:4:19: error:
* No instance for (Eq a) arising from a use of '=='
Possible fix: add (Eq a) to the context of
the type signature for: elem :: forall a. a -> [a] -> Bool
  \end{minted}
\end{itemize}
\end{frame}

\begin{frame}[fragile]
  \frametitle{Instance declarations}

  A given data type \verb"a" has the \emph{instance} of a type class if every function of that class is implemented for \verb"a". An example:

  \metroset{block=fill}
  \begin{exampleblock}{Example}
  \begin{minted}{haskell}
  instance Eq Bool where
    True == True   = True
    False == False = True
    _ == _         = False

    x /= y        = neg (x == y)
  \end{minted}
  \end{exampleblock}
\end{frame}

\begin{frame}[fragile]
  \frametitle{Polymorphism + instance declarations}

\begin{itemize}
  \item A type parameter in an instance declaration might be polymorphic as well:

  \metroset{block=fill}
  \begin{exampleblock}{Example}
  \begin{minted}{haskell}
  instance Eq a => Eq [a] where
    []       == []       = True
    (x : xs) == (y : ys) = x == y && xs == ys
    _        == _        = False
  \end{minted}
\end{exampleblock}
  \item Without the context \verb"Eq a =>", this definition yields type error since we don't know how to perform
  equality comparison of element from \verb"a"
\end{itemize}
\end{frame}

\begin{frame}[fragile]
  \frametitle{Some the \verb"Eq" instances}

\begin{itemize}
\item The \verb"Eq" type class has the following instances (some of them)
  \metroset{block=fill}
  \begin{exampleblock}{\verb"Eq" instances}
  \begin{minted}{haskell}
  instance Eq a => Eq [a]
  instance Eq Ordering
  instance Eq Int
  instance Eq Float
  instance Eq Double
  instance Eq Char
  instance Eq Bool
\end{minted}
  \end{exampleblock}
\item See the standard library source code to have a look at the implementation.
\end{itemize}
\end{frame}

\begin{frame}[fragile]
  \frametitle{The \verb"Show" type class}

\begin{itemize}
  \item The \verb"Show" type class allows one to represent a value as a string:
  \metroset{block=fill}
  \begin{exampleblock}{Some \verb"Eq" instances}
  \begin{minted}{haskell}
  class Show a where
    showsPrec :: Int -> a -> ShowS
    show :: a -> String
    showList :: [a] -> ShowS
    {-# MINIMAL showsPrec | show #-}
  \end{minted}
  \end{exampleblock}
  \item One needs to have a \verb"Show" instance to show a value of a given type.
\end{itemize}
\end{frame}

\begin{frame}[fragile]
  \frametitle{Some of the \verb"Show" instances}

Here are some of the \verb"Show" instances:
\metroset{block=fill}
\begin{exampleblock}{Some \verb"Show" instances}
\begin{minted}{haskell}
  instance Show Integer
  instance Show Int
  instance Show Char
  instance Show Bool
  instance (Show a, Show b) => Show (a, b)
\end{minted}
\end{exampleblock}
\end{frame}

\begin{frame}[fragile]
  \frametitle{Ordering. Motivation}
  \begin{itemize}
  \item Let us take a look at the following quicksort function:

\metroset{block=fill}
\begin{exampleblock}{Some \verb"Show" instances}
\begin{minted}{haskell}
  quicksort :: [a] -> [a]
  quicksort [] = []
  quicksort (x:xs) =
    quicksort small ++ (x : quicksort large)
    where
      small = [y | y <- xs, y <= x]
      large = [y | y <- xs, y > x]
  \end{minted}
  \end{exampleblock}
  \onslide<2->{
  \item Here we have the same situation. The definition of \verb"quicksort" as above is wrong. There exist types elements of which are incomparable, complex numbers, e.g.
  }
\end{itemize}
\end{frame}

\begin{frame}[fragile]
  \frametitle{The \verb"Ord" type class}

The full definition of \verb"Ord" is the following one:

\metroset{block=fill}
\begin{exampleblock}{\verb"Ord"}
\begin{minted}{haskell}
  class Eq a => Ord a where
    compare :: a -> a -> Ordering
    (<), (<=), (>), (>=) :: a -> a -> Bool
    max, min :: a -> a -> a

    compare x y = if x == y then EQ
              else if x <= y then LT
              else GT

    x <= y = case compare x y of
      { GT -> False; _ -> True }

    {-# MINIMAL compare | (<=) #-}
    \end{minted}
    \end{exampleblock}
\end{frame}

\begin{frame}[fragile]
  \frametitle{\verb"Ord" instances}

  \metroset{block=fill}
  \begin{exampleblock}{\verb"Ord" instances}
  \begin{minted}{haskell}
  instance Ord Int
  instance Ord Float
  instance Ord Double
  instance Ord Char
  instance Ord Bool
\end{minted}
\end{exampleblock}
\end{frame}

\begin{frame}[fragile]
  \frametitle{The \verb"Num" type class}

\verb"Num" is a type class with the general interface of usual arithmetic operations.
\metroset{block=fill}
\begin{exampleblock}{\verb"Num"}
\begin{minted}{haskell}
class Num a where
  (+), (-), (*) :: a -> a -> a
  negate, abs, signum :: a -> a
  fromInteger :: Integer -> a
  {-# MINIMAL (+), (*), abs, signum,
      fromInteger, (negate | (-)) #-}
  \end{minted}
\end{exampleblock}
\end{frame}

\begin{frame}[fragile]
  \frametitle{\verb"Num" instances}

\metroset{block=fill}
\begin{exampleblock}{Some \verb"Num" instances}
\begin{minted}{haskell}
  instance Num Integer
  instance Num Int
  instance Num Float
  instance Num Double
  \end{minted}
  \end{exampleblock}
\end{frame}

\begin{frame}[fragile]
  \frametitle{The \verb"Enum" type class}

The \verb"Enum" is type class for types for which we can define an explicit enumeration.
\metroset{block=fill}
\begin{exampleblock}{Some \verb"Enum" instances}
\begin{minted}{haskell}
  class Enum a where
    succ, pred :: a -> a
    toEnum     :: Int -> a
    fromEnum   :: a -> Int

    enumFrom       :: a -> [a]            -- [n..]
    enumFromThen   :: a -> a -> [a]       -- [n,m..]
    enumFromTo     :: a -> a -> [a]       -- [n..m]
    enumFromThenTo :: a -> a -> a -> [a]  -- [n,m..p]
    {-# MINIMAL toEnum, fromEnum #-}
  \end{minted}
  \end{exampleblock}
\end{frame}

\begin{frame}[fragile]
  \frametitle{\verb"Enum" instances}

  \metroset{block=fill}
  \begin{exampleblock}{Some \verb"Num" instances}
  \begin{minted}{haskell}
  instance Enum Integer
  instance Enum Int
  instance Enum Char
  instance Enum Bool
  instance Enum Float
  instance Enum Double
  \end{minted}
  \end{exampleblock}
\end{frame}

\begin{frame}[fragile]
  \frametitle{The \verb"Fractional" type class}

\begin{itemize}
  \item The \verb"Fractional" type class is a general interface for division
  \metroset{block=fill}
  \begin{exampleblock}{Some \verb"Num" instances}
  \begin{minted}{haskell}
  class Num a => Fractional a where
    (/) :: a -> a -> a
    recip :: a -> a
    fromRational :: Rational -> a
    {-# MINIMAL fromRational, (recip | (/)) #-}
  \end{minted}
\end{exampleblock}
  \item We require the \verb"Num a" restriction.
  \item The \verb"Fractional" instances are \verb"Float" and \verb"Double".
\end{itemize}
\end{frame}

\begin{frame}
  \frametitle{Summary}

\onslide<1->{
  On this seminar, we
  \begin{itemize}
    \item took a look at parametric polymorphism to see how to avoid boilerplate
    \item discussed type classes and ad hoc polymorphism
    \item studied such basic type classes as \verb"Eq", \verb"Show", etc
  \end{itemize}
}

\onslide<2->{
On the next seminar, we
  \begin{itemize}
    \item delve into the variety of Haskell data types: algebraic data types, newtypes,
    type synonyms, etc
    \item feel the power of pattern matching
    \item discuss folds
    \item see how to enforce evaluation in Haskell
  \end{itemize}
}
\end{frame}


\end{document}
