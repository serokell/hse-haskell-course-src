\documentclass[10pt]{beamer}
\usepackage[cache=false]{minted}
\usepackage[utf8x]{inputenc}
\usepackage{hyperref}
\usepackage{fontawesome}
\usepackage{graphicx}
\usepackage[english,ngerman]{babel}
\usepackage{bussproofs}

% ------------------------------------------------------------------------------
% Use the beautiful metropolis beamer template
% ------------------------------------------------------------------------------
\usepackage[T1]{fontenc}
\usepackage{fontawesome}
\usepackage{FiraSans}
\newtheorem{defin}{Definition}
\newtheorem{theor}{Theorem}
\newtheorem{prop}{Proposition}
\mode<presentation>
{
  \usetheme[progressbar=foot,numbering=fraction,background=light]{metropolis}
  \usecolortheme{default} % or try albatross, beaver, crane, ...
  \usefonttheme{default}  % or try serif, structurebold, ...
  \setbeamertemplate{navigation symbols}{}
  \setbeamertemplate{caption}[numbered]
  %\setbeamertemplate{frame footer}{My custom footer}
}

% ------------------------------------------------------------------------------
% beamer doesn't have texttt defined, but I usually want it anyway
% ------------------------------------------------------------------------------
\let\textttorig\texttt
\renewcommand<>{\texttt}[1]{%
  \only#2{\textttorig{#1}}%
}

% ------------------------------------------------------------------------------
% minted
% ------------------------------------------------------------------------------


% ------------------------------------------------------------------------------
% tcolorbox / tcblisting
% ------------------------------------------------------------------------------
\usepackage{xcolor}
\definecolor{codecolor}{HTML}{FFC300}

\usepackage{tcolorbox}
\tcbuselibrary{most,listingsutf8,minted}

\tcbset{tcbox width=auto,left=1mm,top=1mm,bottom=1mm,
right=1mm,boxsep=1mm,middle=1pt}

\newtcblisting{myr}[1]{colback=codecolor!5,colframe=codecolor!80!black,listing only,
minted options={numbers=left, style=tcblatex,fontsize=\tiny,breaklines,autogobble,linenos,numbersep=3mm},
left=5mm,enhanced,
title=#1, fonttitle=\bfseries,
listing engine=minted,minted language=r}


% ------------------------------------------------------------------------------
% Listings
% ------------------------------------------------------------------------------
\definecolor{mygreen}{HTML}{37980D}
\definecolor{myblue}{HTML}{0D089F}
\definecolor{myred}{HTML}{98290D}

\usepackage{listings}

% the following is optional to configure custom highlighting
\lstdefinelanguage{XML}
{
  morestring=[b]",
  morecomment=[s]{<!--}{-->},
  morestring=[s]{>}{<},
  morekeywords={ref,xmlns,version,type,canonicalRef,metr,real,target}% list your attributes here
}

\lstdefinestyle{myxml}{
language=XML,
showspaces=false,
showtabs=false,
basicstyle=\ttfamily,
columns=fullflexible,
breaklines=true,
showstringspaces=false,
breakatwhitespace=true,
escapeinside={(*@}{@*)},
basicstyle=\color{mygreen}\ttfamily,%\footnotesize,
stringstyle=\color{myred},
commentstyle=\color{myblue}\upshape,
keywordstyle=\color{myblue}\bfseries,
}


% ------------------------------------------------------------------------------
% The Document
% ------------------------------------------------------------------------------
\title{Functional programming, Seminar No. 8}
\author{Danya Rogozin \\ Lomonosov Moscow State University, \\ Serokell O\"{U}}

\vspace{\baselineskip}

\date{Higher School of Economics \\ The Department of Computer Science}

\begin{document}

\maketitle

\begin{frame}
  \frametitle{Intro}

  On the previous seminar we
  \begin{itemize}
    \item studied such monads as \verb"IO", \verb"Reader", \verb"Writer", and \verb"State"
  \end{itemize}

  \onslide<2->{
  Today we
  \begin{itemize}
    \item investigate monad transformers as an uniform method of the effect combining
  \end{itemize}
  }
\end{frame}

\section{Alternative and MonadPlus classes}

\begin{frame}[fragile]
  \frametitle{Monoids}

  \begin{minted}{haskell}
  class Semigroup a => Monoid a where
    mempty :: a
    mappend :: a -> a -> a
  \end{minted}

  Some of monads are also monoids, e.g., the list data type

  \begin{minted}{haskell}
  instance Monoid [a] where
    mempty = []
    mappend = (++)
  \end{minted}
\end{frame}

\begin{frame}[fragile]
  \frametitle{Two versions of the \verb"Maybe" monoid}

  \begin{minted}{haskell}
  instance Monoid a => Monoid (Maybe a) where
    mempty = Just memty
    Nothing `mappend` _ = Nothing
    _ `mappend` Nothing = Nothing
    (Just a) `mappend` (Just b) = Just (a `mappend` b)
  \end{minted}

  \begin{minted}{haskell}
  instance Monoid (Maybe a) where
    mempty = Nothing
    Nothing `mappend` m = m
    m@(Just _) `mappend` _ = m
  \end{minted}
\end{frame}

\begin{frame}[fragile]
  \frametitle{The \verb"Alternative" class}

  The \verb"Alternative" class is a generalisation of the idea above:

  \begin{minted}{haskell}
    class Applicative f => Alternative (f :: * -> *) where
      empty :: f a
      (<|>) :: f a -> f a -> f a
      some :: f a -> f [a]
      many :: f a -> f [a]
  {-# MINIMAL empty, (<|>) #-}

  infixl 3 <|>
  \end{minted}
\end{frame}

\begin{frame}[fragile]
  \frametitle{The \verb"MonadPlus" class}

  The \verb"Alternative" class has an essential extension called \verb"MonadPlus":

  \begin{minted}{haskell}
  class (Alternative m, Monad m) => MonadPlus m where
    mzero :: m a
    mplus :: m a -> m a -> m a
  \end{minted}

\vspace{\baselineskip}

  This class should satisfy the following conditions:
  \begin{minted}{haskell}
  mzero >>= f  ==  mzero
  v >> mzero   ==  mzero
  \end{minted}
\end{frame}

\begin{frame}[fragile]
  \frametitle{The \verb"MonadPlus" uses}

  \begin{minted}{haskell}
  mfilter :: (MonadPlus m) => (a -> Bool) -> m a -> m a
  mfilter p ma = do
    a <- ma
    if p a then return a else mzero

  guard :: Alternative f => Bool -> f ()
  guard True = pure ()
  guard False =  empty

  when :: Applicative f => Bool -> f () -> f ()
  when p s  = if p then s else pure ()
  \end{minted}
\end{frame}

\section{Monad transformers}

\begin{frame}[fragile]
  \frametitle{How to compose Reader and Writer}

  \begin{minted}{haskell}
  foo :: RWS Int [Int] () Int
  foo i = do
    baseCounter <- ask
    let newCounter = baseCounter + i
    put [baseCounter, newCounter]
    return newCounter

    foo :: State (Int, [Int]) Int
    foo i = do
      x <- gets fst
      let xi = x + i
      put (x, [x, xi])
      return xi
  \end{minted}
\end{frame}

\begin{frame}[fragile]
  \frametitle{Maybe and IO}

The example of a monad composition is the \verb"MaybeIO" monad

  \begin{minted}{haskell}
  newtype MaybeIO a = MaybeIO { runMaybeIO :: IO (Maybe a) }

  instance Monad MaybeIO where
    return x = MaybeIO (return (Just x))
    MaybeIO action >>= f = MaybeIO $ do
      result <- action
        case result of
          Nothing -> return Nothing
          Just x  -> runMaybeIO (f x)
  \end{minted}
\end{frame}

\begin{frame}[fragile]
  \frametitle{The generalisation of the idea above}

  \begin{minted}{haskell}
  newtype MaybeT m a = MaybeT { runMaybeT :: m (Maybe a) }

  instance Monad m => Monad (MaybeT m) where
    return :: a -> MaybeT m a
    return x = MaybeT (return (Just x))

    (>>=) :: MaybeT m a -> (a -> MaybeT m b) -> MaybeT m b
    MaybeT action >>= f = MaybeT $ do
      result <- action
      case result of
        Nothing -> return Nothing
        Just x  -> runMaybeT (f x)
  \end{minted}
\end{frame}

\begin{frame}[fragile]
  \frametitle{The \verb"MonadTrans" class}

\begin{minted}{haskell}
  class MonadTrans (t :: (* -> *) -> * -> *) where
    -- | Lift a computation from
    -- | the argument monad to the constructed monad.
    lift :: (Monad m) => m a -> t m a
\end{minted}

\vspace{\baselineskip}

This class has the following laws:
\begin{minted}{haskell}
lift . return == return
lift (m >>= f) == lift m >>= (lift . f)
\end{minted}
\end{frame}

\begin{frame}[fragile]
  \frametitle{The MonadTrans instances}

  \begin{minted}{haskell}
  transformToMaybeT :: Functor m => m a -> MaybeT m a
  transformToMaybeT = error "homework"

  instance MonadTrans MaybeT where
    lift :: Monad m => m a -> MaybeT m a
    lift = transformToMaybeT
  \end{minted}
\end{frame}

\begin{frame}[fragile]
  \frametitle{The MaybeT example}

  \begin{minted}{haskell}
  emailIsValid :: String -> Bool
  emailIsValid email = '@' `elem` email

  askEmail :: MaybeT IO String
  askEmail = do
    lift $ putStrLn "Input your email, please:"
    email <- lift getLine
    guard $ emailIsValid email
    return email

   main :: IO ()
   main = do
     email <- askEmail
     case email of
       Nothing -> putStrLn "Wrong email."
       Just email' -> putStrLn email'
  \end{minted}
\end{frame}

\begin{frame}[fragile]
  \frametitle{The ReaderT monad}

  \begin{minted}{haskell}
  newtype ReaderT r m a = ReaderT { runReaderT :: r -> m a }

  type LoggerIO a = ReaderT LoggerName IO a

  logMessage :: Text -> LoggerIO ()

  readFileWithLog :: FilePath -> LoggerIO Text
  readFileWithLog path = do
    logMessage $ "Reading file: " <> T.pack (show path)
    lift $ readFile path
  \end{minted}
\end{frame}

\begin{frame}[fragile]
  \frametitle{The ReaderT monad}

  \begin{minted}{haskell}
  writeFileWithLog :: FilePath -> Text -> LoggerIO ()
  writeFileWithLog path content = do
      logMessage $ "Writing to file: " <> T.pack (show path)
      lift $ writeFile path content

  prettifyFileContent :: FilePath -> LoggerIO ()
  prettifyFileContent path = do
    content <- readFileWithLog path
    writeFileWithLog path (format content)

  main :: IO ()
  main = runReaderT (prettifyFileContent "foo.txt") (LoggerName "Application")
  \end{minted}
\end{frame}

\begin{frame}[fragile]
  \frametitle{The MonadReader class}

The class is defined the mtl-package

  \begin{minted}{haskell}
  class Monad m => MonadReader r m | m -> r where
    ask    :: m r
    local  :: (r -> r) -> m a -> m a
    reader :: (r -> a) -> m a

  instance MonadReader r m => MonadReader r (StateT s m) where
    ask    = lift ask
    local  = mapStateT . local
    reader = lift . reader
  \end{minted}
\end{frame}

\begin{frame}[fragile]
  \frametitle{The MonadError class}

  \begin{minted}{haskell}
  class (Monad m) => MonadError e m | m -> e where
    throwError :: e -> m a
    catchError :: m a -> (e -> m a) -> m a

  newtype ExceptT e m a =
    ExceptT { runExceptT :: m (Either e a) }

  runExceptT :: ExceptT e m a -> m (Either e a)

  withExceptT ::
    Functor m => (e -> e') -> ExceptT e m a -> ExceptT e' m a
  \end{minted}
\end{frame}

\begin{frame}[fragile]
  \frametitle{The MonadError class}

  \begin{minted}{haskell}
  foo :: MonadError FooError m => ...
  bar :: MonadError BarError m => ...
  baz :: MonadError BazError m => ...

  data BazError = BazFoo FooError | BazBar BarError

  baz = do
    withExcept BazFoo foo
    withExcept BazBar ba
  \end{minted}
\end{frame}

\end{document}
