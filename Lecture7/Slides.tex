\documentclass[10pt]{beamer}
\usepackage[cache=false]{minted}
\usepackage[utf8x]{inputenc}
\usepackage{hyperref}
\usepackage{fontawesome}
\usepackage{graphicx}
\usepackage[english,ngerman]{babel}
\usepackage{bussproofs}

% ------------------------------------------------------------------------------
% Use the beautiful metropolis beamer template
% ------------------------------------------------------------------------------
\usepackage[T1]{fontenc}
\usepackage{fontawesome}
\usepackage{FiraSans}
\newtheorem{defin}{Definition}
\newtheorem{theor}{Theorem}
\newtheorem{prop}{Proposition}
\mode<presentation>
{
  \usetheme[progressbar=foot,numbering=fraction,background=light]{metropolis}
  \usecolortheme{default} % or try albatross, beaver, crane, ...
  \usefonttheme{default}  % or try serif, structurebold, ...
  \setbeamertemplate{navigation symbols}{}
  \setbeamertemplate{caption}[numbered]
  %\setbeamertemplate{frame footer}{My custom footer}
}

% ------------------------------------------------------------------------------
% beamer doesn't have texttt defined, but I usually want it anyway
% ------------------------------------------------------------------------------
\let\textttorig\texttt
\renewcommand<>{\texttt}[1]{%
  \only#2{\textttorig{#1}}%
}

% ------------------------------------------------------------------------------
% minted
% ------------------------------------------------------------------------------


% ------------------------------------------------------------------------------
% tcolorbox / tcblisting
% ------------------------------------------------------------------------------
\usepackage{xcolor}
\definecolor{codecolor}{HTML}{FFC300}

\usepackage{tcolorbox}
\tcbuselibrary{most,listingsutf8,minted}

\tcbset{tcbox width=auto,left=1mm,top=1mm,bottom=1mm,
right=1mm,boxsep=1mm,middle=1pt}

\newtcblisting{myr}[1]{colback=codecolor!5,colframe=codecolor!80!black,listing only,
minted options={numbers=left, style=tcblatex,fontsize=\tiny,breaklines,autogobble,linenos,numbersep=3mm},
left=5mm,enhanced,
title=#1, fonttitle=\bfseries,
listing engine=minted,minted language=r}


% ------------------------------------------------------------------------------
% Listings
% ------------------------------------------------------------------------------
\definecolor{mygreen}{HTML}{37980D}
\definecolor{myblue}{HTML}{0D089F}
\definecolor{myred}{HTML}{98290D}

\usepackage{listings}

% the following is optional to configure custom highlighting
\lstdefinelanguage{XML}
{
  morestring=[b]",
  morecomment=[s]{<!--}{-->},
  morestring=[s]{>}{<},
  morekeywords={ref,xmlns,version,type,canonicalRef,metr,real,target}% list your attributes here
}

\lstdefinestyle{myxml}{
language=XML,
showspaces=false,
showtabs=false,
basicstyle=\ttfamily,
columns=fullflexible,
breaklines=true,
showstringspaces=false,
breakatwhitespace=true,
escapeinside={(*@}{@*)},
basicstyle=\color{mygreen}\ttfamily,%\footnotesize,
stringstyle=\color{myred},
commentstyle=\color{myblue}\upshape,
keywordstyle=\color{myblue}\bfseries,
}


% ------------------------------------------------------------------------------
% The Document
% ------------------------------------------------------------------------------
\title{Functional programming, Seminar No. 7}
\author{Danya Rogozin \\ Lomonosov Moscow State University, \\ Serokell O\"{U}}

\vspace{\baselineskip}

\date{Higher School of Economics \\ The Department of Computer Science}

\begin{document}

\maketitle

\begin{frame}
  \frametitle{Introduction}

  On the previous seminar, we
  \begin{itemize}
    \item got acquainted with the notion of a monad as a uniform interface for pipelines of actions
  \end{itemize}

\onslide<2->{
Today we
  \begin{itemize}
    \item study such monads as \verb"IO", \verb"Reader", \verb"Writer", and \verb"State"
  \end{itemize}
  }
\end{frame}

\section{The Input/Output Monad}

\begin{frame}[fragile]
  \frametitle{The problem of purity}

  \begin{itemize}
    \item The purity of functions is rather a problem than advantage if we deal with input and output
    \item If input/output functions were pure, then they would yield the same value at the same point, since their behaviour is fully determined by the referential transparency principle.
  \end{itemize}
\end{frame}

\begin{frame}[fragile]
  \frametitle{The problem of purity}
  \begin{itemize}
\item Input/output functions are clearly impure. Here are some example of such functions
\begin{minted}{haskell}
getChar :: IO Char
getLine :: IO String
\end{minted}
\item In fact, these functions have types:
\begin{minted}{haskell}
getChar :: RealWorld -> (RealWorld, Char)
getLine :: RealWorld -> (RealWorld, String)
\end{minted}
\onslide<2->{\item The philosophical question: what is \verb"RealWorld"?}
  \end{itemize}
\end{frame}

\begin{frame}[fragile]
  \frametitle{The approximate definition of \verb"IO"}
  \begin{itemize}
    \item The value of the \verb"IO a" type is such value that can perform input/output actions
    \item The approximate Haskell implementation:
    \begin{minted}{haskell}
    newtype IO a = IO (RealWorld -> (RealWorld, a))
    \end{minted}
    \item According to Hoogle, ``RealWorld is deeply magical. It is primitive... We never manipulate values of type RealWorld... it's only used in the type system''
    \item That is, an engineer has no access to the \verb"RealWorld" and we cannot use the same \verb"RealWorld" twice!
  \end{itemize}
\end{frame}

\begin{frame}[fragile]
\frametitle{The \verb"IO" as a Monad}

\begin{itemize}
\item The rough \verb"Monad" instance
\begin{minted}{haskell}
instance Monad IO where
   return x = IO $ \w -> (w, x)
   m >>= k = IO $ \w ->
     case m w of
       (w', a) -> k a w'
\end{minted}
\item A side effect of every action occurs only onces
\item The order of side effects is strictly determined
\end{itemize}
\end{frame}

\begin{frame}[fragile]
  \frametitle{The basic console input/output function}

  \begin{itemize}
    \item Input:
    \begin{minted}{haskell}
    getChar :: IO Char
    getLine :: IO String
    getContents :: IO String
    \end{minted}
    \item Output:
    \begin{minted}{haskell}
    putStrLn :: String -> IO ()
    print :: Show a => a -> IO ()
    \end{minted}
    \item Input/output:
    \begin{minted}{haskell}
    interact :: (String -> String) -> IO ()
    \end{minted}
  \end{itemize}
\end{frame}

\begin{frame}[fragile]
  \frametitle{The example of \verb"IO"}

  \begin{minted}{haskell}
  main :: IO ()
  main = do
    putStrLn "Hello, what is your name?"
    name <- getLine
    putStrLn $ "Hi, " ++ name
    putStrLn $
      "Now leave me alone, " ++ name ++
      ", I'm tired of you"
  \end{minted}
\end{frame}

\begin{frame}[fragile]
  \frametitle{The \verb"getLine" function closely}

  Let us take a look at the approximate \verb"getLine" implementation:

  \begin{minted}{haskell}
  getLine' :: IO String
  getLine' = do
    c <- getChar
    case c == '\n' of
      True -> return []
      False -> do
        cs <- getLine'
        return (c : cs)
  \end{minted}
\end{frame}

\begin{frame}[fragile]
  \frametitle{The \verb"putStr" functions closely}

  \begin{minted}{haskell}
  putStr' :: String -> IO ()
  putStr' [] = return ()
  putStr' (x : xs) = putChar x >> putStr' xs
  \end{minted}

  A more sophisticated version:

  \begin{minted}{haskell}
  sequence_ :: Monad m => [m a] -> m ()
  sequence_ = foldr (>>) (return ())

  putStr'' :: String -> IO ()
  putStr'' = sequence_ . map putChar
  \end{minted}
\end{frame}

\begin{frame}[fragile]
  \frametitle{The \verb"putStr" functions closely}

  A more sophisticated version of a more sophisticated version:
  \begin{minted}{haskell}
  sequence_ :: Monad m => [m a] -> m ()
  sequence_ = foldr (>>) (return ())

  mapM_ :: Monad m => (a -> m b) -> [a] -> m ()
  mapM_ f = sequence_ . map f

  putStr''' :: String -> IO ()
  putStr''' = mapM_ putChar
  \end{minted}
\end{frame}

\section{The Reader Monad}

\begin{frame}[fragile]
  The \verb"Reader" monad allows to read values from an environment
  \begin{minted}{haskell}
  newtype Reader r a = Reader { runReader :: (r -> a) }

  instance Monad (Reader r) where
    return = error "This is your homework"
    x >>= k = error "This is your homework"
  \end{minted}

  \begin{itemize}
    \item \verb"return" yields an argument ingoring a given environment
    \item \verb"(>>=)" passes a given environment to both computations
    \item The useful combinators:
    \begin{minted}{haskell}
    ask :: Reader r r
    local :: (r -> r) -> Reader r a -> Reader r a
    \end{minted}
  \end{itemize}
\end{frame}

\section{The Writer Monad}

\begin{frame}[fragile]
  \begin{itemize}
  \item The \verb"Writer" monad with the logging features
  \begin{minted}{haskell}
  newtype Writer w a = Writer { runWriter :: (a, w) }

  instance Monoid w => Monad (Writer w) where
    return = error "This is your homework"
    (>>=)  = error "This is your homework"
  \end{minted}
  \item The useful combinators:
  \begin{minted}{haskell}
  tell :: Monoid w => w -> Writer w ()
  listen :: Monoid w => Writer w a -> Writer w (w, a)
  pass :: Monoid w => Writer w (a, w -> w) -> Writer w a
  \end{minted}
  \end{itemize}
\end{frame}

\section{The State Monad}

\begin{frame}[fragile]
  \begin{itemize}
    \item The \verb"State" monad is a monad for a mutable state processing
  \begin{minted}{haskell}
  newtype State s a = State { runState :: s -> (a,s) }

  instance Monad (State s) where
    return = error "This is your homework"
    (>>=)  = error "This is your homework"
  \end{minted}
  \item The useful combinators:
  \begin{minted}{haskell}
  get :: State s s
  put :: s -> State s ()
  \end{minted}
\end{itemize}
\end{frame}

\end{document}
