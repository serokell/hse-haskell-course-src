\documentclass[10pt]{beamer}
\usepackage[cache=false]{minted}
\usepackage[utf8x]{inputenc}
\usepackage{hyperref}
\usepackage{fontawesome}
\usepackage{graphicx}
\usepackage[english,ngerman]{babel}
\usepackage{bussproofs}

% ------------------------------------------------------------------------------
% Use the beautiful metropolis beamer template
% ------------------------------------------------------------------------------
\usepackage[T1]{fontenc}
\usepackage{fontawesome}
\usepackage{FiraSans}
\newtheorem{defin}{Definition}
\newtheorem{theor}{Theorem}
\newtheorem{prop}{Proposition}
\mode<presentation>
{
  \usetheme[progressbar=foot,numbering=fraction,background=light]{metropolis}
  \usecolortheme{default} % or try albatross, beaver, crane, ...
  \usefonttheme{default}  % or try serif, structurebold, ...
  \setbeamertemplate{navigation symbols}{}
  \setbeamertemplate{caption}[numbered]
  %\setbeamertemplate{frame footer}{My custom footer}
}

% ------------------------------------------------------------------------------
% beamer doesn't have texttt defined, but I usually want it anyway
% ------------------------------------------------------------------------------
\let\textttorig\texttt
\renewcommand<>{\texttt}[1]{%
  \only#2{\textttorig{#1}}%
}

% ------------------------------------------------------------------------------
% minted
% ------------------------------------------------------------------------------


% ------------------------------------------------------------------------------
% tcolorbox / tcblisting
% ------------------------------------------------------------------------------
\usepackage{xcolor}
\definecolor{codecolor}{HTML}{FFC300}

\usepackage{tcolorbox}
\tcbuselibrary{most,listingsutf8,minted}

\tcbset{tcbox width=auto,left=1mm,top=1mm,bottom=1mm,
right=1mm,boxsep=1mm,middle=1pt}

\newtcblisting{myr}[1]{colback=codecolor!5,colframe=codecolor!80!black,listing only,
minted options={numbers=left, style=tcblatex,fontsize=\tiny,breaklines,autogobble,linenos,numbersep=3mm},
left=5mm,enhanced,
title=#1, fonttitle=\bfseries,
listing engine=minted,minted language=r}


% ------------------------------------------------------------------------------
% Listings
% ------------------------------------------------------------------------------
\definecolor{mygreen}{HTML}{37980D}
\definecolor{myblue}{HTML}{0D089F}
\definecolor{myred}{HTML}{98290D}

\usepackage{listings}

% the following is optional to configure custom highlighting
\lstdefinelanguage{XML}
{
  morestring=[b]",
  morecomment=[s]{<!--}{-->},
  morestring=[s]{>}{<},
  morekeywords={ref,xmlns,version,type,canonicalRef,metr,real,target}% list your attributes here
}

\lstdefinestyle{myxml}{
language=XML,
showspaces=false,
showtabs=false,
basicstyle=\ttfamily,
columns=fullflexible,
breaklines=true,
showstringspaces=false,
breakatwhitespace=true,
escapeinside={(*@}{@*)},
basicstyle=\color{mygreen}\ttfamily,%\footnotesize,
stringstyle=\color{myred},
commentstyle=\color{myblue}\upshape,
keywordstyle=\color{myblue}\bfseries,
}


% ------------------------------------------------------------------------------
% The Document
% ------------------------------------------------------------------------------
\title{Functional programming, Seminar No. 7}
\author{Danya Rogozin \\ Institute for Information Transmission Problems, RAS \\ Serokell O\"{U}}

\vspace{\baselineskip}

\date{Higher School of Economics \\ The Department of Computer Science}

\begin{document}

\maketitle

\begin{frame}
  \frametitle{Today}

We will study the following monads
\begin{center}
\includegraphics[scale=0.25]{puppies.png}
\end{center}
\end{frame}

\section{Input/Output}

\begin{frame}[fragile]
  \frametitle{The \verb"IO" monad, motivation}
  \begin{itemize}
\item \verb"IO a" is a type whose values are input/output actions that produce values of \verb"a"
\item Here are first examples of \verb"IO" functions
\begin{minted}{haskell}
getChar :: IO Char
getLine :: IO String
\end{minted}
\item In fact, these functions have the following types:
\begin{minted}{haskell}
getChar :: RealWorld -> (RealWorld, Char)
getLine :: RealWorld -> (RealWorld, String)
\end{minted}
\item A philosophical question: what is \verb"RealWorld"?
  \end{itemize}
\end{frame}

\begin{frame}[fragile]
  \frametitle{The approximate definition of \verb"IO"}
  \begin{itemize}
    \item \verb"IO" is defined approximately as follows:
    \begin{minted}{haskell}
    newtype IO a = IO (RealWorld -> (RealWorld, a))
    \end{minted}
    \item According to Hoogle, ``\verb"RealWorld" is deeply magical. It is primitive... We never manipulate values of type RealWorld... it's only used in the type system''
    \item That is, an engineer has no access to values of \verb"RealWorld" and we cannot use the same \verb"RealWorld" twice!
  \end{itemize}
\end{frame}

\begin{frame}[fragile]
\frametitle{The \verb"IO" as a Monad}

 The \verb"Monad" instance (very roughly):
\begin{minted}{haskell}
instance Monad IO where
   return x = IO $ \w -> (w, x)
   m >>= k = IO $ \w ->
     case m w of
       (w', a) -> k a w'
\end{minted}
\begin{itemize}
\item An effect of every action occurs only once.
\item Note that the order of effects matters!
\end{itemize}
\end{frame}

\begin{frame}[fragile]
  \frametitle{Basic console input/output functions}
  \begin{itemize}
    \item Input:
    \begin{minted}{haskell}
    getChar :: IO Char
    getLine, getContents :: IO String
    \end{minted}
    \item Output:
    \begin{minted}{haskell}
    putStrLn :: String -> IO ()
    print :: Show a => a -> IO ()
    \end{minted}
    \item Input/output:
    \begin{minted}{haskell}
    interact :: (String -> String) -> IO ()
    \end{minted}
  \end{itemize}
\end{frame}

\begin{frame}[fragile]
  \frametitle{An example of \verb"IO"}

  \begin{minted}{haskell}
  main :: IO ()
  main = do
    putStrLn "Hello, what is your name?"
    name <- getLine
    putStrLn $ "Hi, " ++ name
    putStrLn $
      "Gotta go, " ++ name ++
      ", have a nice day"
  \end{minted}
\end{frame}

\begin{frame}[fragile]
  \frametitle{The \verb"getLine" function closely}

  Let us take a look at the rough version of \verb"getLine":

  \begin{minted}{haskell}
  getLine' :: IO String
  getLine' = do
    c <- getChar
    case c == '\n' of
      True -> return []
      False -> do
        cs <- getLine'
        return (c : cs)
  \end{minted}
\end{frame}

\begin{frame}[fragile]
  \frametitle{The \verb"putStr" function}

  \begin{minted}{haskell}
  putStr' :: String -> IO ()
  putStr' [] = return ()
  putStr' (x : xs) = putChar x >> putStr' xs
  \end{minted}

  Using \verb"sequence_":

  \begin{minted}{haskell}
  sequence_ :: Monad m => [m a] -> m ()
  sequence_ = foldr (>>) (return ())

  putStr'' :: String -> IO ()
  putStr'' = sequence_ . map putChar
  \end{minted}
\end{frame}

\begin{frame}[fragile]
  \frametitle{The \verb"putStr" function}

  Using \verb"sequence_" and \verb"mapM_":
  \begin{minted}{haskell}
  sequence_ :: Monad m => [m a] -> m ()
  sequence_ = foldr (>>) (return ())

  mapM_ :: Monad m => (a -> m b) -> [a] -> m ()
  mapM_ f = sequence_ . map f

  putStr''' :: String -> IO ()
  putStr''' = mapM_ putChar
  \end{minted}
\end{frame}

\section{Reader, Writer, and State}

\begin{frame}[fragile]
  \frametitle{Reader}
  The \verb"Reader" monad allows one to read values from an environment:
  \begin{minted}{haskell}
  newtype Reader r a = Reader { runReader :: (r -> a) }

  instance Monad (Reader r) where
    -- return :: a -> Reader r a
    return x = Reader $ const x

    -- (>>=) :: Reader r a -> (a -> Reader r b) -> Reader r b
    Reader f >>= k = Reader $ \e -> let v = runReader m e
                                     in runReader (k v) e
  \end{minted}

  \begin{itemize}
    \item Here \verb"(>>=)" passes a given environment to both computations
    \item Useful functions:
    \begin{minted}{haskell}
    ask   :: Reader e e
    asks  :: (e -> a) -> Reader e a
    local :: (e -> b) -> Reader b a -> Reader e a
    \end{minted}
  \end{itemize}
\end{frame}

\begin{frame}[fragile]
  \frametitle{Reader. An example}
  \begin{minted}{haskell}
  data Environment = Environment { ids  :: [Int]
                                 , name :: Int -> String
                                 , near :: Int -> (Int, Int) }

  inEnv :: Int -> Reader Environment Bool
  inEnv i = asks (elem i . ids)

  anyInEnv :: (Int, Int) -> Reader Environment Bool
  anyInEnv (i, j) = inEnv i ||^ inEnv j

  checkNeighbours :: Int -> Reader Environment (Maybe String)
  checkNeighbours i =
    asks (`near` i) >>= \pair ->
    anyInEnv pair   >>= \res  ->
    if res
    then Just <$> asks (`name` i)
    else pure Nothing
  \end{minted}
\end{frame}

\begin{frame}[fragile]
  \frametitle{Writer}
  \begin{itemize}
  \item The \verb"Writer" monad for computation with logs
  \begin{minted}{haskell}
  newtype Writer w a = Writer { runWriter :: (a, w) }

  instance Monoid w => Monad (Writer w) where
    -- return :: a -> Writer w a
    return x = Writer (x, mempty)

    -- (>>=)
    --   :: Writer r a -> (a -> Writer r b) -> Writer r b
    Writer (x,v) >>= f = let (Writer (y, v')) = f x
                          in Writer (y, v `mappend` v')
  \end{minted}
  \item The useful combinators:
  \begin{minted}{haskell}
  tell :: Monoid w => w -> Writer w ()
  listen :: Monoid w => Writer w a -> Writer w (w, a)
  pass :: Monoid w => Writer w (a, w -> w) -> Writer w a
  execWriter :: Writer w a -> w
  \end{minted}
  \end{itemize}
\end{frame}

\begin{frame}[fragile]
  \frametitle{Writer. An example}
  \begin{minted}{haskell}
  binPow :: Int -> Int -> Writer String Int
  binPow 0 _      = return 1
  binPow n a
    | even n    = binPow (n `div` 2) a >>= \b ->
                  Writer (b * b, "Square " ++ show b ++ "\n")
    | otherwise =
        binPow (n - 1) a >>= \b ->
        Writer
          (a * b, "Mul " ++ show a ++ " and " ++ show b ++ "\n")
  \end{minted}
\end{frame}

\begin{frame}[fragile]
  \frametitle{State}
  \begin{itemize}
    \item The \verb"State" monad for processing of mutable states
  \begin{minted}{haskell}
  newtype State s a = State { runState :: s -> (a,s) }

  instance Monad (State s) where
    -- return :: a -> State s a
    return x = State $ \s -> (x, s)

    -- (>>=) :: State s a -> (a -> State s b) -> State s b
    State act >>= f = State $ \s ->
      let (a, s') = act s
      in runState (k a) s'
  \end{minted}
  \item Useful functions:
  \begin{minted}{haskell}
  get       :: State s s
  put       :: s -> State s ()
  modify    :: (s -> s) -> State s ()
  gets      :: (s -> a) -> State s a
  withState :: (s -> s) -> State s a -> State s a
  evalState :: State s a -> s -> a
  execState :: State s a -> s -> s
  \end{minted}
\end{itemize}
\end{frame}

\begin{frame}[fragile]
  \frametitle{State. An example}
  \begin{minted}{haskell}
  type Stack = [Int]

  pop :: State Stack Int
  pop = State $ \(x:xs) -> (x, xs)

  push :: Int -> State Stack ()
  push x = State $ \xs -> ((), x:xs)

  stackOps :: State Stack Int
  stackOps = pop >>= \x -> push 5 >> push 10 >> return x
  \end{minted}
\end{frame}

\end{document}
